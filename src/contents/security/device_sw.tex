\subsection{Software Attacks on Peripherals}
\label{sec:device_attacks}

Threat models for secure architectures generally only consider software attacks
that directly target other components in the software stack running on the CPU.
This assumption results in security arguments with the very desirable property
of not depending on implementation details, such as the structure of the
motherboard hosting the processor chip.

The threat models mentioned above must classify attacks from other motherboard
components as physical attacks. Unfortunately, these models would mis-classify
all the attacks described in this sub-section, which can be carried out solely
by executing software on the victim processor. The incorrect classification
matters in cloud computing scenarios, where physical attacks are significantly
more expensive than software attacks.

Virtually all motherboards store the firmware used to boot the computer in a
flash memory chip that can be written by system software. This implementation
strategy provides an inexpensive avenue for deploying firmware bug fixes. At
the same time, an attack that compromises the system software can subvert the
firmware update mechanism to inject malicious code into the firmware. The
malicious code can be used to carry out a cold boot attack, which is typically
considered a physical attack. Furthermore, malicious firmware can modify the
system software as it is loaded during the boot process, giving the attacker
capabilities that have traditionally been associated with DRAM bus tapping
attacks.

The Intel Management Engine (ME)~\cite{ruan2014intelme} loads its firmware
from the same flash memory chip as the main computer, which opens up the
possibility of compromising its firmware. Due to its vast management
capabilities (\S~\ref{sec:intel_me}), a compromised ME would bring most of the
powers that come with installing an active probe on the DRAM bus, the PCI bus,
the System Management bus (SMBus).

To prevent these attacks, most of the firware is signed. For example, the ME
checks that its firmware was signed by a burned-in Intel public key. However,
both the computer firmware checks \cite{wojtczuk2010bios, furtak2014bios} and
the ME firmware checks \cite{tereshkin2009amt} have been subverted in the past.
Furthermore, \cite{ververis2010security} found many high-level security issues
in the ME's flagship Active Management Technology (AMT) application.


The SGX trusted computing base includes the processor package, and excludes the
other hardware in the computer. It follows that SGX must be able to fend off
attacks from rogue devices, such as the PCIe NIC used to compromise Intel TXT
\cite{wojtczuk2011txt}.


the rowhammer DRAM bit-flipping attack
\cite{kim2014rowhammer, google2015rowhammer, gruss2015rowhammer}.


The SGX threat model explicitly considers SMM to be untrusted. However, it does
not account for malicious code running on the Management Engine. Unfortunately,
the ME, PCH and DMI are Intel-proprietary and largely undocumented, so we
cannot assess the impact of an ME attack on software running inside an SGX
enclave.

The SDM and (TODO: Optimization manual? uncore manuals?) describe a vast array
of temperature sensors that can be used to indirectly measure the power
consumption of various components of an Intel CPU.

