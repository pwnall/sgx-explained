\subsection{Memory Mapping Attacks}
\label{sec:memory_mapping_attacks}



Figure~\ref{fig:sgx_mapping_attack} shows a hypothetical memory mapping attack.
Understanding this type of attack
greatly increases one's ability to reason about SGX's security.

\begin{figure}[hbt]
  \centering
  \includegraphics[width=85mm]{figures/sgx_mapping_attack.pdf}
  \caption{
    An example of a memory mapping attack, which is prevented by SGX. The
    enclave's author intends to disclose a piece of sensitive information only
    when a security check passes. Malicious system software maps the virtual
    address of the procedure called when the security check fails to an EPC
    page that contains the procedure that discloses the sensitive information,
    which is supposed to be called when the security check passes.
  }
  \label{fig:sgx_mapping_attack}
\end{figure}

For simplicity, we assume an enclave that performs a security check to decide
whether to disclose some sensitive information. Depending on the security
check's outcome, the enclave code either calls a \texttt{errorOut} procedure,
or a \texttt{disclose} procedure. We furthermore assume that each procedure's
code starts at a 4KB boundary, and takes up less than 4KB, so each procedure
fits in an EPC page. These requirements seem unrealistic, but the underlying
attack remains an issue in real applications.

In a memory mapping attack, malicious system software sets up the page tables
or EPT in such a way that the virtual address intended to store the
\texttt{errorOut} procedure is actually mapped to an EPC page that contains the
\texttt{disclose} procedure. Without any security measures in place, the
enclave would execute the \texttt{disclose} code and reveal sensitive
information, even though the security check fails.

The SGX security mechanisms, explained throughout the rest of this paper,
prevent enclave code execution if a memory mapping attack occurs. Therefore,
SGX prevents malicious system software from directly obtaining sensitive
information via memory mapping attacks.


% TODO: Figure out how to mention MSR's attacks that collect page fault
%       addresses. \cite{xu2015pagefaults}


