\section{Overview}
\label{sec:intro}

Intel's Software Guard Extensions (SGX) is the latest iteration in a long line
of designs that aim to offer the ability of executing software on a computer
owned by an untrusted party, while at the same time providing some guarantees
about the privacy and integrity of the computation.

SGX implements a \textit{remote attestation model} similar to its predecessors,
TPM~\cite{grawrock2003tpm} and TXT~\cite{grawrock2009txt}. A system that is not
under the computation owner's control produces an attestation assuring the
computation owner that a protected environment (called \textit{enclave} in
Intel's literature) was set up on an SGX-enabled processor, according to the
computation owner's expectations. The attestation also includes information
used to set up a secure communication channel with the software running inside
the enclave.

SGX stands out from its predecessors by the size of the code that the
attestation covers. The attestations produced by the original TPM design
covered all the software running on a computer, and TXT attestations covered
the code inside a VMX \cite{uhlig2005vmx} virtual machine. In SGX, an enclave
only contains the private data in a computation, and the code that operates on
it.

For example, in an application that performs image processing on confidential
medical images, the enclave would contain the code for decrypting images,
the image processing algorithm, and the code for encrypting the results,
whereas the code for reading and writing encrypted data would be left outside
the enclave.
