\subsection{Caching and Memory-Mapped Devices}
\label{sec:memory_io}

The Intel architecture provides a few methods for specifying the cache behavior
for various ranges of the memory address space (\S~\ref{sec:address_spaces}).
This is used for memory mapped to devices, such as a graphics unit's
framebuffer. Caching framebuffer memory is indesirable, because the delay
between the time when a write is issued and the time when the corresponding
cache lines are evicted and written back to memory could lead to corrupted
images on the user's display. This section covers the cacheability mechanisms
presented in the SDM, because understanding them is important for analyzing
SGX.

\subsubsection{Caching Behaviors}

Recent Intel CPUs implement the following caching behaviors.

\textit{Uncacheable} (UC) addresses cause reads and writes to be issued in the
order in which instructions are executed.


Some parts of the memory address space are used to communicate with devices
other than DRAM (\S~\ref{sec:address_spaces}). Two good examples are the
mapping of the SPI flash memory that holds boot firmware
(\S~\ref{sec:motherboard}), and the mapping of a graphics unit's framebuffer.
In the former case, caching is desirable, whereas in the latter case.

To mitigate against these issues,


This section reviews the cacheability controls

is often mapped
into DRAM.
