\HeadingLevelA{SGX Programming Model}
\label{sec:sgx_model}

The central concept of SGX is the \textit{enclave}, a protected environment
that contains the code and data pertaining to a security-sensitive computation.

SGX-enabled processors provide trusted computing by isolating each enclave's
environment from the untrusted software outside the enclave, and by
implementing a software attestation scheme that allows a remote party to
authenticate the software running inside an enclave. SGX's isolation mechanisms
are intended to protect the privacy and integrity of the computation performed
inside an enclave from attacks coming from malicious software executing on the
same computer, as well as from a limited set of physical attacks.

This section summarizes the SGX concepts that make up a mental model which is
sufficient for programmers to author SGX enclaves and to add SGX support to
existing system software. All the information in this section is backed up by
Intel's Software Developer Manual (SDM). The following section builds on the
concepts introduced here to fill in some of the missing pieces in the manual,
and analyzes some of SGX's security properties.


\subsection{SGX Physical Memory Organization}
\label{sec:sgx_prm}

% Intel SGX Resource Enumeration Leaves: SDM S 37.7.2
% Interactions with DMA: SDM S 42.10, SGX2 S 6.10

The enclaves' code and data is stored in \textit{Processor Reserved Memory}
(PRM), which is a subset of DRAM that cannot be directly accessed by other
software, including system software and SMM code. The CPU's integrated memory
controllers (\S~\ref{sec:cpu_die}) also reject DMA transfers targeting the PRM,
thus protecting it from access by other peripherals.

% EPC and Management of EPC Pages: SGX2 S 3.5
% Interactions with Memory Configuration: SGX2 S 6.11
% Memory Type Considerations for PRMRR: SGX2 S 6.11.1
% Interactions of PRMRR with Physical Memory Accesses: SGX2 S 6.11.3.1

The PRM is a continuous range of memory whose bounds are configured using a
base and a mask register with the same semantics as a variable memory type
range~(\S~\ref{sec:cacheability_config}). Therefore, the PRM's size must be an
integer power of two, and its start address must be aligned to the same power
of two. Due to these restrictions, checking if an address belongs to the PRM
can be done very cheaply in hardware, using the circuit outlined in
\S~\ref{sec:cacheability_config}.

The SDM does not describe the PRM and the PRM range registers (PRMRR). These
concepts are documented in the SGX
manuals~\cite{intel2013sgxmanual, intel2014sgx2manual} and in one of the SGX
papers~\cite{mckeen2013sgx}. Curiously, although the SDM generally mirrors the
SGX manuals, it lacks most of the sections that mention the PRM.


\subsubsection{The Enclave Page Cache (EPC)}
\label{sec:sgx_epc}

% Enclave Page Cache: SDM S 37.5
% EPC and Management of EPC Pages: SDM S 39.5, S 39.5.1

The contents of enclaves and the associated data structures are stored in the
\textit{Enclave Page Cache} (EPC), which is a subset of the PRM, as shown in
Figure~\ref{fig:sgx_epc}.

\begin{figure}[hbt]
  \centering
  \includegraphics[width=87mm]{figures/sgx_epc.pdf}
  \caption{
    Enclave data is stored into the EPC, which is a subset of the PRM. The
    PRM is a contiguous range of DRAM that cannot be accessed by system
    software or peripherals.
  }
  \label{fig:sgx_epc}
\end{figure}

The SGX design supports having multiple enclaves on a system at the same time,
which is a necessity in multi-process environments. This is achieved by having
the EPC split into 4~KB pages that can be assigned to different enclaves. The
EPC uses the same page size as the architecture's address translation feature
(\S~\ref{sec:paging}). This is not a coincidence, as future sections will
reveal that the SGX implementation is tightly coupled with the address
translation implementation.

The EPC is managed by the same system software that manages the rest of the
computer's physical memory. The system software, which can be a hypervisor or
an OS kernel, uses SGX instructions to allocate unused pages to enclaves, and
to free previously allocated EPC pages. The system software is expected to
expose enclave creation and management services to application software.

Non-enclave software cannot directly access the EPC, as it is contained in the
PRM. This restriction plays a key role in SGX's enclave isolation guarantees,
but creates an obstacle when the system software needs to load the initial code
and data into a newly created enclave. The SGX design solves this problem by
having the instructions that allocate an EPC page to an enclave also initialize
the page. Most EPC pages are initialized by copying data from a non-PRM memory
page.


\subsubsection{The Enclave Page Cache Map (EPCM)}
\label{sec:sgx_epcm}

% Enclave Page Cache Map (EPCM): SDM S 37.5.1, SDM S 38.19
% SECINFO.FLAGS: SDM S 38.11.1
% PAGE_TYPE Field Definition: SDM S 38.11.2

The SGX design expects the system software to allocate the EPC pages to
enclaves. However, as the system software is not trusted, SGX processors check
the correctness of the system software's allocation decisions, and refuse to
perform any action that would compromise SGX's security guarantees. For
example, if the system software attempts to allocate the same EPC page to two
enclaves, the SGX instruction used to perform the allocation will fail.

In order to perform its security checks, SGX records some information about the
system software's allocation decisions for each EPC page in the
\textit{Enclave Page Cache Map}~(EPCM). The EPCM is an array with one entry
per EPC page, so computing the address of a page's EPCM entry only requires a
bitwise shift operation and an addition.

The EPCM's contents is only used by SGX's security checks. Under normal
operation, the EPCM does not generate any software-visible behavior, and
enclave authors and system software developers can mostly ignore it.
Therefore, the SDM only describes the EPCM at a very high level, listing the
information contained within and noting that the EPCM is ``trusted memory''.
The SDM does not disclose the storage medium or memory layout used by the EPCM.

The EPCM uses the information in Table~\ref{fig:sgx_epcm_ownership_fields} to
track the ownership of each EPC page. We defer a full discussion of the EPCM to
a later section, because its contents is intimiately coupled with all of SGX's
features, which will be described over the next few sections.

\begin{table}[hbt]
  \centering
  \begin{tabularx}{\columnwidth}{| l | r | X |}
  \hline
  \textbf{Field} & \textbf{Bits} & \textbf{Description}\\
  \hline
  VALID & 1 & 0 for un-allocated EPC pages \\
  \hline
  PT & 8 & page type \\
  \hline
  ENCLAVESECS & 52 & identifies the enclave owning the page \\
  \hline
  \end{tabularx}
  \caption{
    The fields in an EPCM entry that track the ownership of pages.
  }
  \label{fig:sgx_epcm_ownership_fields}
\end{table}

The SGX instructions that allocate an EPC page set the VALID bit of the
corresponding EPCM entry to 1, and refuse to operate on EPC pages whose VALID
bit is already set.

The instruction used to allocate an EPC page also determines the page's
intended usage, which is recorded in the \textit{page type} (PT) field of the
corresponding EPCM entry. The pages that store an enclave's code and data are
considered to have a \textit{regular} type (PT\_REG in the SDM). The pages
dedicated to the storage of SGX's supporting data structures are tagged with
special types. For example, the PT\_SECS type identifies pages that hold SGX
Enclave Control Structures, which will be described in the following section.
The other EPC page types will be described in future sections.

Last, a page's EPCM entry also identifies the enclave that owns the EPC page.
This information is used by the mechanisms that enforce SGX's isolation
guarantees to prevent an enclave from accessing another enclave's private
information. As the EPCM identifies a single owning enclave for each EPC page,
it is impossible for enclaves to communicate via shared memory using EPC pages.
Fortunately, enclaves can share untrusted non-EPC memory, as will be discussed
in \S~\ref{sec:sgx_paging}.


\subsubsection{The SGX Enclave Control Structure (SECS)}
\label{sec:sgx_secs}

% Data Structures and Enclave Operation: SDM S 37.4
% SGX Enclave Control Structure (SECS): SDM S 38.7, S 38.7.1

SGX stores per-enclave metadata in a
\textit{SGX Enclave Control Structure}~(SECS) associated with each enclave.
Each SECS is stored in a dedicated EPC page with the page type PT\_SECS. These
pages are not intended to be mapped into any enclave's address space, and are
exclusively used by the CPU's SGX implementation.

% Constructing an Enclave: SDM S 39.1
% ECREATE: SDM S 39.1.1, S 41.3
% Implicit vs. Explicit accesses: SDM S 38.5.3
% Implicit accesses: SDM S 38.5.3.2

An enclave's identity is almost synonymous to its SECS. The first step in
bringing an enclave to life allocates an EPC page to serve as the enclave's
SECS, and the last step in destroying an enclave deallocates the page holding
its SECS. The EPCM entry field identifying the enclave that owns an EPC page
points to the enclave's SECS. The system software uses the virtual address of
an enclave's SECS to identify the enclave when invoking SGX instructions.

% Access Control Requirements: SDM S 38.3

All SGX instructions take virtual addresses as their inputs. Given that SGX
instructions use SECS addresses to identify enclaves, the system software must
create entries in its page tables pointing to the SECS of the enclaves it
manages. However, the system software cannot access any SECS page, as these
pages are stored in the PRM. SECS pages are not intended to be mapped inside
their enclaves' virtual address spaces, and SGX-enabled processors explictly
prevent enclave code from accessing SECS pages.

% SGX Enclave Control Structure (SECS): SDM S 38.7, S 38.7.1

This seemingly arbitrary limitation is in place so that the SGX implementation
can store sensitive information in the SECS, and be able to assume that no
potentially malicious software will access that information. For example, the
SDM states that each enclave's measurement is stored in its SECS. If software
would be able to modify an enclave's measurement, SGX's software attestation
scheme would provide no security assurances.

The SECS is strongly coupled with many of SGX's features. Therefore, the pieces
of information that make up the SECS will be gradually introduced as the
different aspects of SGX are described.

\subsection{The Memory Layout of an SGX Enclave}
\label{sec:sgx_enclave_layout}

SGX was designed to minimize the effort required to convert application code to
take advantage of enclaves. History suggests this is a wise decision, as a
large factor in the continued dominance of the Intel architecture is its
ability to maintain backward compatibility. To this end, SGX enclaves were
designed to be conceptually similar to the leading software modularization
construct, dynamically loaded libraries, which are packaged as \texttt{.so}
files on Unix, and \texttt{.dll} files on Windows.

For simplicity, we describe the interaction between enclaves and non-enclave
software assuming that each enclave is used by exactly one application process,
which we shall refer to as the enclave's \textit{host process}. We do note,
however, that the SGX design does not explicitly prohibit having an enclave be
shared by multiple application processes.


\subsubsection{The Enclave Linear Address Range (ELRANGE)}
\label{sec:sgx_elrange}

% SGX Enclave Control Structure (SECS): SDM S 38.7

Each enclave designates an area in its virtual address space, called the
\textit{enclave linear address range} (ELRANGE), which is used to map the code
and the sensitive data stored in the enclave's EPC pages. The virtual address
space outside ELRANGE is mapped to access non-EPC memory via the same virtual
addresses as the enclave's host process, as shown in
Figure~\ref{fig:sgx_elrange}.

\begin{figure}[hbt]
  \centering
  \includegraphics[width=85mm]{figures/sgx_elrange.pdf}
  \caption{
    An enclave's EPC pages are accessed using a dedicated region in the
    enclave's virtual address space, called ELRANGE. The rest of the virtual
    address space is used to access the memory of the host process. The memory
    mappings are established using the page tables managed by system software.
  }
  \label{fig:sgx_elrange}
\end{figure}

The SGX design guarantees that the enclave's memory accesses inside ELRANGE
obey the virtual memory abstraction~(\S~\ref{sec:paging_concepts}), while
memory accesses outside ELRANGE receive no guarantees. Therefore, enclaves must
store all their code and private data inside ELRANGE, and must consider the
memory outside ELRANGE to be an untrusted interface to the outside world.

The word ``linear'' in ELRANGE references the linear addresses produced by the
vestigial segmentation feature~(\S~\ref{sec:segments}) in the 64-bit Intel
architecture. For most purposes, ``linear'' can be treated as a synonym for
``virtual''.

ELRANGE is specified using a base (the BASEADDR field) and a size (the SIZE)
in the enclave's SECS~(\S~\ref{sec:sgx_secs}). ELRANGE must meet the same
constraints as a variable memory type range (\S~\ref{sec:cacheability_config})
and as the PRM range~(\S~\ref{sec:sgx_prm}), namely the size must be a power of
2, and the base must be aligned to the size. These restrictions are in place
so that checking whether an address belongs to an enclave's ELRANGE can be done
very inexpensively in either hardware~(\S~\ref{sec:cacheability_config}) or
software.

When an enclave represents a dynamic library, it is natural to have ELRANGE be
set to the memory range reserved for the library by the loader. The ability to
access non-enclave memory from enclave code makes it easy to reuse existing
library code that expects to work with pointers to memory buffers managed by
code in the host process.


\subsubsection{SGX Enclave Attributes}
\label{sec:sgx_secs_attributes}

The execution environment of an enclave is heavily influenced by the value of
the ATTRIBUTES field in the enclave's SECS~(\ref{sec:sgx_secs}). The rest of
this work will refer to the field's sub-fields, shown in
Table~\ref{fig:sgx_secs_attributes}), as \textit{enclave attributes}.

% ATTRIBUTES: SDM S 37.8.1

\begin{table}[hbt]
  \centering
  \begin{tabularx}{\columnwidth}{| l | r | X |}
  \hline
  \textbf{Field} & \textbf{Bits} & \textbf{Description} \\
  \hline
  DEBUG & 1 & Opts into enclave debugging features. \\
  \hline
  XFRM & 64 & The value of XCR0~(\S~\ref{sec:registers}) while this enclave's
              code is executed. \\
  \hline
  MODE64BIT & 1 & Set for 64-bit enclaves. \\
  \hline
  \end{tabularx}
  \caption{
    An enclave's attributes are the sub-fields in the ATTRIBUTES field of the
    enclave's SECS. This table shows a subset of the attributes defined in the
    SGX documentation.
  }
  \label{fig:sgx_secs_attributes}
\end{table}

The most important attribute, from a security perspective, is the DEBUG flag.
When this flag is set, it enables the use of SGX's debugging features for this
enclave. These debugging features include the ability to read and modify most
of the enclave's memory. Therefore, DEBUG should only be set in a development
environment, as it causes the enclave to lose all the SGX security guarantees.

SGX guarantees that enclave code will always run with the XCR0
register~(\S~\ref{sec:registers}) set to the value indicated by
\textit{extended features request mask}~(XFRM). Enclave authors are expected to
use XFRM to specify the set of architectural extensions enabled by the compiler
used to produce the enclave's code. Having XFRM be explicitly specified allows
Intel to design new architectural extensions that change the semantics of
existing instructions, such as Memory Protection Extensions (MPX), without
having to worry about the security implications on enclave code that was
developed without an awareness of the new features.

The MODE64BIT flag is set to true for enclaves that use the 64-bit Intel
architecture. From a security standpoint, this flag should not even exist, as
supporting a secondary architecture adds unnecessary complexity to the SGX
implementation, and increases the probability that security vulnerabilities
will creep in. In the interest of mental sanity, this work does not analyze the
behavior of SGX for enclaves whose MODE64BIT flag is cleared. However, a
security researcher that wishes to find vulnerabilities in SGX might wish to
study this area.

Last, the INIT flag is always false when the enclave's SECS is created. The
flag is set to true at a certain point in the enclave lifecycle, which will be
summarized in \S~\ref{sec:sgx_enclave_lifecycle}.


\subsubsection{Address Translation for SGX Enclaves}
\label{sec:sgx_paging}

% Access Control Requirements: SDM S 38.3
% Interactions with VMX: SDM S 42.5, S 42.5.{1,2,3,4,5}

Under SGX, the operating system and hypervisor are still in full control of the
page tables and EPTs, and each enclave's code uses the same address translation
process and page tables~(\S~\ref{sec:paging}) as its host application. This
minimizes the amount of changes required to add SGX support to existing system
software. At the same time, having the page tables managed by untrusted system
software opens up SGX to the address translation attacks described in
\S~\ref{sec:address_translation_attacks}. As future sections will reveal, a
good amount of the complexity in SGX's design can be attributed to the need to
prevent these attacks.

SGX's active memory mapping attacks defense mechanisms revolve around ensuring
that each EPC page can only be mapped at a specific virtual
address~(\S~\ref{sec:segments}). When an EPC page is allocated, its intended
virtual address is recorded in the EPCM entry for the page, in the ADDRESS
field.

When an address translation (\S~\ref{sec:paging}) result is the physical
address of an EPC page, the CPU ensures\footnote{A mismatch triggers a general
protection fault (\#GP, \S~\ref{sec:faults}).} that the virtual address given
to the address translation process matches the expected virtual address
recorded in the page's EPCM entry.

SGX also prevents against some passive memory mapping attacks and fault
injection attacks by ensuring that the access permissions of each EPC page
always match the enclave author's intentions. The access permissions for each
EPC page are specified when the page is allocated, and recorded in the
\textit{readable}~(R), \textit{writable}~(W), and \textit{executable}~(X)
fields in the page's EPCM entry, shown in
Table~\ref{fig:sgx_epcm_access_fields}.

\begin{table}[hbt]
  \centering
  \begin{tabularx}{\columnwidth}{| l | r | X |}
  \hline
  \textbf{Field} & \textbf{Bits} & \textbf{Description}\\
  \hline
  ADDRESS & 48 & the virtual address used to access this page\\
  \hline
  R & 1 & allow reads by enclave code\\
  \hline
  W & 1 & allow writes by enclave code\\
  \hline
  X & 1 & allow execution of code inside the page, inside enclave\\
  \hline
  \end{tabularx}
  \caption{
    The fields in an EPCM entry that indicate the enclave's intended virtual
    memory layout.
  }
  \label{fig:sgx_epcm_access_fields}
\end{table}

When an address translation (\S~\ref{sec:paging}) resolves into an EPC page,
the corresponding EPCM entry's fields override the access permission attributes
(\S~\ref{sec:page_table_attributes}) specified in the page tables. For example,
the W field in the EPCM entry overrides the writable (W) attribute, and the X
field overrides the disable execution (XD) attribute.

It follows that an enclave author must include memory layout information along
with the enclave, in such a way that the system software loading the enclave
will know the expected virtual memory address and access permissions for each
enclave page. In return, the SGX design guarantees to the enclave authors that
the system software, which manages the page tables and EPT, will not be able to
set up an enclave's virtual address space in a manner that is inconsistent with
the author's expectations.

The \texttt{.so} and \texttt{.dll} file formats, which are SGX's intended
enclave delivery vehicles, already have provisions for specifying the virtual
addresses that a software module was designed to use, as well as the desired
access permissions for each of the module's memory areas.

Last, a SGX-enabled CPU will ensure that the virtual memory inside
ELRANGE~(\S~\ref{sec:sgx_elrange}) is mapped to EPC pages. This prevents the
system software from carrying out an address translation attack where it maps
the enclave's entire virtual address space to DRAM pages outside the PRM, which
do not trigger any of the checks above, and can be directly accessed by the
system software.


\subsubsection{The Thread Control Structure (TCS)}
\label{sec:sgx_tcs}

% Thread Control Structure (TCS): SDM S 38.8, S 38.8.{1,2,3,4}

The SGX design fully embraces multi-core processors. It is possible for
multiple logical processors~(\S~\ref{sec:cpu_die}) to concurrently execute the
same enclave's code at the same time, via different threads.

The SGX implementation uses a \textit{Thread Control Structure}~(TCS) for each
logical processor that executes an enclave's code. It follows that an enclave's
author must provision at least as many TCS instances as the maximum number of
concurrent threads that the enclave is intended to support.

Each TCS is stored in a dedicated EPC page whose EPCM entry type is PT\_TCS.
The SDM describes the first few fields in the TCS. These fields are considered
to belong to the architectural part of the structure, and therefore are
guaranteed to have the same semantics on all the processors that support SGX.
The rest of the TCS is not documented.

% Access Control Requirements: SDM S 38.3
% EDBGRD, EDBGWR: SDM S 41.3

The contents of an EPC page that holds a TCS cannot be directly accessed, even
by the code of the enclave who owns the TCS. This restriction is similar to the
restriction on accessing EPC pages holding SECS instances. However, the
architectural fields in a TCS can be read by enclave debugging instructions.

The architectural fields in the TCS layout  the context
switches~(\S~\ref{sec:registers}) performed by a logical processor when it
transitions between executing non-enclave and enclave code.

For example, the OENTRY field specified the value loaded in the instruction
pointer (RIP) when the TCS is used to start executing enclave code, so the
enclave author has strict control over the entry points available to enclave's
host application. Furthermore, the OFSBASGX and OFSBASGX fields specify the
base addresses loaded in the FS and GS segment
registers~(\S~\ref{sec:segments}), which typically point to Thread Local
Storage (TLS).


\subsubsection{The State Save Area (SSA)}
\label{sec:sgx_ssa}

% Interactions with the Processor Extended State and Miscellaneous State:
%     SDM S 42.7
% Requirements and Architecture Overview: SDM S 42.7.1

% State Save Area (SSA) Frame: SDM S 38.9

When the processor encounters a hardware exception~(\S~\ref{sec:faults}), such
as an interrupt~(\S~\ref{sec:interrupts}), while executing the code inside an
enclave, it performs a privilege level switch (\S~\ref{sec:faults}) and invokes
a hardware exception handler provided by the system software. Before executing
the exception handler, however, the processor needs a secure area to store the
enclave code's execution context~(\S~\ref{sec:registers}), so that the
information in the execution context is not revealed to the untrusted system
software.

% Relevant Fields in Various Data Structures: SDM S 42.7.2
% SECS.SSAFRAMESIZE: SDM S 42.7.2.2

In the SGX design, the area used to store an enclave thread's execution context
while a hardware exception is handled is called a \texttt{State Save
Area}~(SSA), illustrated in Figure~\ref{fig:sgx_enclave_layout}. Each TCS
references a contiguous sequence of SSAs. The \textit{offset of the SSA
array}~(OSSA) field specifies the location of the first SSA in the enclave's
virtual address space. The \textit{number of SSAs}~(NSSA) field indicates the
number of available SSAs.

\begin{figure}[hbt]
  \centering
  \includegraphics[width=85mm]{figures/sgx_enclave_layout.pdf}
  \caption{
    A possible layout of an enclave's virtual address space. Each enclave has a
    SECS, and one TCS per supported concurrent thread. Each TCS points to a
    sequence of SSAs, and specifies initial values for RIP and for the base
    addresses of FS and GS.
  }
  \label{fig:sgx_enclave_layout}
\end{figure}

Each SSA starts at the beginning of an EPC page, and uses up the number of EPC
pages that is specified in the SSAFRAMESIZE field of the enclave's SECS. These
alignment and size restrictions most likely simplify the SGX implementation by
reducing the number of special cases that it needs to handle.

% Relevant Fields in Various Data Structures: SDM S 42.7.2
% SECS.ATTRIBUTES.XFRM: SDM S 42.7.2.1

An enclave thread's execution context consists of the general-purpose registers
(GPRs) and the result of the XSAVE instruction~(\S~\ref{sec:registers}).
Therefore, the size of the execution context depends on the requested-feature
bitmap~(RFBM) used by to XSAVE. All the code in an enclave uses the same RFBM,
which is declared in the XFRM enclave attribute~\ref{sec:sgx_secs_attributes}.
The number of EPC pages reserved for each SSA, specified in SSAFRAMESIZE,
must\footnote{\texttt{ECREATE}~(\S~\ref{sec:sgx_ecreate}) fails if SSAFRAMESIZE
is too small.} be large enough to fit the XSAVE output for the feature bitmap
specified by XFRM.

SSAs are stored in regular EPC pages, whose EPCM page type is PT\_REG.
Therefore, the SSA contents is accessible to enclave software. The SSA layout
is architectural, and is completely documented in the SDM. This opens up
possibilities for an enclave exception handler that is invoked by the host
application after a hardware execption occurs, and acts upon the information in
a SSA.

\subsection{The Life Cycle of an SGX Enclave}
\label{sec:sgx_enclave_lifecycle}

An enclave's life cycle is deeply intertwined with resource management,
specifically the allocation of EPC pages. Therefore, the instructions that
transition between different life cycle states can only be executed by the
system software. The system software is expected to expose the SGX instructions
described below as enclave loading and teardown services.

The following subsections describe the major steps in an enclave's lifecycle,
which is illustrated by Figure~\ref{fig:sgx_enclave_lifecycle}.

\begin{figure}[hbt]
  \centering
  \includegraphics[width=85mm]{figures/sgx_enclave_lifecycle.pdf}
  \caption{
    The SGX enclave life cycle management instructions and state transition
    diagram
  }
  \label{fig:sgx_enclave_lifecycle}
\end{figure}


% Enclave Entry and Exiting : SDM S 39.2, S 39.2.1
% ECREATE, EADD, EREMOVE: SDM S 41.3

\subsubsection{Creation}
\label{sec:sgx_ecreate}

An enclave is born when the system software issues the \texttt{ECREATE}
instruction, which turns a free EPC page into the SECS~(\S~\ref{sec:sgx_secs})
for the new enclave.

\texttt{ECREATE} initializes the newly created SECS using the information in a
non-EPC page owned by the system software. This page specifies the values for
all the SECS fields defined in the SDM, such as BASEADDR and SIZE, using an
architectural layout that is guaranteed to be preserved by future
implementations.

While is very likely that the actual SECS layout used by initial SGX
implementations matches the architectural layout quite closely, future
implementations are free to deviate from this layout, as long as they maintain
the ability to initialize the SECS using the architectural layout.
Software cannot access an EPC page that holds a SECS, so it cannot become
dependent on an internal SECS layout. This is a stronger version of the
encapsulation used in the Virtual Machine Control
Structure~(VMCS,~\S~\ref{sec:vmx}).

\texttt{ECREATE} validates the information used to initialize the SECS, and
results in a page fault~(\#PF,~\S~\ref{sec:faults}) or general protection
fault~(\#GP,~\S~\ref{sec:faults}) if the information is not valid. For example,
if the SIZE field is not a power of two, \texttt{ECREATE} results in \#GP. This
validation, combined with the fact that the SECS is not accessible by software,
simplifies the implementation of the other SGX instructions, which can assume
that the information inside the SECS is valid.

Last, \texttt{ECREATE} initializes the enclave's INIT attribute (sub-field of
the ATTRIBUTES field in the enclave's SECS, \S~\ref{sec:sgx_secs_attributes})
to the false value. The enclave's code cannot be executed until the INIT
attribute is set to true, which happens in the initialization stage that will
be described in \S~\ref{sec:sgx_einit_overview}.


\subsubsection{Loading}
\label{sec:sgx_eadd}

\texttt{ECREATE} marks the newly created SECS as \textit{uninitialized}. While
an enclave's SECS is in this state, the system software can use \texttt{EADD}
instructions to load the initial code and data into the enclave. \texttt{EADD}
is used to create both TCS pages (\S~\ref{sec:sgx_tcs}) and regular pages.

% Page Information (PAGEINFO): SDM S 38.10
% Security Information (SECINFO): SDM S 38.11

\texttt{EADD} reads its input data from a \textit{Page Information}~(PAGEINFO)
structure, illustrated in Figure~\ref{fig:sgx_pageinfo}. The structure's
contents are only used to communicate information to the SGX implementation, so
it is entirely architectural and documented in the SDM.

\begin{figure}[hbt]
  \centering
  \includegraphics[width=65mm]{figures/sgx_pageinfo.pdf}
  \caption{
    The PAGEINFO structure supplies input data to SGX instructions such as
    \texttt{EADD}.
  }
  \label{fig:sgx_pageinfo}
\end{figure}

Currently, the PAGEINFO structure contains the virtual address of the EPC page
that will be allocated (LINADDR), the virtual address of the non-EPC page whose
contents will be copied into the newly allocated EPC page (SRCPGE), a virtual
address that resolves to the SECS of the enclave that will own the page (SECS),
and values for some of the fields of the EPCM entry associated with the newly
allocated EPC page (SECINFO).

The SECINFO field in the PAGEINFO structure is actually a virtual memory
address, and points to a \textit{Security Information}~(SECINFO) structure,
some of which is also illustrated in Figure~\ref{fig:sgx_pageinfo}. The SECINFO
structure contains the newly allocated EPC page's access permissions (R, W, X)
and its EPCM page type (PT\_REG or PT\_TCS). Like PAGEINFO, the SECINFO
structure is solely used to communicate data to the SGX implementation, so its
contents are also entirely architectural. However, most of the structure's
64 bytes are reserved for future use.

Both the PAGEINFO and the SECINFO structures are prepared by the system
software that invokes the \texttt{EADD} instruction, and therefore must be
contained in non-EPC pages. Both structures must be aligned to their sizes --
PAGEINFO is 32 bytes long, so each PAGEINFO instance must be 32-byte aligned,
while SECINFO has 64 bytes, and therefore each SECINFO instance must be
64-byte aligned. The alignment requirements likely simplify the SGX
implementation by reducing the number of special cases that must be handled.

\texttt{EADD} validates its inputs before modifying the newly allocated EPC
page or its EPCM entry. Most importantly, attempting to \texttt{EADD} a page to
an enclave whose SECS is in the initialized state will result in a \#GP.
Furthermore, attempting to \texttt{EADD} an EPC page that is already allocated
(the VALID field in its EPCM entry is 1) results in a \#PF. \texttt{EADD} also
ensures that the page's virtual address falls within the enclave's ELRANGE, and
that all the reserved fields in SECINFO are set to zero.

While loading an enclave, the system software will also use the
\texttt{EEXTEND} instruction, which updates the enclave's measurement used in
the software attestation process. Software attestation is discussed in
\S~\ref{sec:sgx_attestation}.


\subsubsection{Initialization}
\label{sec:sgx_einit_overview}

% EINIT Token Structure (EINITTOKEN): SDM S 38.14

After loading the initial code and data pages into the enclave, the system
software must use a \textit{Launch Enclave}~(LE) to obtain an EINIT Token
Structure, via an under-documented process that will be described in more
detail in \S~\ref{sec:sgx_launch_enclave}. The token is then provided to the
\texttt{EINIT} instruction, which marks the enclave's SECS as
\textit{initialized}.

The LE is a privileged enclave provided by Intel, and \textbf{is a prerequisite
for the use of enclaves authored by parties other than Intel}. The LE is an
SGX enclave, so it must be created, loaded and initialized using the processes
described in this section. However, the LE is cryptographically
signed~(\S~\ref{sec:integrity_crypto}) with a special Intel key that is
hard-coded into the SGX implementation, and that causes \texttt{EINIT} to
initialize the LE  without checking for a valid EINIT Token Structure.

When \texttt{EINIT} completes successfully, it sets the enclave's INIT
attribute to true. This opens the way for ring 3~(\S~\ref{sec:rings})
application software to execute the enclave's code, using the SGX instructions
described in \S~\ref{sec:sgx_threads}. On the other hand, once INIT is set to
true, \texttt{EADD} cannot be invoked on that enclave anymore, so the system
software must load all the pages that make up the enclave's initial state
before executing the \texttt{EINIT} instruction.


\subsubsection{Teardown}
\label{sec:sgx_eremove}

After the enclave has done the computation it was designed to perform, the
system software executes the \texttt{EREMOVE} instruction to deallocate the
EPC pages used by the enclave.

\texttt{EREMOVE} marks an EPC page as available by setting the VALID field of
the page's EPCM entry to 0 (zero). Before freeing up the page, \texttt{EREMOVE}
makes sure that there is no logical processor executing code inside the enclave
that owns the page to be removed.

An enclave is completely destroyed when the EPC page holding its SECS is freed.
\texttt{EREMOVE} refuses to deallocate a SECS page if it is referenced by any
other EPCM entry's ENCLAVESECS field, so an enclave's SECS page can only be
deallocated after all the enclave's pages have been deallocated.

\subsection{The Life Cycle of an SGX Thread}
\label{sec:sgx_threads}

Between the time when an enclave is
initialized~(\S~\ref{sec:sgx_einit_overview}) and the time when it is torn
down~(\S~\ref{sec:sgx_eremove}), the enclave's code can be executed by any
application process that has the enclave's EPC pages mapped into its virtual
address space.

% Internal CREGs: SDM S 41.1.4
% Access Control Requirements: SDM S 38.3

When executing the code inside an enclave, a logical processor is said to be
\textit{in enclave mode}, and the code that it executes can access the
regular~(PT\_REG,~\S~\ref{sec:sgx_epcm}) EPC pages that belong to the currently
executing enclave. When a logical process is outside enclave mode, it bounces
any memory accesses inside the Processor Reserved Memory
range~(PRM,~\S~\ref{sec:sgx_prm}), which includes the EPC.

Each logical processor that executes enclave code uses a Thread Control
Structure~(TCS,~\S~\ref{sec:sgx_tcs}). When a TCS is used by a logical
processor, it is said to be \textit{busy}, and it cannot be used by any other
logical processor.  Figure~\ref{fig:sgx_tcs_lifecycle} illustrates the
instructions used by a host process to execute enclave code and their
interactions with the TCS that they target.

\begin{figure}[hbt]
  \centering
  \includegraphics[width=75mm]{figures/sgx_tcs_lifecycle.pdf}
  \caption{
    The stages of the life cycle of an SGX Thread Control Structure (TCS) that
    has two State Save Areas (SSAs).
  }
  \label{fig:sgx_tcs_lifecycle}
\end{figure}


\subsubsection{Synchronous Enclave Entry and Exit}
\label{sec:sgx_enclave_mode}
\label{sec:sgx_eenter}
\label{sec:sgx_eexit}

Assuming that no hardware exception occurs, an enclave's host process uses the
\texttt{EENTER} instruction to execute enclave code. When the enclave code
finishes performing its task, it uses the \texttt{EEXIT} instruction to return
the execution control to the host process that invoked it.

% ENCLU - Execute an Enclave User Function of Specified Leaf Number: SDM S 41.2

\texttt{EENTER}, illustrated in Figure~\ref{fig:sgx_eenter} can only be
executed by unprivileged application software running at ring
3~(\S~\ref{sec:rings}), and results in an undefined instruction (\#UD) fault if
is executed by system software.

\begin{figure}[hbt]
  \centering
  \includegraphics[width=87mm]{figures/sgx_eenter.pdf}
  \caption{
    Data flow diagram for a subset of the logic in \texttt{EENTER}. The figure
    omits the logic for disabling debugging features, such as hardware
    breakpoints and performance monitoring events.
  }
  \label{fig:sgx_eenter}
\end{figure}

\texttt{EENTER} switches the logical processor to enclave mode, but does not
perform a privilege level switch~(\S~\ref{sec:faults}). Therefore, enclave code
always executes at ring 3, with the same privileges as the application code
that calls it. This makes it possible for an infrastructure owner to allow
user-supplied software to create and use enclaves, while having the assurance
that the OS kernel and hypervisor can still protect the infrastructure from
buggy or malicious software.

% EENTER - Enters an Enclave: SDM S 41.4.1
% Current State Save Area Frame (CSSA): SDM S 38.8.3
% Number of State Save Area Frames (NSSA): SDM S 38.8.4

\texttt{EENTER} takes the virtual address of a TCS as its input, and requires
that the TCS is \textit{available} (not busy), and that at least one State Save
Area~(SSA,~\S~\ref{sec:sgx_ssa}) is available in the TCS. The latter check is
implemented by making sure that the \textit{current SSA index}~(CSSA) field in
the TCS is less than the number of SSAs (NSSA) field. The SSA indicated by CSSA
is used in the event that a hardware exception occurs while enclave code is
executed, which shall be discussed in the following section.

\texttt{EENTER} transitions the logical processor into enclave mode, and sets
the instruction pointer (RIP) to the value indicated by the \textit{entry point
offset}~(OENTRY) field in the TCS that it receives. \textit{EENTER} is used by
an untrusted caller to execute code in a protected environment, and therefore
has the same security considerations as
\texttt{SYSCALL}~(\S~\ref{sec:privilege_switches}, which is used to call into
system software. Setting RIP to the value indicated by OENTRY guarantees to the
enclave author that the enclave code will only be invoked at well defined
points, and prevents a malicious host application from bypassing any security
checks that the enclave author may perform.

% Interactions with the Processor Extended State and Misc State: SDM S 42.7
% SECS.ATTRIBUTES.XFRM: SDM S 42.7.2.1

\texttt{EENTER} also sets XCR0~(\S~\ref{sec:registers}), the register that
controls which extended architectural features are in use, to the value of the
XFRM field in the SECS of the enclave that owns the TCS. Ensuring that XCR0 is
set according to the enclave author's intentions prevents a malicious operating
system from bypassing an enclave's security by enabling architectural features
that the enclave is not prepared to handle. This lets Intel introduce
extensions that change the architectural behavior of existing instructions,
such as Memory Protection Extensions (MPX), without having to worry about
introducing security vulnerabilities in SGX enclaves.

Last, \texttt{EENTER} loads the bases of the segment
registers~(\S~\ref{sec:segments}) FS and GS using values specified in the TCS.
The segments' selectors and types are hard-coded to safe values for ring 3 data
segments. This aspect of the SGX design makes it easy to implement per-thread
Thread Local Storage (TLS). For 64-bit enclaves, this is a convenience feature
rather than a security measure, as enclave code can securely load new bases
into FS and GS using the \texttt{WRFSBASE} and \texttt{WRGSBASE} instructions.

The \texttt{EENTER} implementation backs up the old values of the registers
that it modifies, so they can be restored when the enclave finishes its
computation. Just like \texttt{SYSCALL}, \texttt{EEENTER} saves the address of
the following instruction in the RCX register.

Interestingly, the SDM states that the old values of the XCR0, FS and GS
registers are saved in new registers dedicated to the SGX implementation.
However, given that they will only be used on an enclave exit, we expect that
the registers are saved in DRAM, in the reserved area in the TCS.

Like \texttt{SYSCALL}, \texttt{EENTER} does not modify the stack pointer
register (RSP). To avoid any security exploits, enclave code should set RSP to
point to a stack area that is entirely contained in EPC pages. Multi-threaded
enclaves can easily implement per-thread stack areas by setting up each
thread's TLS area to include a pointer to the thread's stack, and by setting
RSP to the value obtained by reading the TLS area pointed by the FS or GS
segment.

\texttt{EEXIT} can only be executed while the logical processor is in enclave
mode, and results in an undefined instruction (\#UD) fault if is executed in
any other circumstances. In a nutshell, the instruction returns the processor
to ring 3 outside enclave mode and restores the registers saved by
\texttt{EENTER}, which were described above.

Unlike \texttt{SYSRET}, \texttt{EEXIT} sets RIP to the value read from RBX,
after exiting enclave mode. This is inconsistent with \texttt{EENTER}, which
saves the RIP value to RCX. Unless this inconsistency stems from an error in
the SDM, enclave code must be sure to note the difference.

The SDM explicitly states that \texttt{EEXIT} does not modify most registers,
so enclave authors must make sure to clear any secrets stored in the
processor's registers before returning control to the host process.
Furthermore, enclave software will most likely cause a fault in its caller if
it doesn't restore the stack pointer RSP and the stack frame base pointer RBP
to the values that they had when \texttt{EENTER} was called.

It may seem unfortunate that enclave code can induce faults in its caller.
For better or for worse, this perfectly matches the case where an application
calls into a dynamically loaded module. More specifically, the module's code is
also responsible for preserving stack-related registers, and a buggy module
might jump anywhere in the application code of the host process.

At a first glance, it may seem elegant to have \texttt{EENTER} store the
contents of the XCR0, FS, and GS registers in the SSA, and have \texttt{EEXIT}
restore them from the SSA. However, this approach would break the Intel
architecture's guarantees that only system software can modify XCR0, and
application software can only load segment registers using selectors that index
into the GDT or LDT set up by system software (\S~\ref{sec:segments}).
Specifically, a malicious application could modify these privileged registers
by creating an enclave that writes the desired values to the SSA locations
backing up the registers, and then executes \texttt{EEXIT}.


\subsubsection{Asynchronous Enclave Exit}
\label{sec:sgx_aex}
\label{sec:sgx_eresume}



\subsection{EPC Page Eviction}
\label{sec:sgx_epc_eviction}

Modern OS kernels take advantage of address translation~(\S~\ref{sec:paging})
to implement page swapping, also referred to as paging~(\S~\ref{sec:paging}).
In a nutshell, paging allows the OS kernel to over-commit the computer's DRAM
by evicting rarely used memory pages to a slower storage medium that is
referred to as the disk.

Paging is a key contributor to utilizing a computer's resources effectively.
For example, a desktop system whose user runs multiple programs concurrently
can evict memory pages allocated to inactive applications without a significant
degradation in user experience.

Unfortunately, the OS cannot be allowed to evict an enclave's EPC pages via the
same methods that are used to implement page swapping for DRAM memory outside
the PRM range. In the SGX threat model, enclaves do not trust the system
software, so the SGX design offers an EPC page eviction method that can defend
against a malicious OS that attempts any of the active address translation
attacks described in \S~\ref{sec:address_translation_attacks}.

The price of the security afforded by SGX is that an OS kernel that supports
evicting EPC pages must use a modified page swapping implementation that
interacts with the SGX mechanisms. Enclave authors can mostly ignore EPC
evictions, similarly to how today's application developers can ignore the OS
kernel's paging implementation.

As illustrated in Figure~\ref{fig:sgx_page_eviction}, SGX supports evicting
EPC pages to DRAM pages outside the PRM range. The system software is expected
to use its existing page swapping implementation to evict the contents of these
pages out of DRAM and onto a disk.

\begin{figure}[hbt]
  \centering
  \includegraphics[width=85mm]{figures/sgx_page_eviction.pdf}
  \caption{
    SGX offers a method for the OS to evict EPC pages into non-PRM DRAM. The
    OS can then use its standard paging feature to evict the pages out of DRAM.
  }
  \label{fig:sgx_page_eviction}
\end{figure}

SGX's eviction feature revolves around the \texttt{EWB} instruction, described
in detail in \S~\ref{sec:sgx_ewb}. Essentially, \texttt{EWB} evicts an EPC page
into a DRAM page outside the EPC and marks the EPC page as available, by
zeroing the VALID field in the page's EPCM entry.

The SGX design relies on symmetric key cryptograpy~\ref{sec:crypto_keys} to
guarantee the privacy and integrity of the evicted EPC pages, and on
nonces~(\S~\ref{sec:freshness_crypto}) to guarantee the freshness of the pages
brought back into the EPC. These nonces are stored in Version Arrays~(VAs),
covered in \S~\ref{sec:sgx_va}, which are EPC pages dedicated to nonce storage.

Before an EPC page is evicted and freed up for use by other enclaves, the SGX
implementation must ensure that no TLB has address translations associated with
the evicted page, in order to avoid the TLB-based address translation attack
described in \S~\ref{sec:tlb_mapping_attacks}.

As explained in \S~\ref{sec:sgx_epc}, SGX leaves the system software in charge
of managing the EPC. It naturally follows that the SGX instructions described
in this section, which are used to implement EPC paging, are only available to
system software, which runs at ring 0~\S~\ref{sec:rings}.

In today's software stacks~(\S~\ref{sec:rings}), only the OS kernel implements
page swapping in order to support the over-committing of DRAM. The hypervisor
is only used to partition the computer's physical resources between operating
systems. Therefore, this section is written with the expectation that the OS
kernel will also take on the responsibility of EPC page swapping. For
simplicity, we often use the term ``OS kernel'' instead of ``system software''.
The reader should be aware that the SGX design does not preclude a system where
the hypervisor implements its own EPC page swapping. Therefore, ``OS kernel''
should really be read as ``the system software that performs EPC paging''.


\subsubsection{Page Eviction and the TLBs}
\label{sec:sgx_eblock}

One of the least promoted accomplishments of SGX is that it does not add any
security checks to the memory execution units~(\S~\ref{sec:cpu_core},
\S~\ref{sec:out_of_order}). Instead, SGX's access control checks occur after an
address translation~(\S~\ref{sec:paging}) is performed, right before the
translation result is written into the TLBs~(\S~\ref{sec:tlbs}). This aspect
is generally downplayed throughout the SDM, but it becomes visible when
explaining SGX's EPC page eviction mechanism.

A full discussion of SGX's access control checks merits its own section, and is
defered to \S~\ref{sec:xxx}. The EPC page eviction mechanisms can be explained
using only two requirements from SGX's security model. First, when a logical
processor exits an enclave, either via
\texttt{EEXIT}~(\S~\ref{sec:sgx_eexit}) or via an AEX~(\S~\ref{sec:sgx_aex}),
its TLBs are flushed. Second, when an EPC page is deallocated from an enclave,
all logical processors executing that enclave's code must be directed to exit
the enclave. This is sufficient to guarantee the removal of any TLB entry
targeting the deallocated EPC.

System software can cause a logical processor to exit an enclave by sending it
an Inter-Processor Interrupt (IPI,~\S~\ref{sec:interrupts}), which will trigger
an AEX when received. Essentially, this is a very coarse-grained TLB shootdown.

SGX does not trust system software. Therefore, before marking an EPC page's
EPCM entry as free, the SGX implementation must ensure that the OS kernel has
flushed all the TLBs that might contain translations for the page. Furthermore,
performing IPIs and TLB flushes for each page eviction would add a significant
overhead to a paging implementation, so the SGX design allows a batch of pages
to be evicted using a single IPI / TLB flush sequence.

% Enclave Page Cache Map (EPCM): SDM S 38.19

The TLB flush verification logic relies on a 1-bit EPCM entry field called
BLOCKED. As shown in Figure~\ref{fig:sgx_page_states}, the VALID and BLOCKED
fields yield three possible EPC page states. A page is \textit{free} when both
bits are zero, \textit{in use} when VALID is zero and BLOCKED is one, and
\textit{blocked} when both bits are one.

\begin{figure}[hbt]
  \centering
  \includegraphics[width=80mm]{figures/sgx_page_states.pdf}
  \caption{
    The VALID and BLOCKED bits in an EPC page's EPCM entry can be in one of
    three states. \texttt{EADD} and its siblings allocate new EPC pages.
    \texttt{EREMOVE} permanently deallocates an EPC page. \texttt{EBLOCK}
    blocks an EPC page so it can be evicted using \texttt{EWB}.\texttt{ELDB}
    and \texttt{ELDU} load an evicted page back into the EPC.
  }
  \label{fig:sgx_page_states}
\end{figure}

Blocked pages are not considered accessible to enclaves. If an address
translation results in a blocked EPC page, the SGX implementation causes the
translation to result in a Page Fault~(\#PF,~\S~\ref{sec:faults}). This
guarantees that once a page is blocked, the CPU will not create any new TLB
entries pointing to it.

Furthermore, every SGX instruction makes sure that the EPC pages that it
operates on are not blocked. For example, \texttt{EENTER} ensures that the TCS
it is given is not blocked, that its enclave's SECS is not blocked, and that
every page in the current SSA is not blocked.

% Eviction of Enclave Pages: SDM S 39.5.3

In order to evict a batch of EPC pages, the OS kernel must first issue
\texttt{EBLOCK} instructions targeting them. The OS is also expected to remove
the EPC page's mapping from page tables, but is not trusted to do so.

After all the desired pages have been blocked, the OS kernel must execute an
\texttt{ETRACK} instruction, which directs the SGX implementation to keep track
of which logical processors have had their TLBs flushed. \texttt{ETRACK}
requires the virtual address of an enclave's SECS~(\S~\ref{sec:sgx_secs}). If
the OS wishes to evict a batch of EPC pages belonging to multiple enclaves, it
must issue an \texttt{ETRACK} for each enclave.

Following the \texttt{ETRACK} instructions, the OS kernel must induce enclave
exits on all the logical processors that are executing code inside the enclaves
that have been \texttt{ETRACK}ed. The SGX design expects that the OS will use
IPIs to cause AEXes in the logical processors whose TLBs must be flushed.

The EPC page eviction process is completed when the OS executes an \texttt{EWB}
instruction for each EPC page to be evicted. This instruction, which will be
fully described in \S~\ref{sec:sgx_ewb}, writes an encrypted version of the EPC
page to be evicted into DRAM, and then frees the page by clearing the VALID and
BLOCKED bits in its EPCM entry. Before carrying out its tasks, \texttt{EWB}
ensures that the EPC page that it targets has been blocked, and checks the
state set up by \texttt{ETRACK} to make sure that all the relevant TLBs have
been flushed.

An evicted page can be loaded back into the EPC via the \texttt{ELDU} and
\texttt{ELDB} instructions. Both instructions start up with a free EPC page and
a DRAM page that has the evicted contents of an EPC page, decrypt the DRAM
page's contents into the EPC page, and restore the corresponding EPCM entry.
The only difference between \texttt{ELDU} and \texttt{ELDB} is that the latter
sets the BLOCKED bit in the page's EPCM entry, whereas the former leaves it
cleared.

\texttt{ELDU} and \texttt{ELDB} resemble \texttt{ECREATE} and \texttt{EADD},
in the sense that they populate a free EPC page. Since the page that they
operate on was free, the SGX security model predicates that no TLB entries can
possibly target it. Therefore, these instructions do not require a mechanism
similar to \texttt{EBLOCK} or \texttt{ETRACK}.


\subsubsection{The Version Array (VA)}
\label{sec:sgx_va}
\label{sec:sgx_epa}

% Version Array (VA): SDM S 38.18
% EPC and Management of EPC Pages: SDM S 39.5, 39.5.{2,3,4,5,6}

When \texttt{EWB} evicts the contents of an EPC, it creates an 8-byte
nonce~(\S~\ref{sec:freshness_crypto}) that Intel's documentation calls a
\textit{page version}. SGX's freshness guarantees are built on the assumption
that nonces are stored securely, so \texttt{EWB} stores the nonce that it
creates inside a \textit{Version Array}~(VA).

Version Arrays are EPC pages that are dedicated to storing nonces generated by
EWB. Each VA is divided into slots, and each slot is exactly large enough to
store one nonce. Given that the size of an EPC page is 4KB, and each nonce
occupies 8 bytes, it follows that each VA has 512 slots.

% EPA: SDM S 41.3

VA pages are allocated using the \texttt{EPA} instruction, which takes in the
virtual address of a free EPC page, and turns it into a Version Array with
empty slots. VA pages are identified by the PT\_VA type in their EPCM entries.
Like SECS pages, VA pages have the ENCLAVEADDRESS fields in their EPCM entries
set to zero, and cannot be accessed directly by any software, including
enclaves.

% EBLOCK, EREMOVE: SDM S 41.3

Unlike the other page types discussed so far, VA pages are not associated with
any enclave. This means they can be deallocated via \texttt{EREMOVE} without
any restriction. However, freeing up a VA page whose slots are in use
effectively discards the nonces in those slots, which results in losing the
ability to load the corresponding evicted pages back into the EPC. Therefore,
it is unlikely that a correct OS implementation will ever call \texttt{EREMOVE}
on a VA with non-free slots.

% EPA, EWB: SDM S 41.3

According to the pseudo-code for \texttt{EPA} and \texttt{EWB} in the SDM, SGX
uses the zero value to represent the free slots in a VA, implying that all the
generated nonces have to be non-zero. This also means that \texttt{EPA}
initializes a VA simply by zeroing the underlying EPC page. However, since
software cannot access a VA's contents, neither the use of a special value, nor
the value itself is architectural.


\subsubsection{Enclave IDs}
\label{sec:sgx_eid}

The \texttt{EWB} and \texttt{ELDU} / \texttt{ELDB} instructions use an
\textit{enclave ID}~(EID) to identify the enclave that owns an evicted page.
The EID has the same purpose as the ENCLAVESECS~(\S~\ref{sec:sgx_epcm}) field
in an EPCM entry, which is also used to identify the enclave that owns an EPC
page. This section explains the need for having two values represent the same
concept by comparing the two values and their uses.

% Enclave Page Cache Map (EPCM): SDM S 37.5.1, SDM S 38.19

The SDM states that ENCLAVESECS field in an EPCM entry is used to identify the
SECS of the enclave owning the associated EPC page, but stops short of
describing its format. SGX instructions never expose the value of the
ENCLAVESECS field to software so, in theory, its representation can even change
between SGX implementations.

% ENCLAVESECS compared with CR_ACTIVE_SECS
% EENTER: TMP_SECS <- address of SECS for TCS; CR_ACTIVE_SECS <- TMP_SECS

However, we will later argue that the most plausible representation for the
ENCLAVESECS field is the physical address of the enclave's SECS. Therefore, the
ENCLAVESECS value associated with a given enclave will change if the enclave's
SECS is evicted from the EPC and loaded back at a different location. It
follows that the ENCLAVESECS value is only suitable for identifying an enclave
while its SECS remains in the EPC.

% Internal CREGs: SDM S 41.1.4
% ECREATE: SDM S 41.3

According to the SDM, the EID field is a 64-bit field stored in an enclave's
SECS. \texttt{ECREATE}'s pseudocode in the SDM reveals that an enclave's ID is
generated when the SECS is allocated, by atomically incrementing a global
counter. Assuming that the counter does not roll over\footnote{A 64-bit counter
incremented at 4Ghz would roll over in slightly more than 136 years}, this
process guarantees that every enclave created during a power cycle has a unique
EID.

Although the SDM does not specifically guarantee this, the EID field in an
enclave's SECS does not appear to be modified by any instruction. This makes
the EID's value suitable for identifying an enclave throughout its lifetime,
even across evictions of its SECS page from the EPC.


\subsubsection{Evicting an EPC Page}
\label{sec:sgx_ewb}

The system software evicts an EPC page using the \texttt{EWB} instruction,
which produces all the data needed to restore the evicted page at a later time
via the \texttt{ELDU} instruction, as shown in Figure~\ref{fig:sgx_eviction}.

\begin{figure}[hbt]
  \centering
  \includegraphics[width=85mm]{figures/sgx_eviction.pdf}
  \caption{
    The \texttt{EWB} instruction outputs the encrypted contents of the
    evicted EPC page, a subset of the fields in the page's EPCM entry, a
    MAC tag, and a nonce. All this information is used by the \texttt{ELDB} or
    \texttt{ELDU} instruction to load the evicted page back into the EPC, with
    privacy, integrity and freshness guarantees.
  }
  \label{fig:sgx_eviction}
\end{figure}

% EWB: SDM S 41.3

\texttt{EWB}'s output consists of an encrypted version of the evicted EPC
page's contents, a subset of the fields in the EPCM entry corresponding to the
page, the nonce discussed in \S~\ref{sec:sgx_va}, and a message authentication
code~(MAC,~\S~\ref{sec:integrity_crypto}) tag. With the exception of the nonce,
\texttt{EWB} writes its output in DRAM outside the PRM area, so the system
software can choose to further evict it to disk.

The EPC page contents is encrypted, to protect the privacy of the enclave's
data while the page is stored in the untrusted DRAM outside the PRM range.
Without the use of encryption, the system software could learn the contents of
an EPC page by evicting it from the EPC.

% Security Information (SECINFO): SDM S 38.11, S 38.11.{1,2}

The page metadata is stored in a \textit{Page Information}~(PAGEINFO)
structure, illustrated in Figure~\ref{fig:sgx_ewb_pageinfo}. This structure is
similar to the PAGEINFO structure described in \S~\ref{sec:sgx_eadd} and
depicted in Figure~\ref{fig:sgx_pageinfo}, except that the SECINFO field has
been replaced by a PCMD field, which contains the virtual address of a
\textit{Page Crypto Metadata}~(PCMD) structure.

\begin{figure}[hbt]
  \centering
  \includegraphics[width=75mm]{figures/sgx_ewb_pageinfo.pdf}
  \caption{
    The PAGEINFO structure used by the \texttt{EWB} and \texttt{ELDU} /
    \texttt{ELDB} instructions
  }
  \label{fig:sgx_ewb_pageinfo}
\end{figure}

The LINADDR field in the PAGEINFO structure is used to store the ADDRESS field
in the EPCM entry, which indicates expected virtual address used to access the
page. The PCMD structure embeds the \textit{Security Information}~(SECINFO)
described in \S~\ref{sec:sgx_eadd}, which is used to store the page type~(PT)
and the access permission flags~(R, W, X) in the EPCM entry. The PCMD structure
also stores the enclave's ID~(EID,~\S~\ref{sec:sgx_eid}). These fields are
later used by \texttt{ELDU} or \texttt{ELDB} to populate the EPCM entry for the
EPC page that is loaded back in.

The metadata described above is stored unencrypted, so the OS has the option of
using the information inside as-is for its own bookkeeping. This has no
negative impact on security, because the metadata is not confidential. In fact,
with the exception of the enclave ID, all the metadata fields are specified by
the system software when \texttt{ECREATE} is called. The enclave ID is only
useful for identifying the enclave that the EPC page belongs to, and the system
software already has this information as well.

Asides from the metadata described above, the PCMD structure also stores the
MAC tag generated by \texttt{EWB}. The MAC tag covers the authenticity of the
EPC page contents, the metadata, and the nonce. The MAC tag is checked by
\texttt{ELDU} and \texttt{ELDB}, which will only load an evicted page back into
the EPC if the MAC verification confirms the authenticity of the page data,
metadata, and nonce. This security check protects against the page swapping
attacks described in \S~\ref{sec:page_swapping_attacks}.

% Eviction of an SECS Page: SDM S 39.5.5

Similarly to \texttt{EREMOVE}, \texttt{EWB} will only evict the EPC page
holding an enclave's SECS if there is no other EPCM entry whose ENCLAVESECS
field references the SECS. At the same time, as an optimization, the SGX
implementation does not perform \texttt{ETRACK}-related checks when evicting a
SECS. This is safe because a SECS is only evicted if the EPC has no pages
belonging to the SECS' enclave, which implies that there isn't any TCS
belonging to the enclave in the EPC, so no processor can be executing enclave
code.

% Eviction of a Version Array Page: SDM S 39.5.6

The pages holding Version Arrays can be evicted, just like any other EPC page.
VA pages are never accessible by software, so they can't have any TLB entries
pointing to them. Therefore, \texttt{EWB} evicts VA pages without performing
any \texttt{ETRACK}-related checks. The ability to evict VA pages has profound
implications that will be discussed in \S~\ref{sec:sgx_eviction_trees}.

\texttt{EWB}'s dataflow, shown in detail in Figure~\ref{fig:sgx_ewb}, has an
aspect that can be confusing to OS developers. The instruction reads the
virtual address of the EPC page to be evicted from a register (RBX) and writes
it to the LINADDR field of the PAGEINFO structure that it is provded. The
separate input (RBX) could have been removed by having the EPC page's address
be provided in the LINADDR field.

\begin{figure}[hbt]
  \centering
  \includegraphics[width=85mm]{figures/sgx_ewb.pdf}
  \caption{
    The data flow of the EWB instruction that evicts an EPC page. The page's
    content is encrypted in a non-EPC RAM page. A nonce is created and saved
    in an empty slot inside a VA page. The page's EPCM metadata and a MAC
    are saved in a separate area in non-EPC memory.
  }
  \label{fig:sgx_ewb}
\end{figure}


\subsubsection{Loading an Evicted Page Back into EPC}

% AEX Operational Detail: SDM S 40.4.1

After an EPC page belonging to an enclave is evicted, any attempt to access the
page from enclave code will result in a Page Fault~(\#PF,~\S~\ref{sec:faults}).
The \#PF will cause the logical processor to exit enclave mode via
AEX~(\S~\ref{sec:sgx_aex}), and then invoke the OS kernel's page fault handler.

Page faults receive special handling from the AEX process. While leaving the
enclave, the AEX logic specifically checks if the hardware exception that
triggerred the AEX was \#PF. If that is the case, the AEX implementation clears
the least significant 12 bits of the CR2 register, which stores the virtual
address whose translation caused a page fault.

In general, the OS kernel's page handler needs to be able to extract the
virtual page number~(VPN,~\S~\ref{sec:paging_vpn}) from CR2, so that it knows
which memory page needs to be loaded back into DRAM. The OS kernel may also be
able to use the 12 least signficant address bits, which are not part of the
VPN, to better predict the application software's memory access patterns.
However, unlike the bits that make up the VPN, the bottom 12 bits are not
absolutely necessary for the fault handler to carry out its job. Therefore,
SGX's AEX implementation clears these 12 bits, in order to limit the amount of
information that is learned by the page fault handler.

% Loading an Enclave Page: SDM S 39.5.4
% TODO: page fault -> AEX -> ELDB / ELDU

When the OS page fault handler examines the address in the CR2 register and
determines that the faulting address is inside the EPC, it is generally
expected to use the \texttt{ELDU} or \texttt{ELDB} instruction to load the
evicted page back into the EPC. If the outputs of \texttt{EWB} have been
evicted from DRAM to a slower storage medium, the OS kernel will have to read
the outputs back into DRAM before invoking \texttt{ELDU} / \texttt{ELDB}.

\texttt{ELDU} and \texttt{ELDB} verify the MAC tag produced by \texttt{EWB},
described in \S~\ref{sec:sgx_ewb}. This prevents the OS kernel from performing
the page swapping-based active address translation attack described in
\S~\ref{sec:page_swapping_attacks}.


\subsubsection{Eviction Trees}
\label{sec:sgx_eviction_trees}

The SGX design allows VA pages to be evicted from the EPC, just like enclave
pages. When a VA page is evicted from EPC, all the nonces stored by the VA
slots become inaccessible to the processor. Therefore, the evicted pages
associated with these nonces cannot be restored by \texttt{ELDB} until the
OS loads the VA page back into the EPC.

In other words, an evicted page depends on the VA page storing its nonce, and
cannot be loaded back into the EPC until the VA page is loaded back in as well.
The dependency graph created by this relationship is a forest of
\texttt{eviction trees}. An eviction tree, shown in
Figure~\ref{fig:sgx_eviction_tree}, has enclave EPC pages as leaves, and VA
pages as inner nodes. A page's parent is the VA page that holds its nonce.
Since \texttt{EWB} always outputs a nonce in a VA page, the root node of each
eviction tree is always a VA page in the EPC.

\begin{figure}[hbt]
  \centering
  \includegraphics[width=80mm]{figures/sgx_eviction_tree.pdf}
  \caption{
    A version tree formed by evicted VA pages and enclave EPC pages. The
    enclave pages are leaves, and the VA pages are inner nodes. The OS controls
    the tree's shape, which impacts the performance of evictions, but not their
    correctness.
  }
  \label{fig:sgx_eviction_tree}
\end{figure}

A straightforward inductive argument shows that when an OS wishes to load an
evicted enclave page back into the EPC, it needs to load all the VA pages on
the path from the eviction tree's root to the leaf corresponding to the enclave
page. Therefore, the number of page loads required to satisfy a page fault
inside the EPC depends on the shape of the eviction tree that contains the
page.

The SGX design leaves the OS in complete control of the shape of the eviction
trees. This has no negative impact on security, as the tree shape only impacts
the performance of the eviction scheme, and not its correctness.

\subsection{SGX Enclave Measurement}
\label{sec:sgx_measurement}
\label{sec:sgx_mrenclave}

SGX implements a software attestation scheme that follows the general
principles outlined in \S~\ref{sec:generic_software_attestation}. For the
purposes of this section, the most relevant principle is that a remote party
authenticates an enclave based on its measurement, which is intended to
identify the software that is executing inside the enclave. The remote party
compares the enclave measurement reported by the trusted hardware with an
expected measurement, and only proceeds if the two values match.

\S~\ref{sec:sgx_enclave_lifecycle} explains that an SGX enclave is built using
the \texttt{ECREATE}~(\S~\ref{sec:sgx_ecreate}),
\texttt{EADD}~(\S~\ref{sec:sgx_eadd}) and \texttt{EEXTEND} instructions.
After the enclave is initialized via
\texttt{EINIT}~(\S~\ref{sec:sgx_einit_overview}) the instructions mentioned
above cannot be used anymore. As the SGX measurement scheme follows the
principles outlined in \S~\ref{sec:generic_measurement}, the measurement of an
SGX enclave is obtained by computing a secure
hash~(\S~\ref{sec:integrity_crypto}) over the inputs to the \texttt{ECREATE},
\texttt{EADD} and \texttt{EEXTEND} instructions used to create the enclave and
load the initial code and data into its memory. \texttt{EINIT} finalizes the
hash that represents the enclave's measurement.

Along with the enclave's contents, the enclave author is expected to specify
the sequence of instructions that should be used in order to create an enclave
whose measurement will match the expected value used by the remote party in the
software attestation process. The \texttt{.so} and \texttt{.dll} dynamically
loaded library file formats, which are SGX's intended enclave delivery methods,
already include informal specifications for loading algorithms. We expect the
informal loading specifications to serve as the starting points for
specifications that prescribe the exact sequences of SGX instructions that
should be used to create enclaves from \texttt{.so} and \texttt{.dll} files.

As argued in \S~\ref{sec:generic_measurement}, an enclave's measurement is
computed using a secure hashing algorithm, so the system software can only
build an enclave that matches an expected measurement by following the exact
sequence of instructions specified by the enclave's author.

% SGX Enclave Control Structure (SECS): SDM S 38.7, S 38.7.1

The SGX design uses the 256-bit SHA-2~\cite{fips2015shs} secure hash function
to compute its measurements. SHA-2 is a block hash
function~(\S~\ref{sec:integrity_crypto}) that operates on 64-byte blocks, uses
a 32-byte internal state, and produces a 32-byte output. Each enclave's
measurement is stored in the MRENCLAVE field of the enclave's SECS. The 32-byte
field stores the internal state and final output of the 256-bit SHA-2 secure
hash function.


\subsubsection{Measuring \texttt{ECREATE}}

% ECREATE: SDM S 41.3

The \texttt{ECREATE} instruction, overviewed in \S~\ref{sec:sgx_ecreate},
first initializes the MRENCLAVE field in the newly created SECS using the
256-bit SHA-2 initialization algorithm, and then extends the hash with
the 64-byte block depicted in Table~\ref{fig:ecreate_mrenclave}.

\begin{table}[hbt]
  \centering
  \begin{tabularx}{\columnwidth}{| r | r | X |}
  \hline
  \textbf{Offset} & \textbf{Size} & \textbf{Description}\\
  \hline
  0 & 8 & "ECREATE\textbackslash{}0" \\
  \hline
  8 & 8 & SECS.SSAFRAMESIZE (\S~\ref{sec:sgx_ssa}) \\
  \hline
  16 & 8 & SECS.SIZE (\S~\ref{sec:sgx_elrange}) \\
  \hline
  32 & 8 & 32 zero (0) bytes \\
  \hline
  \end{tabularx}
  \caption{
    64-byte block extended into MRENCLAVE by \texttt{ECREATE}
  }
  \label{fig:ecreate_mrenclave}
\end{table}

The enclave's measurement does not include the BASEADDR field. The omission is
intentional, as it allows the system software to load an enclave at any virtual
address inside a host process that satisfies the ELRANGE
restrictions~(\S~\ref{sec:sgx_elrange}), without changing the enclave's
measurement. This feature can be combined with a compiler that generates
position-independent enclave code to obtain relocatable enclaves.

The enclave's measurement includes the \texttt{SSAFRAMESIZE} field, which
guarantees that the SSAs~\ref{sec:sgx_ssa} created by AEX and used by
\texttt{EENTER}~(\S~\ref{sec:sgx_eenter}) and
\texttt{ERESUME}~(\S~\ref{sec:sgx_eresume}) have the size that is expected by
the enclave's author. Leaving this field out of an enclave's measurement would
allow a malicious enclave loader to attempt to attack the enclave's security
checks by specifying a bigger SSAFRAMESIZE than the enclave's author intended,
which could cause the SSA contents written by an AEX to overwrite the enclave's
code or data.

The enclave's measurement does not include the ATTRIBUTES field in the SECS.
However, the SGX software attestation definitely needs to cover the ATTRIBUTES
field, and especially its XFRM sub-field~(\S~\ref{sec:sgx_ssa}) that decides
the value of XCR0~(\S~\ref{sec:registers}) while the enclave's code is
executing, and therefore determines the extended architectural features that
are used by the enclave. If XFRM is not covered, a malicious enclave loader
could attempt to subvert an enclave's security checks by setting XFRM to a
value that enables architectural extensions which change the semantics of
instructions used by the enclave, but still produces an \texttt{XSAVE} output
that fits in SSAFRAMESIZE.

As mentioned above, the ATTRIBUTES field is not included in the enclave's
measurement. Instead, it is included directly in the information that is
covered by the attestation signature, which will be discussed in
\S~\ref{sec:sgx_attestation_signature}.


\subsubsection{Measuring \texttt{EADD}}

% EADD and EEXTEND Interaction: SDM S 39.1.1
% EADD: SDM S 41.3

The \textit{EADD} instruction, described in \S~\ref{sec:sgx_eadd}, extends the
SHA-2 hash in MRENCLAVE with the 64-byte block shown in
Table~\ref{fig:eadd_mrenclave}.

\begin{table}[hbt]
  \centering
  \begin{tabularx}{\columnwidth}{| r | r | X |}
  \hline
  \textbf{Offset} & \textbf{Size} & \textbf{Description}\\
  \hline
  0 & 8 &
  "EADD\textbackslash{}0\textbackslash{}0\textbackslash{}0\textbackslash{}0" \\
  \hline
  8 & 8 & PAGEINFO.LINADDR \\
  \hline
  16 & 48 & SECINFO (first 48 bytes) \\
  \hline
  \end{tabularx}
  \caption{
    64-byte block extended into MRENCLAVE by \texttt{EADD}
  }
  \label{fig:eadd_mrenclave}
\end{table}

The address included in the measurement is the address where the
\texttt{EADD}ed page is expected to be mapped in the enclave's virtual address
space. This ensures that the system software sets up the enclave's virtual
memory layout according to the enclave author's specifications. If a malicious
enclave loader attempts to set up the enclave's layout incorrectly, perhaps in
order to mount an active address translation
attack~(\S~\ref{sec:memory_mapping_attacks}), the loaded enclave's measurement
will differ from the measurement expected by the enclave's author.

The virtual address of the newly created page is measured relatively to the
start of the enclave's ELRANGE. In other words, the value included in the
measurement is LINADDR - BASEADDR. This makes the enclave's measurement
invariant to BASEADDR changes, which is desirable for relocatable enclaves.
Measuring the relative addresses still preserves all the information about the
memory layout inside ELRANGE, and therefore has no negative security impact.

\texttt{EADD} also measures the first 48 bytes of the SECINFO
structure~(\S~\ref{sec:sgx_eadd}) provided to \texttt{EADD}, which contains the
page type (PT) and access permissions (R, W, X) field values used to initialize
the page's EPCM entry. By the same argument as above, including these values in
the measurement guarantees that the memory layout built by the system software
loading the enclave matches the specifications of the enclave author.

The EPCM field values mentioned above take up less than one byte in the SECINFO
structure, and the rest of the bytes are reserved and expected to be
initialized to zero. This leaves plenty of expansion room for future SGX
features.


\subsubsection{Measuring \texttt{EEXTEND}}

The most notable omission from Table~\ref{fig:eadd_mrenclave} is the data used
to initialize the newly created EPC page. Therefore, \texttt{EADD}'s
measurements can guarantee, for example, that an enclave's memory layout
consists of three executable pages followed by five writable data pages.
However, the measurements don't cover the code or data loaded in these pages.

% EEXTEND: SDM S 41.3

This gap is filled by the \texttt{EEXTEND} instruction, which exists solely for
the reason of measuring data loaded inside the enclave. The instruction reads
in a virtual address, and extends the enclave's measurement hash with the
five 64-byte blocks in Table~\ref{fig:eextend_mrenclave}, which have the effect
of guaranteeing the contents of a 256-byte chunk of data in the enclave's
memory.

\begin{table}[hbt]
  \centering
  \begin{tabularx}{\columnwidth}{| r | r | X |}
  \hline
  \textbf{Offset} & \textbf{Size} & \textbf{Description}\\
  \hline
  0 & 8 & "EEXTEND\textbackslash{}0" \\
  \hline
  8 & 8 & ENCLAVEOFFSET \\
  \hline
  16 & 48 & 48 zero (0) bytes \\
  \hline
  \hline
  64 & 64 & bytes 0 - 64 in the chunk \\
  \hline
  \hline
  128 & 64 & bytes 64 - 128 in the chunk \\
  \hline
  \hline
  192 & 64 & bytes 128 - 192 in the chunk \\
  \hline
  \hline
  256 & 64 & bytes 192 - 256 in the chunk \\
  \hline
  \end{tabularx}
  \caption{
    64-byte blocks extended into MRENCLAVE by \texttt{EEXTEND}
  }
  \label{fig:eextend_mrenclave}
\end{table}

It is essential that the message blocks used by \texttt{EEXTEND} include the
address of the 256-byte chunk, in addition to the contents of the data chunk.
If the address were not included, a malicious enclave loader could mount the
memory mapping attack described in \S~\ref{sec:memory_mapping_attacks} and
illustrated in Figure~\ref{fig:active_mapping_attack}. The malicious loader
would \texttt{EADD} the \texttt{errorOut} page contents at the virtual address
intended for \texttt{disclose}, \texttt{EADD} the \texttt{disclose} page
contents at the virtual address intended for \texttt{errorOut}, and then
\texttt{EEXTEND} the pages in the wrong order. If \texttt{EEXTEND} would not
include the address of the data chunk that is measured, the steps above would
yield the same measurement as the correctly constructed enclave.


\subsubsection{Measuring \texttt{EINIT}}

% EINIT: SDM S 41.3



\subsection{SGX Enclave Versioning Support}
\label{sec:sgx_versioning}

The software attestation model~(\S~\ref{sec:generic_software_attestation})
introduced by the Trusted Platform Module~(\S~\ref{sec:tpm}) relies on a
measurement~(\S~\ref{sec:sgx_measurement}), which is essentially a content
hash, to identify the software inside a container. The downside of using
content hashes for identity is that there is no relation between the identities
of containers that hold different versions of the same software.

In practice, it is highly desirable that systems based on secure containers
can handle software updates without having access to the remote party in the
initial software attestation process. This entails having the ability to
migrate secrets between the container that has the old version of the software
and the container that has the updated version. This requirement translates
into a need for a separate identity system that can recogize the relationship
between two versions of the same software.

SGX supports the migration of secrets between enclaves that represent different
versions of the same software, as shown in
Figure~\ref{fig:sgx_secret_migration}.

\begin{figure}[hbt]
  \centering
  \includegraphics[width=87mm]{figures/sgx_secret_migration.pdf}
  \caption{
    SGX has a certificate-based enclave identity scheme, which can be used to
    migrate secrets between enclaves that contain different versions of the
    same software module. Here, enclave A's secrets are migrated to enclave B.
  }
  \label{fig:sgx_secret_migration}
\end{figure}

The secret migration feature relies on a one-level certificate
hierarchy~(~\S~\ref{sec:certificates}), where each enclave author is a
Certificate Authority, and each enclave receives a certificate from its author.
These certificates must be formatted as Signature Structures~(SIGSTRUCT), which
are described in \S~\ref{sec:sgx_sigstruct}. The information in these
certificates is the basis for an enclave identity scheme, presented in
\S~\ref{sec:sgx_certificate_identity}, which can recognize the relationship
between different versions of the same software.

The \texttt{EINIT} instruction~(\S~\ref{sec:sgx_einit_overview}) examines the
target enclave's certificate and uses the information in it to populate
the SECS~(\S~\ref{sec:sgx_secs}) fields that describe the enclave's
certificate-based identity. This process is summarized in
\S~\ref{sec:sgx_einit_sigstruct}.

Last, the actual secret migration process is based on the key derivation
service implemented by the \texttt{EGETKEY} instruction, which is described
in \S~\ref{sec:sgx_egetkey}. The sending enclave uses the \texttt{EGETKEY}
instruction to obtain a symmetric key~(\S~\ref{sec:crypto_keys}) based on its
identity, encrypts its secrets with the key, and hands off the encrypted
secrets to the untrusted system software. The receiving enclave passes the
sending enclave's identity to \texttt{EGETKEY}, obtains the same symmetric key
as above, and uses the key to decrypt the secrets received from system
software.

The symmetric key obtained from \texttt{EGETKEY} can be used in conjunction
with cryptographic primitives that protect the
privacy~(\S~\ref{sec:privacy_crypto}) and
integrity~(\S~\ref{sec:integrity_crypto}) of an enclave's secrets while they
are migrated to another enclave by the untrusted system software. However,
symmetric keys alone cannot be used to provide freshness
guarantees~(\S~\ref{sec:crypto_primitives}), so secret migration is subject to
replay attacks. This is acceptable when the secrets being migrated are
immutable, such as when the secrets are encryption keys obtained via software
attestation


\subsubsection{Enclave Certificates}
\label{sec:sgx_sigstruct}
\label{sec:sgx_mrsigner}

% Enclave Signature Structure (SIGSTRUCT): SDM S 38.13

The SGX design requires that each enclave must have a certificate issued by its
author. This requirement is enforced by
\texttt{EINIT}~(\S~\ref{sec:sgx_einit_overview}), which refuses to operate on
enclaves without valid certificates.

The SGX implementation consumes certificates formatted as
\textit{Signature Structures}~(SIGSTRUCT), which are intended to be generated
by an enclave building toolchain, as shown in Figure~\ref{fig:sgx_sigstruct}.

\begin{figure}[hbt]
  \centering
  \includegraphics[width=85mm]{figures/sgx_sigstruct.pdf}
  \caption{
    An enclave's Signature Structure (SIGSTRUCT) is intended to be generated by
    an enclave building toolchain that has access to the enclave author's
    private RSA key.
  }
  \label{fig:sgx_sigstruct}
\end{figure}

A SIGSTRUCT certificate consists of metadata fields, the most interesting of
which are presented in Table~\ref{fig:sgx_sigstruct_info}, and an RSA
signature that guarantees the authenticity of the metadata, formatted as shown
in Table~\ref{fig:sgx_sigstruct_rsa}. The semantics of the fields will be
revealed in the following sections.

\begin{table}[hbt]
  \centering
  \begin{tabularx}{\columnwidth}{| l | r | X |}
  \hline
  \textbf{Field} & \textbf{Bytes} & \textbf{Description} \\
  \hline
  ENCLAVEHASH & 32 & Must equal the enclave's
                     measurement~(\S~\ref{sec:sgx_mrenclave}). \\
  \hline
  ISVPRODID & 32 & Differentiates modules signed by the same public key. \\
  \hline
  ISVSVN & 32 & Differentiates versions of the same module. \\
  \hline
  VENDOR & 4 & Differentiates Intel enclaves. \\
  \hline
  ATTRIBUTES & 16 & Constrains the enclave's attributes. \\
  \hline
  ATTRIBUTEMASK & 16 & Constrains the enclave's attributes. \\
  \hline
  \end{tabularx}
  \caption{
    A subset of the metadata fields in a SIGSTRUCT enclave certificate.
  }
  \label{fig:sgx_sigstruct_info}
\end{table}

\begin{table}[hbt]
  \centering
  \begin{tabularx}{\columnwidth}{| l | r | X |}
  \hline
  \textbf{Field} & \textbf{Bytes} & \textbf{Description} \\
  \hline
  MODULUS & 384 & RSA key modulus \\
  \hline
  EXPONENT & 4 & RSA key public exponent \\
  \hline
  SIGNATURE & 384 & RSA signature (See \S~\ref{sec:sgx_rsa_check}) \\
  \hline
  Q1 & 384 & Simplifies RSA signature verification.
             (See \S~\ref{sec:sgx_rsa_check}) \\
  \hline
  Q2 & 384 & Simplifies RSA signature verification.
             (See \S~\ref{sec:sgx_rsa_check}) \\
  \hline
  \end{tabularx}
  \caption{
    The format of the RSA signature used in a SIGSTRUCT enclave certificate.
  }
  \label{fig:sgx_sigstruct_rsa}
\end{table}

The enclave certificates must be signed by RSA
signatures~(\S~\ref{sec:integrity_crypto}) that follow the method described in
RFC 3447~\cite{jonsson2003pkcsv21}, using 256-bit SHA-2~\cite{fips2015shs} as
the hash function that reduces the input size, and the padding method described
in PKCS \#1 v1.5~\cite{kaliski1998pkcs1v15}.

The SGX implementation only supports 3072-bit RSA keys whose public exponent is
3. The key size is likely chosen to meet FIPS'
recommendation~\cite{fips2012keysize}, which makes SGX eligible for use in U.S.
government applications. The public exponent 3 affords a simplified signature
verification algorithm, which is discussed in \S~\ref{sec:sgx_rsa_check}. The
simplified algorithm also requires the fields Q1 and Q2 in the RSA signature,
which are also described in \S~\ref{sec:sgx_rsa_check}.


\subsubsection{Certificate-Based Enclave Identity}
\label{sec:sgx_certificate_identity}

An enclave's identity is determined by three fields in its
certificate~(\S~\ref{sec:sgx_sigstruct}): the modulus of the RSA key used to
sign the certificate (MODULUS), the enclave's product ID (ISVPRODID) and the
security version number (ISVSVN).

% EINIT: SDM S 41.3
% Enclave Signature Structure (SIGSTRUCT): SDM S 38.13

The public RSA key used to issue a certificate identifies the enclave's author.
All RSA keys used to issue enclave certificates must have the public exponent
set to 3, so they are only differentiated by their moduli. SGX does not use the
entire modulus of a key, but rather a 256-bit SHA-2 hash of the modulus. This
is called a \textit{signer measurement}~(MRSIGNER), to parallel the name of
\textit{enclave measurement}~(MRENCLAVE) for the SHA-2 hash that identifies an
enclave's contents.

The SGX implementation relies on a hard-coded MRSIGNER value to recognize
certificates issued by Intel. Enclaves that have an Intel-issued certificate
can receive additional privileges, which are discussed in
\S~\ref{sec:sgx_attestation}.

An enclave author can use the same RSA key to issue certificates for enclaves
that represent different software modules. Each module is identified by a
unique Product ID~(ISVPRODID) value. Conversely, all the enclaves whose
certificates have the same ISVPRODID and are issued by the same RSA key
(and therefore have the same MRENCLAVE) are assumed to represent different
versions of the same software module. Enclaves whose certificates are signed
by different keys are always assumed to contain different software modules.

% Security Version Numbers: SDM S 39.4.2

Enclaves that represent different versions of a module can have different
security version numbers (SVN). The SGX design disallows the migration of
secrets from an enclave with a higher SVN to an enclave with a lower SVN. This
restriction is intended to assist with the distribution of security patches,
as follows.

If a security vulnerability is discovered in an enclave, the author can release
a fixed version with a higher SVN. As users upgrade, SGX will facilitate the
migration of secrets from the vulnerable version of the enclave to the fixed
version. Once an user's secrets are migrated, the SVN restrictions in SGX will
deflect any attack based on building the vulnerable enclave version and using
it to read the migrated secrets.

Software upgrades that add functionality should not be accompanied by an SVN
increase, as SGX allows secrets to be migrated freely between enclaves with
matching SVN values. As explained above, a software module's SVN should only be
incremented when a security vulnerability is found. SIGSTRUCT only allocates
2 bytes to the ISVSVN field, which translates to 65,536 possible SVN values.
This space can be exhausted if a large team (incorrectly) sets up a continunous
build system to allocate a new SVN for every software build that it produces,
and each code change triggers a build.


\subsubsection{CPU Security Version Numbers}
\label{sec:sgx_cpusvn}

% Security Version Numbers: SDM S 39.4.2
% Hardware Security Version: SDM S 39.4.2.2

The SGX implementation itself has a security version number (CPUSVN), which is
used in the key derivation process implemented~\cite{intel2013patent1} by
\texttt{EGETKEY}, in addition to the enclave's identity information. CPUSVN is
a 128-bit value that, according to the SDM, reflects the processor's microcode
update version.

% EINIT: SDM S 41.3

The SDM does not describe the structure of CPUSVN, but it states that comparing
CPUSVN values using integer comparison is not meaningful, and that only some
CPUSVN values are valid. Furthermore, CPUSVNs admit an ordering relationship
that has the same semantics as the ordering relationship between enclave SVNs.
Specifically, an SGX implementation will consider all SGX implementations with
lower SVNs to be compromised due to security vulnerabilities, and will not
trust them.

An SGX patent~\cite{intel2013patent1} discloses that CPUSVN is a concatenation
of small integers representing the SVNs of the various components that make up
SGX's implementation. This structure is consistent with all the statements made
in the SDM.


\subsubsection{Establishing an Enclave's Identity}
\label{sec:sgx_einit_sigstruct}

When the \texttt{EINIT}~(\S~\ref{sec:sgx_einit_overview}) instruction prepares
an enclave for code execution, it also sets the SECS~(\S~\ref{sec:sgx_secs})
fields that make up the enclave's certificate-based identity, as shown in
Figure~\ref{fig:sgx_einit_sigstruct}.

\begin{figure}[hbt]
  \centering
  \includegraphics[width=85mm]{figures/sgx_einit_sigstruct.pdf}
  \caption{
    \texttt{EINIT} verifies the RSA signature in the enclave's certificate. If
    the certificate is valid, the information in it is used to populate the
    SECS fields that make up the enclave's certificate-based identity.
  }
  \label{fig:sgx_einit_sigstruct}
\end{figure}

% EINIT: SDM S 41.3
% Enclave Signature Structure (SIGSTRUCT): SDM S 38.13

\texttt{EINIT} requires the virtual address of the SIGSTRUCT certificate
issued to the enclave, and uses the information in the certificate to
initialize the certificate-based identity information in the enclave's SECS.
Before using the information in the certificate, \texttt{EINIT} first verifies
its RSA signature. The SIGSTRUCT fields Q1 and Q2, along with the RSA exponent
3, facilitate a simplified verification algorithm, which is discussed in
\S~\ref{sec:sgx_rsa_check}.

If the SIGSTRUCT certificate is found to be properly signed, \texttt{EINIT}
follows the steps discussed in the following few paragraphs to ensure that the
certificate was issued to the enclave that is being initialized. Once the
checks are completed, \texttt{EINIT} computes MRSIGNER, the 256-bit SHA-2 hash
of the MODULUS field in the SIGSTRUCT, and writes it into the enclave's SECS.
\texttt{EINIT} also copies the ISVPRODID and ISVSVN fields from SIGSTRUCT into
the enclave's SECS. As explained in \S~\ref{sec:sgx_certificate_identity},
these fields make up the enclave's certificate-based identity.

\texttt{EINIT} performs a few checks to make sure that the enclave undergoing
initialization was indeed authorized by provided the SIGSTRUCT certificate. The
most obvious check involves making sure that the MRENCLAVE value in SIGSTRUCT
equals the enclave's measurement, which is stored in the MRENCLAVE field in
the enclave's SECS.

% SECS.ATTRIBUTES.XFRM: SDM S 42.7.2.1

However, MRENCLAVE does not cover the enclave's attributes, which are stored in
the ATTRIBUTES field of the SECS. As discussed in
\S~\ref{sec:sgx_ecreate_mrenclave_no_attributes}, omitting ATTRIBUTES from
MRENCLAVE facilitates writing enclaves that have optimized implementations
which can use architectural extensions when present, and also have fallback
implementations that work on CPUs without the extensions. Such enclaves can
execute corectly when built with a variety of values in the
XFRM~(\S~\ref{sec:sgx_secs_attributes}, \S~\ref{sec:sgx_ssa}) attribute. At
the same time, allowing system software to use arbitrary values in the
ATTRIBUTES field would compromise SGX's security guarantees.

When an enclave uses software
attestation~(\S~\ref{sec:generic_software_attestation}) to gain access to
secrets, the ATTRIBUTES value used to build it is included in the SGX
attestation signature~(\S~\ref{sec:sgx_attestation}). This gives the remote
party in the attestation process the opportunity to reject an enclave built
with an undesirable ATTRIBUTES value. However, when secrets are obtained using
the migration process facilitated by certificate-based identities, there is no
remote party that can check the enclave's attributes.

The SGX design solves this problem having enclave authors convey the set of
acceptable attribute values for an enclave in the ATTRIBUTES and ATTRIBUTEMASK
fields of the SIGSTRUCT certificate issued for the enclave. \texttt{EINIT} will
refuse to initialize an enclave using a SIGSTRUCT if the bitwise AND between
the ATTRIBUTES field in the enclave's SECS and the ATTRIBUTESMASK field in the
SIGSTRUCT does not equal the SIGSTRUCT's ATTRIBUTES field. This check prevents
enclaves with undesirable attributes from obtaining and potentially leaking
secrets using the migration process.

Any enclave author can use SIGSTRUCT to request that any of the bits in an
enclave's ATTRIBUTES field is zero. However, certain bits can only be set to
one for enclaves that are signed by Intel. \texttt{EINIT} has a mask of
restricted ATTRIBUTES bits, discussed in \S~\ref{sec:sgx_attestation}. The
\texttt{EINIT} implementation contains a hard-coded MRSIGNER value that is used
to identify Intel's privileged enclaves, and only allows privileged enclaves to
be built with an ATTRIBUTES value that matches any of the bits in the
restricted mask. This check is essential to the security of the SGX software
attestation process, which is described in \S~\ref{sec:sgx_attestation}.

Last, \texttt{EINIT} also inspects the VENDOR field in SIGSTRUCT. The SDM
description of the VENDOR field in the section dedicated to SIGSTRUCT suggests
that the field is essentially used to distinguish between special enclaves
signed by Intel, which use a VENDOR value of 0x8086, and everyone else's
enclaves, which should use a VENDOR value of zero. However, the \textit{EINIT}
pseudocode seems to imply that the SGX implementation only checks that
VENDOR is either zero or 0x8086.


\subsubsection{Enclave Key Derivation}
\label{sec:sgx_egetkey}

SGX's secret migration mechanism is based on the symmetric key derivation
service that is offered to enclaves by the \texttt{EGETKEY} instruction,
illustrated in Figure~\ref{fig:sgx_egetkey}.

\begin{figure}[hbt]
  \centering
  \includegraphics[width=87mm]{figures/sgx_egetkey.pdf}
  \caption{
    \texttt{EGETKEY} implements a key derivation service that is primarily used
    by SGX's secret migration feature. The key derivation material is derived
    from the SECS of the calling enclave, the information in a Key Request
    structure, and a.
  }
  \label{fig:sgx_egetkey}
\end{figure}

The keys produced by \texttt{EGETKEY} are derived based on the identity
information in the current enclave's SECS and on a secret stored in secure
hardware inside the SGX-enabled processor. The pieces of information used to
derive the keys are selected by a Key Request (KEYREQUEST) structure, shown in
Table~\ref{fig:sgx_keyrequest}.

% Key Request (KEYREQUEST): SDM S 38.17

\begin{table}[hbt]
  \centering
  \begin{tabularx}{\columnwidth}{| l | r | X |}
  \hline
  \textbf{Field} & \textbf{Bytes} & \textbf{Description} \\
  \hline
  KEYNAME & 2 & The desired key type; secret migration uses ``Seal'' keys \\
  \hline
  KEYPOLICY & 2 & The identity information (MRENCLAVE and/or MRSIGNER) \\
  \hline
  ISVSVN & 2 & The enclave SVN used in derivation\\
  \hline
  CPUSVN & 16 & SGX implementation SVN used in derivation \\
  \hline
  ATTRIBUTEMASK & 16 & Selects enclave attributes \\
  \hline
  KEYID & 32 & Random bytes \\
  \hline
  \end{tabularx}
  \caption{
    A subset of the fields in the KEYREQUEST structure. These fields
  }
  \label{fig:sgx_keyrequest}
\end{table}

% KDF uses FIPS SP 800-180 and AES-CMAC
%   US 8,972,746 B2 - 44:17-29
%   US 9,807,200 B2 - 42:32-48
% EGETKEY uses a fused key and derivation
%   US 8,972,746 B2 - 51:34-65
%   US 9,807,200 B2 - 49:34-65
%   ISCA SGX Slide 99

The SDM does not specify the key derivation algorithm, but the SGX
patents~\cite{intel2013patent1, intel2013patent2} disclose that the keys are
derived using the method described in FIPS SP 800-108~\cite{fips2009kdf} using
AES-CMAC~\cite{fips2005cmac} as a Pseudo-Random Function (PRF). The same
patents state that the secret used for key derivation is stored in e-fuses in
the CPU, which is confirmed by the ISCA 2015 SGX
tutorial~\cite{intel2015iscasgx}.

This additional information implies that all \texttt{EGETKEY} invocations that
use the same key derivation material will result in the same key, even across
CPU power cycles. Furthermore, it is impossible for an adversary to obtain the
key produced from a specific key derivation material without access to the
secret stored in the CPU's e-fuses. The following paragraphs discuss the pieces
of data used in the key derivation material.

% Key Request (KEYREQUEST): SDM S 38.17
% EGETKEY: SDM S 41.4.1
% Key Derivation: SDM Table 41-43

The KEYNAME field in KEYREQUEST always participates in the key generation
material. It indicates the type of the key to be generated. While the SGX
design define a few key types, the secret migration feature always uses ``Seal
keys''. The other key types are used by the SGX software attestation process,
which will be outlined in \S~\ref{sec:sgx_attestation}.

The KEYPOLICY field in KEYREQUEST has two flags that indicate if the MRENCLAVE
and MRSIGNER fields in the enclave's SECS will be used for key derivation.
Although the fields admits 4 values, only two seem to make sense, as argued
below.

Setting the MRENCLAVE flag in KEYPOLICY ties the derived key to the current
enclave's measurement, which reflects its contents. No other enclave will be
able to obtain the same key. This is useful when the derived key is used to
encrypt enclave secrets so they can be stored by system software in
non-volatile memory, and thus survive power cycles.

If the MRSIGNER flag in KEYPOLICY is set, the derived key is tied to the public
RSA key that issued the enclave's certificate. Therefore, other enclaves issued
by the same author may be able to obtain the same key, subject to the
restrictions below. This is the only KEYPOLICY value that allows for secret
migration.

It makes little sense to have no flag set in KEYPOLICY. In this case, the
derived key has no useful security property, as it can be obtained by other
enclaves that are completely unrelated to the enclave invoking
\texttt{EGETKEY}. Conversely, setting both flags is redundant, as setting
MRENCLAVE alone will cause the derived key to be tied to the current enclave,
which is the strictest possible policy.

The KEYREQUEST structure specifies the enclave
SVN~(ISVSVN,~\S~\ref{sec:sgx_certificate_identity}) and SGX implementation
SVN~(CPUSVN,~\S~\ref{sec:sgx_cpusvn}) that will be used in the key derivation
process. However, \texttt{EGETKEY} will reject the derivation request and
produce an error code if the desired enclave SVN is greater than the current
enclave's SVN, or if the desired SGX implementation SVN is greater than the
current implementation's SVN.

The SVN restrictions prevent the migration of secrets from enclaves with higher
SVNs to enclaves with lower SVNs, or from SGX implementations with higher SVNs
to implementations with lower SVNs. \S~\ref{sec:sgx_certificate_identity}
argues that the SVN restrictions can reduce the impact of security
vulnerabilities in enclaves and in SGX's implementation.

\texttt{EGETKEY} always uses the ISVPROD value from the current enclave's SECS
for key derivation. It follows that secrets can never flow between enclaves
whose SIGSTRUCT certificates assign them different Product IDs.

Similarly, the key derivation material always includes the value of an 128-bit
Owner Epoch (OWNEREPOCH) SGX configuration register. This register is intended
to be set by the computer's firmware to a value that represents the computer's
current user. If the computer changes ownership, the new owner can change the
Owner Epoch value, which invalidates all the old keys produced by
\texttt{EGETKEY}.

The \texttt{EGETKEY} derivation material also includes a 256-bit value supplied
by the enclave, in the KEYID field. This makes it possible for an enclave to
generate a collection of keys from \texttt{EGETKEY}, instead of a single key.
The SDM states that KEYID should be populated with a random number, and is
intended to help prevent key wear-out.

Last, the key derivation material includes the bitwise AND of the
ATTRIBUTES~(\S~\ref{sec:sgx_secs_attributes}) field in the enclave's SECS and
the ATTRIBUTESMASK field in the KEYREQUEST structure. The mask has the effect
of removing some of the ATTRIBUTES bits from the key derivation material,
making it possible to migrate secrets between enclaves with different
attributes. \S~\ref{sec:sgx_ecreate_mrenclave_no_attributes} and
\S~\ref{sec:sgx_einit_sigstruct} explain the need for this feature, as well as
its security implications.

Before adding the masked attributes value to the key generation material, the
\texttt{EGETKEY} implementation forces the mask bits corresponding to the
INIT and DEBUG attributes~(\S~\ref{sec:sgx_secs_attributes}) to be set. From a
practical standpoint, this means that secrets will never be migrated between
enclaves that support debugging and production enclaves.

\subsection{SGX Software Attestation}
\label{sec:sgx_attestation}

The software attestation scheme implemented by SGX follows the principles
outlined in \S~\ref{sec:generic_software_attestation}. An SGX-enabled processor
computes a measurement of the code and data that is loaded in each enclave,
which is similar to the measurement computed by the TPM~(\S~\ref{sec:tpm}). The
software inside an enclave can start a process that results in an SGX
attestation signature, which includes the enclave's measurement and an enclave
message.

\begin{figure}[hbt]
  \centering
  \includegraphics[width=85mm]{figures/sgx_attestation_overview.pdf}
  \caption{
    Setting up an SGX enclave and undergoing the software attestation process
    involves the SGX instructions \texttt{EINIT} and \texttt{EREPORT}, and two
    special enclaves authored by Intel, the SGX Launch Enclave and the SGX
    Quoting Enclave.
  }
  \label{fig:sgx_attestation_overview}
\end{figure}

% EPID signing takes too long for microcode.
%   US 8,972,746 B2 - 33:16-29, 33:58

The cryptographic primitive used in SGX's attestation signature is too complex
to be implemented in hardware, so signing process is performed by a privileged
\textit{Quoting Enclave}, which is issued by Intel, and can access the SGX
attestation key. This enclave is discused in \S~\ref{sec:sgx_quoting_enclave}.

Pushing the signing functionality into the Quoting Enclave creates the need for
a secure communication path between an enclave that desires to undergo
software attestation and the Quoting Enclave. The SGX design solves this
problem with a local attestation mechanism that can be used by an enclave to
prove its identity to any other enclave hosted by the same SGX-enabled CPU.
This scheme, described in \S~\ref{sec:sgx_ereport}, is implemented by the
\texttt{EREPORT} instruction.

% Keys: SDM S 39.4.3
% ISCA SGX Slides 104, 105, 106

The SGX attestation key used by the Quoting Enclave does not exist at the time
SGX-enabled processors leave the factory. The attestation key is provisioned
later, using a largely undocumented process that is known to involve at least
one other enclave issued by Intel, and at least two special \texttt{EGETKEY}
key types. The publicly available details of this process are summarized in
\S~\ref{sec:sgx_quoting_enclave}.

The SGX implementation contained in a CPU's hardware does not directly enforce
the enclave attribute checks that decide which enclaves can access the CPU
secrets used for software attestation. The potentially complex restrictions on
enclave attributes are instead enforced by the \textit{Launch Enclave}, which
is an enclave issued by Intel that gets to approve every other enclave before
it is initialized by \textit{EINIT}~(\S~\ref{sec:sgx_einit_overview}). The
officially documented information about the approval process is discussed in
\S~\ref{sec:sgx_launch_enclave}.

% Enclave License
%   US 8,972,746 B2 - 34:6-23
% Licenses are evaluated into Permits (which became EINITTTOKEN)
%   US 8,972,746 B2 - 34:24-29, 35:27-59
% EINIT requires a Permit to launch a production enclave
%   US 8,972,746 B2 - 35:60-67, 36:1-3, 36:47-52, 38:4-65
% License Enclave creates Permit
%   US 8,972,746 B2 - 36:40-46, 37:50-67, 38:1-3
% EMKPERMIT seems to have gotten merged into EINIT
%   US 8,972,746 B2 - 36:53-67, 37:1-7, 37:24-49
% License became SIGSTRUCT
%   US 8,972,746 B2 - 37:8-23

One of the SGX patents~\cite{intel2013patent1} discloses in no uncertain terms
that the Launch Enclave was introduced to ensure that each enclave's author has
a business relationship with Intel, and implements a software licensing system.
\S~\ref{sec:sgx_licensing} briefly discusses the implications, should this turn
out to be true.


\subsubsection{Local Attestation}
\label{sec:sgx_ereport}

An enclave proves its identity to another \textit{target enclave} via the
\texttt{EREPORT} instruction shown in Figure~\ref{fig:sgx_ereport}. The SGX
instruction produces an attestation \textit{Report} (REPORT) that
cryptographically binds a message supplied by the enclave with the enclave's
measurement-based~(\S~\ref{sec:sgx_measurement}) and
certificate-based~(\S~\ref{sec:sgx_certificate_identity}) identities. The
cryptographic binding is accomplished by a MAC tag computed using a symmetric
key that is only shared between the target enclave and the SGX implementation.

\begin{figure}[hbt]
  \centering
  \includegraphics[width=85mm]{figures/sgx_ereport.pdf}
  \caption{
    \texttt{EREPORT} data flow
  }
  \label{fig:sgx_ereport}
\end{figure}

The \texttt{EREPORT} instruction reads the current enclave's identity
information from the enclave's SECS~(\S~\ref{sec:sgx_secs}), and uses it to
populate the REPORT structure. Specifically, \texttt{EREPORT} copies the
SECS fields indicating the enclave's measurement (MRENCLAVE), certificate-based
identity (MRSIGNER, ISVPROD, ISVSVN), and attributes (ATTRIBUTES). The
attestation report also includes the SVN of the SGX implementation (CPUSVN)
and a 64-byte (512-bit) message supplied by the enclave.

The target enclave that receives the attestation report can convince itself of
the report's authenticy by verifying its MAC tag, using a key obtained by
invoking \texttt{EGETKEY}~(\S~\ref{sec:sgx_egetkey}) and asking for a report
key. \texttt{EREPORT} uses the same key derivation process as \texttt{EGETKEY}
does when invoked with KEYNAME set to the value that represents a Report Key.

The report key returned by \texttt{EGETKEY} is derived from a secret embedded
in the processor~(\S~\ref{sec:sgx_egetkey}), and the key material includes the
target enclave's measurement. The target enclave can be assured that the MAC
tag in the report was produced by the SGX implementation, for the following
reasons. The cryptographic properties of the underlying key derivation and
MAC algorithms ensure that only the SGX implementation can produce the MAC tag,
as it is the only entity who can access the processor's secret, and it would be
impossible for an attacker to derive the report key without knowing the
processor's secret. The SGX design guarantees that key produced by
\texttt{EGETKEY} depends on the calling enclave's measurement, so only the
target enclave can obtain the key used to produce the MAC tag in the report.

\texttt{EREPORT} requires the virtual address of a
\textit{Report Target Info}~(TARGETINFO) structure that contains the
measurement-based identity and attributes of the target enclave. These values
are used to derive the same key that \texttt{EGETKEY} would.

When deriving a report key, \texttt{EGETKEY} does not use the fields
corresponding to the enclave's certificate-based identity (MRSIGNER, ISVPRODID,
ISVSVN) in the key generation material. However, it does use the SVN of the
SGX implementation, which is also written to the REPORT structure.

Last, \texttt{EREPORT} sets the KEYID field in the key generation material to
the contents of an SGX configuration register (CR\_REPORT\_KEYID) that is
initialized with a random value when SGX is initialized. The KEYID value is
also saved in the attestation report, but it is not covered by the MAC tag.


\subsubsection{Enclave Approval}
\label{sec:sgx_launch_enclave}

% ATTRIBUTES: SDM S 38.7.1
% EINIT: SDM S 41.3
% EINITTOKENKEY is bit 5, INTEL_ONLY_MASK is 0x20

Instead of implementing potentially complex access checks

% EINIT Token Structure (EINITTOKEN): SDM S 38.14

\begin{figure}[hbt]
  \centering
  \includegraphics[width=87mm]{figures/sgx_einittoken.pdf}
  \caption{
    The SGX Launch Enclave computes the EINITTOKEN.
  }
  \label{fig:sgx_einittoken}
\end{figure}

The SDM states that the MAC that protects EINITTOKEN's authenticity is computed
using a block cipher-based MAC~(CMAC,~\cite{fips2005cmac}), but stops short of
specifying the underlying cipher. One of the SGX papers~\cite{anati2013sgx}
states that SGX implementation uses a CMAC based on 128-bit AES.



% MRSIGNER: SDM S 39.4.1.2

After an enclave gets cleared by the SGX Launch Enclave and is initialized via
\texttt{EINIT}, it can authenticate itself to a remote party by participating
in a software attestation process.


\subsubsection{The Quoting Enclave}
\label{sec:sgx_quoting_enclave}

While the SDM paints a complete picture of the local attestation mechanism, it
is a lot more secretive about the Quoting Enclave and the underlying keys.
Fortunately, the SGX patents~\cite{intel2013patent1, intel2013patent2} shed
some light on the topic.



\subsubsection{Licensing}
\label{sec:sgx_licensing}

However, the software attestation scheme in SGX's design, summarized in
Figure~\ref{fig:sgx_attestation_overview} is unnecessarily complicated by the
decision to deeply couple software attestation with
\textbf{an enclave licensing mechanism that allows Intel to force itself as an
intermediary in the distribution of all enclave software}.

\HeadingLevelB{SGX Enclave Launch Control}
\label{sec:sgx_launch_control}

The SGX design includes a launch control process, which introduces an
unnecessary approval step that is required before running most enclaves on a
computer. The approval decision is made by the \textit{Launch Enclave} (LE),
which is an enclave issued by Intel that gets to approve every other enclave
before it is initialized by \textit{EINIT}~(\S~\ref{sec:sgx_einit_overview}).
The officially documented information about this approval process is discussed
in \S~\ref{sec:sgx_launch_enclave}.

% Enclave License
%   US 8,972,746 B2 - 34:6-23
% Licenses are evaluated into Permits (which became EINITTTOKEN)
%   US 8,972,746 B2 - 34:24-29, 35:27-59
% EINIT requires a Permit to launch a production enclave
%   US 8,972,746 B2 - 35:60-67, 36:1-3, 36:47-52, 38:4-65
% License Enclave creates Permit
%   US 8,972,746 B2 - 36:40-46, 37:50-67, 38:1-3
% EMKPERMIT seems to have gotten merged into EINIT
%   US 8,972,746 B2 - 36:53-67, 37:1-7, 37:24-49
% License became SIGSTRUCT
%   US 8,972,746 B2 - 37:8-23

The SGX patents~\cite{intel2013patent1, intel2013patent2} disclose in no
uncertain terms that the Launch Enclave was introduced to ensure that each
enclave's author has a business relationship with Intel, and implements a
software licensing system. \S~\ref{sec:sgx_licensing} briefly discusses the
implications, should this turn out to be true.

The remainder of the section argues that the Launch Enclave should be removed
from the SGX design. \S~\ref{sec:sgx_provisioning_privacy} explains that the
LE is not required to enforce the computer owner's launch control policy, and
concludes that the LE is only meaningful if it enforces a policy that is
detrimental to the computer owner. \S~\ref{sec:sgx_enclaves_vs_system} debunks
the myth that an enclave can host malware, which is likely to be used to
justify the LE. \S~\ref{sec:sgx_enclaves_vs_av} argues that Anti-Virus (AV)
software is not fundamentally incompatible with enclaves, further disproving
the theory that Intel needs to actively police the software that runs inside
enclaves.


\HeadingLevelC{Enclave Attributes Access Control}
\label{sec:sgx_launch_enclave}

% ATTRIBUTES: SDM S 38.7.1

The SGX design requires that all enclaves be vetted by a Launch Enclave~(LE),
which is only briefly mentioned in Intel's official documentation. Neither its
behavior nor its interface with the system software is specified. We speculate
that Intel has not been forthcoming about the LE because of its role in
enforcing software licensing, which will be discussed in
\S~\ref{sec:sgx_licensing}. This section abstracts away the licensing aspect
and assumes that the LE enforces a black-box Launch Control Policy.

The LE approves an enclave by issuing an \textit{EINIT Token}~(EINITTOKEN),
using the process illustrated in Figure~\ref{fig:sgx_einittoken}. The
EINITTOKEN structure contains the approved enclave's
measurement-based~(\S~\ref{sec:sgx_measurement}) and
certificate-based~(\S~\ref{sec:sgx_certificate_identity}) identities, just like
a local attestation REPORT~(\S~\ref{sec:sgx_ereport}). This token is inspected
by \texttt{EINIT}~(\S~\ref{sec:sgx_einit_overview}), which refuses to
initialize enclaves with incorrect tokens.

\begin{figure}[hbt!]
  \centering
  \includegraphics[width=95mm]{figures/sgx_einittoken.pdf}
  \caption{
    The SGX Launch Enclave computes the EINITTOKEN.
  }
  \label{fig:sgx_einittoken}
\end{figure}

% EINIT Token Structure (EINITTOKEN): SDM S 38.14

While an EINIT token is handled by untrusted system software, its integrity is
protected by a MAC tag~(\S~\ref{sec:integrity_crypto}) that is computed using a
\textit{Launch Key} obtained from \texttt{EGETKEY}. The \texttt{EINIT}
implementation follows the same key derivation process as \texttt{EGETKEY} to
convince itself that the EINITTOKEN provided to it was indeed generated by an
LE that had access to the Launch Key.

The SDM does not document the MAC algorithm used to confer integrity guarantees
to the EINITTOKEN structure. However, the \texttt{EINIT} pseudocode verifies
the token's MAC tag using the same function that the \textit{EREPORT}
pseudocode uses to create the REPORT structure's MAC tag. It follows that the
reasoning in \S~\ref{sec:sgx_ereport} can be reused to conclude that EINITTOKEN
structures are MACed using AES-CMAC with 128-bit keys.

% EGETKEY: SDM S 41.4.1
% Key Derivation: SDM Table 41-43

The \texttt{EGETKEY} instruction only derives the Launch Key for enclaves that
have the LAUNCHKEY attribute set to true. The Launch Key is derived using the
same process as the Seal Key~(\S~\ref{sec:sgx_egetkey}). The derivation material
includes the current enclave's versioning information (ISVPRODID and ISVSVN)
but it does not include the main fields that convey an enclave's identity,
which are MRSIGNER and MRENCLAVE. The rest of the derivation material follows
the same rules as the material used for Seal Keys.

The EINITTTOKEN structure contains the identities of the approved enclave
(MRENCLAVE and MRSIGNER) and the approved enclave attributes (ATTRIBUTES). The
token also includes the information used for the Launch Key derivation,
which includes the LE's Product ID (ISVPRODIDLE), SVN (ISVSVNLE), and the
bitwise AND between the LE's ATTRIBUTES and the ATTRIBUTEMASK used in the
KEYREQUEST (MASKEDATTRIBUTESLE).

The EINITTOKEN information used to derive the Launch Key can also be used
by \texttt{EINIT} for damage control, e.g. to reject tokens issued by Launch
Enclaves with known security vulnerabilities. The reference pseudocode supplied
in the SDM states that \texttt{EINIT} checks the DEBUG bit in the
MASKEDATTRIBUTESLE field, and will not initialize a production enclave using
a token issued by a debugging LE. It is worth noting that MASKEDATTRIBUTESLE is
guaranteed to include the LE's DEBUG attribute, because \texttt{EGETKEY} forces
the DEBUG attribute's bit in the attributes mask to 1
(\S~\ref{sec:sgx_egetkey}).

The check described above make it safe for Intel to supply SGX enclave
developers with a debugging LE that has its DEBUG attribute set, and performs
minimal or no security checks before issuing an EINITTOKEN. The DEBUG attribute
disables SGX's integrity protection, so the only purpose of the security checks
performed in the debug LE would be to help enclave development by mimicking its
production counterpart. The debugging LE can only be used to launch any enclave
with the DEBUG attribute set, so it does not undermining Intel's ability to
enforce a Launch Control Policy on production enclaves.

% EINIT: SDM S 41.3
% EINITTOKENKEY is bit 5, INTEL_ONLY_MASK is 0x20

The enclave attributes access control system described above relies on the LE
to reject initialization requests that set privileged attributes such as
PROVISIONKEY on unauthorized enclaves. However, the LE cannot vet itself, as
there will be no LE available when the LE itself needs to be initialized.
Therefore, the Launch Key access restrictions are implemented in hardware.

\texttt{EINIT} accepts an EINITTOKEN whose VALID bit is set to zero, if
the enclave's MRSIGNER~(\S~\ref{sec:sgx_mrsigner}) equals a hard-coded value
that corresponds to an Intel public key. For all other enclave authors, an
invalid EINIT token causes \texttt{EINIT} to reject the enclave and produce an
error code.

This exemption to the token verification policy provides a way to bootstrap the
enclave attributes access control system, namely using a zeroed out EINITTOKEN
to initialize the Launch Enclave. At the same time, the cryptographic
primitives behind the MRSIGNER check guarantee that only Intel-provided
enclaves will be able to bypass the attribute checks. This does not change
SGX's security properties because Intel is already a trusted party, as it is
responsible for generating the Provisioning Keys and Attestation Keys used by
software attestation~(\S~\ref{sec:sgx_quoting_enclave}).

Curiously, the \texttt{EINIT} pseudocode in the SDM states that the instruction
enforces an additional restriction, which is that all enclaves with the
LAUNCHKEY attribute must have its certificate issued by the same Intel public
key that is used to bypass the EINITTTOKEN checks. This restriction appears to
be redundant, as the same restriction could be enforced in the Launch Enclave.


\HeadingLevelC{Licensing}
\label{sec:sgx_licensing}

The SGX patents~\cite{intel2013patent1, intel2013patent2} disclose that
\texttt{EINIT} Tokens and the Launch Enclave~(\S~\ref{sec:sgx_launch_enclave})
were introduced to verify that the SIGSTRUCT certificates associated with
production enclaves are issued by enclave authors who have a business
relationship with Intel. In other words, the Launch Enclave is intended to be
\textbf{an enclave licensing mechanism that allows Intel to force itself as an
intermediary in the distribution of all enclave software}.

The SGX patents are likely to represent an early version of the SGX design, due
to the lengthy timelines associated with patent application approval.
In light of this consideration, we cannot make any claims about Intel's current
plans. However, given that we know for sure that Intel considered enclave
licensing at some point, we briefly discuss the implications of implementing
such a licensing plan.

Intel has a near-monopoly on desktop and server-class processors, and being
able to decide which software vendors are allowed to use SGX can effectively
put Intel in a position to decide winners and losers in many software markets.

Assuming SGX reaches widespread adoption, this issue is the software security
equivalent to the Net Neutrality debates that have pitted the software industry
against telecommunication giants. Given that virtually all competent software
development companies have argued that losing Net Neutrality will stifle
innovation, it is fairly safe to assume that Intel's ability to regulate access
to SGX will also stifle innovation.

Furthermore, from a historical perspective, the enclave licensing scheme
described in the SGX patents is very similar to Verified Boot, which was
briefly discussed in \S~\ref{sec:sgx_related_tpm}. Verified Boot has mostly
received negative reactions from software developers, so it is likely that an
enclave licensing scheme would meet the same fate, should the developer
community become aware of it.


\HeadingLevelC{System Software Can Enforce a Launch Policy}
\label{sec:sgx_provisioning_privacy}

\S~\ref{sec:sgx_enclave_lifecycle} explains that the SGX instructions used to
load and initialize enclaves (\texttt{ECREATE}, \texttt{EADD}, \texttt{EINIT})
can only be issued by privileged system software, because they manage the EPC,
which is a system resource.

A consequence on the restriction that only privileged software can issue
\texttt{ECREATE} and \texttt{EADD} instructions is that the system software is
able to track all the public contents that is loaded into each enclave. The
privilege requirements of \texttt{EINIT} mean that the system software can also
examine each enclave's SIGSTRUCT. It follows that the system software has
access to a superset of the information that the Launch Enclave might use.

Furtheremore, \texttt{EINIT}'s privileged instruction status means that the
system software can perform its own policy checks before allowing application
software to initialize an enclave. So, the system software can enforce a Launch
Control Policy set by the computer's owner. For example, an IaaS cloud service
provider may use its hypervisor to implement a Launch Control Policy that
limits what enclaves its customers are allowed to execute.

Given that the system software has access to a superset of the information that
the Launch Enclave might use, it is easy to see that the set of policies that
can be enforced by system software is a superset of the policies that can be
supported by an LE. Therefore, the only rational explanation for the existence
of the LE is that it was designed to implement a Launch Control Policy that is
not beneficial to the computer owner.

As an illustration of this argument, we consider the case of restricting access
to \texttt{EGETKEY}'s Provisioning keys~(\S~\ref{sec:sgx_quoting_enclave}).
The derivation material for Provisioning keys does not include OWNEREPOCH, so
malicious enclaves can potentially use these keys to track a CPU chip package
as it exchanges owners. For this reason, the SGX design includes a simple
access control mechanism that can be used by system software to limiting
enclave access to Provisioning keys. \texttt{EGETKEY} refuses to derive
Provisioning keys for enclaves whose PROVISIONKEY attribute is not set to true.

It follows that a reasonable Launch Control Policy would only allow the
PROVISIONKEY attribute to be set for the enclaves that implement software
attestation, such as Intel's Provisioning Enclave and Quoting Enclave. This
policy can easily be implemented by system software, given its exclusive access
to the \texttt{EINIT} instruction.

The only concern with the approach outlined above is that a malicious system
software might abuse the PROVISIONKEY attribute to generate a unique identifier
for the hardware that it runs on, similar to the much maligned Intel Processor
Serial Number~\cite{intel1999psn}. We dismiss this concern by pointing out that
system software has access to many unique identifiers, such as the
Media Access Control~(MAC) address of the Ethernet adapter integrated into the
motherboard's chipset~(\S~\ref{sec:motherboard}).


\HeadingLevelC{Enclaves Cannot Damage the Host Computer}
\label{sec:sgx_enclaves_vs_system}
% NOTE: This is a slightly edited answer to a question we received.

SGX enclaves execute at the lowest privilege level (user mode / ring 3), so
they are subject to the same security checks as their host application. For
example, modern operating systems set up the I/O maps~(\S~\ref{sec:segments})
to prevent application software from directly accessing the I/O address
space~(\S~\ref{sec:address_spaces}), and use the supervisor (S) page table
attribute~(\S~\ref{sec:page_table_attributes}) to deny application software
direct access to memory-mapped devices~(\S~\ref{sec:address_spaces}) and to the
DRAM that stores the system software. Enclave software is subject to I/O
privilege checks and address translation checks, so a malicious enclave cannot
directly interact with the computer's devices, and cannot tamper the system
software.

It follows that software running in an enclave has the same means to compromise
the system software as its host application, which come down to exploiting a
security vulnerability. The same solutions used to mitigate vulnerabilities
exploited by application software (e.g.,
\texttt{seccomp/bpf}~\cite{kim2013seccompbpf}) apply to enclaves.

The only remaining concern is that an enclave can perform a denial of service
(DoS) attack against the system software. The rest of this section addresses
the concern.

The SGX design provides system software the tools it needs to protect itself
from enclaves that engage in CPU hogging and DRAM hogging. As enclaves cannot
perform I/O directly, these are the only two classes of DoS attacks available
to them.

An enclave that attempts to hog an LP assigned to it can be preempted by the
system software via an Inter-Processor Interrupt~(IPI,~\S~\ref{sec:interrupts})
issued from another processor. This method is available as long as the system
software reserves at least one LP for non-enclave computation.

Furthermore, most OS kernels use tick schedulers, which use a real-time clock
(RTC) configured to issue periodical interrupts (ticks) to all cores. The RTC
interrupt handler invokes the kernel's scheduler, which chooses the thread that
will get to use the logical processor until the next RTC interrupt is received.
Therefore, kernels that use tick schedulers always have the opportunity to
de-schedule enclave threads, and don't need to rely on the ability to send
IPIs.

In SGX, the system software can always evict an enclave's EPC pages to non-EPC
memory, and then to disk. The system software can also outright deallocate an
enclave's EPC pages, though this will probably cause the enclave code to
encounter page faults that cannot be resolved. The only catch is that the EPC
pages that hold metadata for running enclave threads cannot be evicted or
removed. However, this can easily be resolved, as the system software can
always preempt enclave threads, using one of the methods described above.

% TODO(pwnall): Move the following paragraphs into Sanctum.
%Sanctum gives the system software less control over the DRAM regions allocated
%to enclaves, in order to hide the enclaves' memory access patterns.
%Specifically, the system software cannot reclaim a DRAM region from an enclave
%without the enclave's cooperation. However, the system software can always
%completely terminate an enclave and reclaim all its memory.
%
%Sanctum's enclaves can only be terminated when all their threads are stopped.
%Therefore, when the system software decides that an enclave is hogging CPU or
%DRAM, it can preempt all the enclave's threads, using the methods described
%above, and then terminate the enclave.


\HeadingLevelC{Interaction with Anti-Virus Software}
\label{sec:sgx_enclaves_vs_av}
% NOTE: This is a slightly edited answer to a question we received.

Today's anti-virus (AV) systems are glorified pattern matchers. AV software
simply scans all the executable files on the system and the memory of running
processes, looking for bit patterns that are thought to only occur in malicious
software. These patterns are somewhat pompously called ``virus signatures".

SGX (and TXT, to some extent) provides a method for executing code in an
isolated container that we refer to as an enclave. Enclaves are isolated from
all the other software on the computer, including any AV software that might be
installed.

The isolation afforded by SGX opens up the possibility for bad actors to
structure their attacks as a generic loader that would end up executing a
malicious payload without tripping the AV's pattern matcher.  More
specifically, the attack would create an enclave and initialize it with a
generic loader that looks innocent to an AV. The loader inside the enclave
would obtain an encrypted malicious payload, and would undergo software
attestation with an Internet server to obtain the payload's encryption key. The
loader would then decrypt the malicious payload and execute it inside the
enclave.

In the scheme suggested here, the malicious payload only exists in a decrypted
form inside an enclave's memory, which cannot be accessed by the AV. Therefore,
the AV's pattern matcher will not trip.

This issue does not have a solution that maintains the status-quo for the AV
vendors. The attack described above would be called a protection scheme if the
payload would be a proprietary image processing algorithm, or a DRM scheme.

On a brighter note, enclaves do not bring the complete extinction of AV, they
merely require a change in approach. Enclave code always executes at the lowest
privilege mode (ring 3 / user mode), so it cannot perform any I/O without
invoking the services of system software. For all intents and purposes, this
effectively means that enclave software cannot perform any malicious action
without the complicity of system software. Therefore, enclaves can be policed
effectively by intelligent AV software that records and filters the I/O
performed by software, and detects malicious software according to the actions
that it performs, rather than according to bit patterns in its code.

Furthermore, SGX's enclave loading model allows the possibility of performing
static analysis on the enclave's software. For simplicity, assume the existence
of a standardized static analysis framework.  The initial enclave contents is
not encrypted, so the system software can easily perform static analysis on it.
Dynamically loaded code or Just-In-Time code generation (JIT) can be handled by
requiring that all enclaves that use these techniques embed the static analysis
framework and use it to analyze any dynamically loaded code before it is
executed. The system software can use static verification to ensure that
enclaves follow these rules, and refuse to initialize any enclaves that fail
verification.

In conclusion, enclaves in and of themselves don't introduce new attack vectors
for malware. However, the enclave isolation mechanism is fundamentally
incompatible with the approach employed by today's AV solutions. Fortunately,
it is possible (though non-trivial) to develop more intelligent AV software for
enclave software.

