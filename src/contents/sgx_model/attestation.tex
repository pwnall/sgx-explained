\subsection{SGX Software Attestation}
\label{sec:sgx_attestation}

The software attestation scheme implemented by SGX follows the principles
outlined in \S~\ref{sec:generic_software_attestation}. An SGX-enabled processor
computes a measurement of the code and data that is loaded in each enclave,
which is similar to the measurement computed by the TPM~(\S~\ref{sec:tpm}). The
software inside an enclave can start a process that results in an SGX
attestation signature, which includes the enclave's measurement and an enclave
message. However, the details of the SGX attestation scheme are unnecessarily
complicated by Intel's desire to force itself as an intermediary in the
distribution of enclave software.

Under the SGX design, there are two sides to an enclave's identity. The
enclave's measurement, described in \S~\ref{sec:sgx_measurement}, reflects all
the code and data loaded in the enclave when it is initialized. The enclave's
signature, described in \S~\ref{sec:sgx_sigstruct}, is an endorsement of the
enclave's contents issued by a software vendor.




\subsubsection{Enclave Signature}
\label{sec:sgx_sigstruct}
\label{sec:sgx_mrsigner}

% Enclave Signature Structure (SIGSTRUCT): SDM S 38.13

\subsubsection{Enclave Key Generation}
\label{sec:sgx_egetkey}


\subsubsection{Local Attestation}

The enclave asks the CPU to produce an \textit{enclave report}, which contains
the attestation challenge, the enclave's measurement value, an
enclave-generated 256-byte message, and an HMAC
 with a symmetric key shared between the CPU and a special
\textit{signing enclave} produced by Intel. The untrusted system software
executes the signing enclave, which verifies the report and produces an
\textit{attestation signature} that covers the challenge and the enclave's
measurement.


\subsubsection{Enclave Initialization}
\label{sec:sgx_einit}

% EINIT Token Structure (EINITTOKEN): SDM S 38.14


\subsubsection{The Attestation Signature}
\label{sec:sgx_attestation_signature}

