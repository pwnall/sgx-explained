\subsection{Interrupts}
\label{sec:interrupts}

Peripherals use \textit{interrupts} to signal the occurrence of an event that
must be handled by system software. For example, a keyboard triggers interrupts
when a key is pressed or depressed. System software also relies on interrupts
to implement preemptive multi-threading.

Interrupts are a kind of hardware exception (\S~\ref{sec:faults}). Receiving an
interrupt causes an execution core to perform a privilege level switch and to
start executing the system software's interrupt handling code. Therefore, the
security concerns in \S~\ref{sec:faults} also apply to interrupts, with the
added twist that interrupts occur independently of the instructions executed by
the interrupted code, whereas most faults are triggered by the actions of the
application software that incurs them.

Given the importance of interrupts when assessing a system's security, this
section outlines the interrupt triggering and handling processes described
in the SDM.

% Advanced Programmable Interrupt Controller (APIC): SDM S 10
% Message Signaled Interrupts (MSI): SDM S 10.11

Peripherals use bus-specific protocols to signal interrupts. For example, PCIe
relies on \textit{Message Signaled Interrupts} (MSI), which are memory writes
issued to specially designed memory addresses. The bus-specific interrupt
signals are received by the \textit{I/O Advanced Programmable Interrupt
Controller} (IOAPIC) in the PCH, shown in Figure~\ref{fig:motherboard}.

% Local APIC ID: SDM S 10.4.6
% Extended xAPIC (x2APIC): SDM S 10.12

The IOAPIC routes interrupt signals to one or more \textit{Local Advanced
Programmable Interrupt Controllers} (LAPICs). As shown in
Figure~\ref{fig:cpu_die}, each logical CPU has a LAPIC that can receive
interrupt signals from the IOAPIC. The IOAPIC routing process assigns each
interrupt to an 8-bit \textit{interrupt vector} that is used to identify the
interrupt sources, and to a 32-bit \textit{APIC ID} that is used to identify
the LAPIC that receives the interrupt.

% Handling Interrupts: SDM S 10.8

Each LAPIC uses a 256-bit \textit{Interrupt Request Register} (IRR) to track
the unserviced interrupts that it has received, based on the interrupt vector
number. When the corresponding logical processor is available, the LAPIC copies
the highest-priority unserviced interrupt vector to the
\textit{In-Service Register} (ISR), and invokes the logical processor's
interrupt handling process.

% Issuing Interprocessor Interrupts: SDM S 10.6
% Interrupt Command Register (ICR): SDM S 10.6.1

At the execution core level, interrupt handling reuses many of the mechanisms
of fault handling (\S~\ref{sec:faults}). The interrupt vector number in the
LAPIC's ISR is used to locate an interrupt handler in the IDT, and the handler
is invoked, possibly after a privilege switch is performed. The interrupt
handler does the processing that the device requires, and then writes the
LAPIC's \textit{End Of Interrupt} (EOI) register to signal the fact that it has
completed handling the interrupt.

Interrupts are treated like faults, so interrupt handlers have full control
over the execution environment of the application being interrupted. This is
used to implement pre-emptive multi-threading, which relies on a clock device
that generates interrupts periodically, and on an interrupt handler that
performs context switches.

System software can cause an interrupt on any logical processor by writing the
target processor's APIC ID into the \textit{Interrupt Command Register} (ICR)
of the LAPIC associated with the logical processor that the software is running
on. These interrupts, called \textit{Inter-Processor Interrupts} (IPI), are
needed to implement TLB shoot-downs (\S~\ref{sec:tlbs}).
