\begin{abstract}

Intel's Software Guard Extensions (SGX) is a set of extensions to the Intel
architecture that aims to provide integrity and confidentiality guarantees to
security-sensitive computation performed on a computer where all the privileged
software (kernel, hypervisor, etc) is potentially malicious.

This paper analyzes Intel SGX, based on the 3
papers~\cite{mckeen2013sgx, anati2013sgx, hoekstra2013sgx} that introduced it,
on the Intel Software Developer's Manual~\cite{intel2015sdm} (which supersedes
the SGX manuals~\cite{intel2013sgxmanual, intel2014sgx2manual}), on an ISCA
2015 tutorial~\cite{intel2015iscasgx}, and on two
patents~\cite{intel2013patent1, intel2013patent2}. We use the papers,
reference manuals, and tutorial as primary data sources, and only draw on the
patents to fill in missing information.

This paper does not reflect the information available in two
papers~\cite{johnson2016sgxprovisioning, gueron2016sgxmee} that were published
after the first version of this paper.

This paper's contributions are a summary of the Intel-specific architectural
and micro-architectural details needed to understand SGX, a detailed and
structured presentation of the publicly available information on SGX, a series
of intelligent guesses about some important but undocumented aspects of SGX,
and an analysis of SGX's security properties.

\end{abstract}
