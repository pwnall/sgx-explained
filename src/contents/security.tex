\section{Security Background}
\label{sec:security_background}

Most systems rely on some cryptographic primitives for security. Unfortunately,
these primitives have many assumptions, and building a secure system on top of
them is a highly non-trivial endeavor. It follows that a system's security
analysis should be particularly interested in what cryptographic primitives are
used, and how they are integrated into the system.

\S~\ref{sec:crypto_primitives} and \S~\ref{sec:crypto_constructs} lay the
foundations for such an analysis by summarizing the primitives used by the
secure architectures that we are interested in, and by describing the most
common constructs built using these primitives.
\S~\ref{sec:generic_software_attestation} builds on these concepts and
describes software attestation, which is the most popular method for
establishing trust in a secure architecture.

Having looked at the cryptographic foundations for building secure systems, we
turn our attention to the attacks that secure architecture have to withstand.
Asides from forming a security checklist for architecture design, these attacks
build intuition that helps understand design decisions in the architectures
that we are interested in.

The attacks that can be performed on a computer system are broadly classified
into physical attacks and software attacks. In \textit{physical attacks}, the
attacker takes advantage of a system's physical implementation details to
perform an operation that bypasses the limitations set by the computer
system's software abstraction layers. In contrast, \textit{software attacks}
are performed solely by executing software on the victim computer.
\S~\ref{sec:physical_attacks} summarizes the main types of physical attacks.

The distinction between software and physical attacks is particularly relevant
in cloud computing scenarios, where gaining software access to the computer
running a victim's software can be accomplished with a credit card backed by
modest funds~\cite{ristenpart2009colocation}, whereas physical access is a
more difficult prospect that requires trespass, coercion, or social engineering
on the cloud provider's employees.

However, the distinction between software and physical attacks is blurred by
the attacks presented in \S~\ref{sec:device_attacks}, which exploit
programmable peripherals connected to the victim computer's bus in order to
carry out actions that are normally associated with physical attacks.

While the vast majority of software attacks exploit a bug in a software
component, there are a few attack classes that deserve attention from
architecture designers. Memory mapping attacks, described in
\S~\ref{sec:address_translation_attacks}, become a possibility on architectures
where the system software is not trusted. Cache timing attacks, summarized in
\S~\ref{sec:cache_timing} exploit microarchitectural behaviors that are
completely observable in software, but dismissed by the security analyses of
most systems.

\subsection{Cryptographic Primitives}
\label{sec:crypto_primitives}

This section overviews the cryptosystems used by secure architectures. We are
interested in cryptographic primitives that guarantee privacy, integrity, and
freshness, and we treat these primitives as black boxes, focusing on their use
in larger systems. \cite{katz2014crypto} covers the mathematics behind
cryptography, while \cite{ferguson2011crypto} covers the topic of building
systems out of cryptographic primitives. Table~\ref{fig:crypto_primitives}
summarizes the primitives covered in this section.

\begin{table}[hbt]
  \centering
  \begin{tabular}{| l | l | l |}
  \hline
  \textbf{Guarantee} & \textbf{Symmetric} & \textbf{Asymmetric} \\
                     & \textbf{Keys} & \textbf{Keys} \\
  \hline
  Privacy & \textit{Encryption} & \textit{Encryption} \\
          & AES-GCM, & RSA with \\
          & AES-CTR  & PKCS \#1 v2.0 \\
  \hline
  Integrity & \textit{HMAC},   & \textit{Signatures} \\
            & \textit{Authenticated} & \\
            & \textit{encryption} & \\
            & HMAC-AES-SHA & DSS-RSA, \\
            & AES-GCM & DSS-ECC \\
  \hline
  Freshness & \textit{Nonces} + & \textit{Nonces} + \\
            & integrity & integrity \\
  \hline
  \end{tabular}
  \caption{
    Summary of cryptographic primitives
  }
  \label{fig:crypto_primitives}
\end{table}

A message whose \textit{privacy} is protected can be transmitted over an
insecure medium without an adversary being able to obtain the information in
the message. When \textit{integrity} protection is used, the receiver is
guaranteed to either obtain a message that was transmitted by the sender, or to
notice that an attacker tampered with the message's content.

When multiple messages get transmitted over an untrusted medium, a
\textit{freshness} guarantee assures the receiver that she will obtain the
latest message coming from the sender, or will notice an attack. A freshness
guarantee is stronger than the equivalent integrity guarantee, because the
latter does not protect against \textit{replay attacks} where the attacker
replaces a newer message with an older message coming from the same sender.

The following example further illustrates these concepts. Suppose Alice is a
wealthy investor who wishes to either \textsc{buy} or \textsc{sell} an item
every day. Alice cannot trade directly, and must relay her orders to her
broker, Bob, over a network connection owned by Eve.

A communication system with privacy guarantees would prevent Eve from
distinguishing between a \textsc{buy} and a \textsc{sell} order, as illustrated
in Figure~\ref{fig:privacy_attack}. Without privacy, Eve would know Alice's
order before it is is placed by Bob, so Eve would presumably gain a financial
advantage at Alice's expense.

\begin{figure}[hbt]
  \centering
  \includegraphics[width=85mm]{figures/privacy_attack.pdf}
  \caption{
    In a privacy attack, Eve sees the message sent by Alice to Bob and can
    understand the information inside it. In this case, Eve can tell that the
    message is a \textbf{buy} order, and not a \textbf{sell} order.
  }
  \label{fig:privacy_attack}
\end{figure}

A system with integrity guarantees would prevent Eve from replacing Alice's
message with a false order, as shown in Figure~\ref{fig:integrity_attack}. In
this example, without integrity guarantees, Eve could replace Alice's message
with a \textsc{sell-everything} order, and buy Alice's assets at a very low
price.

\begin{figure}[hbt]
  \centering
  \includegraphics[width=85mm]{figures/integrity_attack.pdf}
  \caption{
    In an integrity attack, Eve replaces Alice's message with her own. In this
    case, Eve sends Bob a \textbf{sell-everything} order.
    understand the information inside it. In this case, Eve can tell that the
    message is a \textbf{buy} order, and not a \textbf{sell} order.
  }
  \label{fig:integrity_attack}
\end{figure}

Last, a communication system that guarantees freshness would ensure that Eve
cannot perform the replay attack pictured in Figure~\ref{fig:freshness_attack},
where she would replace Alice's message with an older message. Without
freshness guarantees, Eve could mount the following attack, which bypasses
both privacy and integrity guarantees. Over a few days, Eve would copy and
store Alice's messages from the network. When an order would reach Bob, Eve
would observe the market and determine if the order was \textsc{buy} or
\textsc{sell}. After building up a database of messages labeled \textsc{buy} or
\textsc{sell}, Eve would replace Alice's message with an old message of her
choice.

\begin{figure}[hbt]
  \centering
  \includegraphics[width=85mm]{figures/integrity_attack.pdf}
  \caption{
    In a freshness attack, Eve replaces Alice's message with a message that she
    sent at an earlier time. In this example, Eve builds a database of labeled
    messages over time, and is able to send Bob her choice of a \textsc{buy} or
    a \textsc{sell} order.
  }
  \label{fig:freshness_attack}
\end{figure}


\subsubsection{Cryptographic Keys}
\label{sec:crypto_keys}

All cryptographic primitives that we describe here rely on \textit{keys}, which
are small pieces of information that must only be disclosed according to
specific rules. A large part of a system's security analysis focuses on
ensuring that the keys used by the underlying cryptographic primitives are
produced and handled according to the primitives' assumptions.

Each cryptographic primitive has an associated \textit{key generation
algorithm} that uses random data to produce a unique key. The random data is
produced by a \textit{cryptographically strong pseudo-random number generator}
(CSPRNG) that expands a small amount of \textit{random seed} data into a much
larger amount of data, which is computationally indistinguishable from true
random data. The random seed must be obtained from a true source of randomness
whose output cannot be predicted by an adversary, such as the least significant
bits of the temperature readings coming from a hardware sensor.

\textit{Symmetric key} cryptography requires that all the parties in the system
establish a shared \textit{secret key}, which is usually referred to as ``the
key''. Typically, one party executes the key generation algorithm and securely
transmits the resulting key to the other parties, as illustrated in
Figure~\ref{fig:symmetric_key_generation}. The channel used to
distribute the key must provide privacy and integrity guarantees, which is a
non-trivial logistical burden. The symmetric key primitives mentioned here do
not make any assumption about the key, so the key generation algorithm simply
grabs a fixed number of bits from the CSPRNG.

\begin{figure}[hbt]
  \centering
  \includegraphics[width=87mm]{figures/symmetric_key_generation.pdf}
  \caption{
    In symmetric key cryptography, a secret key is shared by the parties that
    wish to communicate securely.
  }
  \label{fig:symmetric_key_generation}
\end{figure}

The salient feature of \textit{asymmetric key} cryptography is that it does not
require a private channel for key distribution. Each party executes the key
generation algorithm, which produces a \textit{private key} and a
\textit{public key} that are mathematically related. Each party's public key
is distributed to the other parties over a channel with integrity guarantees,
as shown in Figure~\ref{fig:asymmetric_key_generation}.
Asymmetric key primitives are more flexible than their symmetric key
counterparts, but are more complicated and consume more computational
resources.

\begin{figure}[hbt]
  \centering
  \includegraphics[width=87mm]{figures/asymmetric_key_generation.pdf}
  \caption{
    An asymmetric key generation algorithm produces a private key and an
    associated public key. The private key is held confidential, while the
    public key is given to any party who wishes to securely communicate with
    the private key's holder.
  }
  \label{fig:asymmetric_key_generation}
\end{figure}


\subsubsection{Privacy}
\label{sec:privacy_crypto}

Many cryptosystems that provide integrity guarantees are built upon
\textit{block ciphers} that operate on fixed-size message blocks. The sender
transforms a block using an \textit{encryption} algorithm, and the receiver
inverts the transformation using a \textit{decryption} algorithm. The
encryption algorithms in block ciphers obfuscate the message block's content in
the output, so that an adversary who does not have the decryption key cannot
obtain the original message block from the encrypted output.

Symmetric key encryption algorithms use the same secret key for encryption and
decryption, as shown in Figure~\ref{fig:symmetric_block_cipher}, while
asymmetric key block ciphers use the public key for encryption, and the
corresponding private key for decryption, as shown in
Figure~\ref{fig:asymmetric_block_cipher}.
\begin{figure}[hbt]
  \centering
  \includegraphics[width=85mm]{figures/symmetric_block_cipher.pdf}
  \caption{
    In a symmetric key secure permutation (block cipher), the same secret key
    must be provided to both the encryption and the decryption algorithm.
  }
  \label{fig:symmetric_block_cipher}
\end{figure}

\begin{figure}[hbt]
  \centering
  \includegraphics[width=85mm]{figures/asymmetric_block_cipher.pdf}
  \caption{
    In an asymmetric key block cipher, the encryption algorithm operates on a
    public key, and the decryption algorithm uses the corresponding private
    key.
  }
  \label{fig:asymmetric_block_cipher}
\end{figure}

The most popular block cipher based on symmetric keys at the time of this
writing is the
\textit{American Encryption Standard}~(AES)~\cite{daemen1999aes, fips2001aes},
with two variants that operate on 128-bit blocks using 128-bit keys or 256-bit
keys. AES is a \textit{secure permutation} function, as it can transform any
128-bit block into another 128-bit block. Recently, the United States
\textit{National Security Agency}~(NSA) required the use of 256-bit AES keys
for protecting sensitive information~\cite{nsa2015suiteb}.

The most deployed asymmetric key block cipher is the
\textit{Rivest-Shamir-Adelman}~(RSA)~\cite{rivest1978rsa} algorithm. RSA has
variable key sizes, and 3072-bit key pairs are considered to provide the same
security as 128-bit AES keys~\cite{fips2012keysize}.

A block cipher does not necessarily guarantee privacy, when used  on its own.
An easy to see issue is that in our previous example, a block cipher would
generate the same encrypted output for any of Alice's \textsc{buy} orders, as
they all have the same content. Furthermore, each block cipher has its own
assumptions that can lead to subtle vulnerabilities if the cipher is used
directly.

Symmetric key block ciphers are combined with operating modes to form symmetric
encryption schemes. Most operating modes require a random
\textit{initialization vector} (IV) to be used for each message, as shown in
Figure~\ref{fig:symmetric_encryption}. When analyzing the security of systems
based on these cryptosystems, an understanding of the IV generation process is
as important as ensuring the confidentiality of the encryption key.

\begin{figure}[hbt]
  \centering
  \includegraphics[width=85mm]{figures/symmetric_encryption.pdf}
  \caption{
    Symmetric key block ciphers are combined with operating modes. Most
    operating modes require a random initialization vector (IV) to be generated
    for each encrypted message.
  }
  \label{fig:symmetric_encryption}
\end{figure}

Counter (CTR) and Cipher Block Chaining (CBC) are examples of operating modes
recommended~\cite{fips2001ctr} by the United States \textit{National Institute
of Standards and Technology}~(NIST), which informs the NSA's requirements.
Combining a block cipher, such as AES, with an operating mode, such as CTR,
results in an encryption method, such as AES-CTR, which can be used to add
privacy guarantees.

In the asymmetric key setting, there is no concept equivalent to operating
modes. Each block cipher has its own assumptions, and requires a specialized
scheme for general-purpose usage.

The RSA algorithm is used in conjunction with \textit{padding methods}, the
most popular of which are the methods described in the \textit{Public-Key
Cryptography Standard} (PKCS) \#1 versions 1.5~\cite{kaliski1998pkcs1v15} and
2.0~\cite{kaliski1998pkcs1v2}. A security analysis of a system that uses
RSA-based encryption must take the padding method into consideration. For
example, the padding in PKCS \#1 v1.5 can leak the private key under certain
circumstances~\cite{bleichenbacher1998pkcs1v15cca}. While PKCS \#1 v2.0 solves
this issue, it is complex enough that some implementations have their own
security issues~\cite{manger2001pkcs1v20attack}.

Asymmetric encryption algorithms have much higher computational requirements
than symmetric encryption algorithms. Therefore, when non-trivial quantities of
data is encrypted, the sender generates a single-use secret key that is used
to encrypt the data, and encrypts the secret key with the receiver's public
key, as shown in Figure~\ref{fig:asymmetric_encryption}.

\begin{figure}[hbt]
  \centering
  \includegraphics[width=87mm]{figures/asymmetric_encryption.pdf}
  \caption{
    Asymmetric key encryption is generally used to bootstrap a symmetric
    key encryption scheme.
  }
  \label{fig:asymmetric_encryption}
\end{figure}


\subsubsection{Integrity}
\label{sec:integrity_crypto}

Many cryptosystems that provide integrity guarantees are built upon
\textit{secure hashing} functions. These hash functions operate on an unbounded
amount of input data and produce a small fixed-size output. Secure hash
functions have a few guarantees, such as \textit{pre-image resistance}, which
states that an adversary cannot produce input data corresponding to a given
hash output. The most popular secure hashing function at the time of this
writing is the \textit{Secure Hashing Algorithm}~(SHA)~\cite{eastlake2001sha1}.

In the symmetric key setting, integrity guarantees are typically obtained using
a \textit{Hash Message Authentication Code}~(HMAC)~\cite{krawczyk1997hmac}
construction built on top of a secure hash function, such as SHA. The message
sender runs the HMAC algorithm and sends the output \textit{HMAC tag} along
with the original message, as shown in Figure~\ref{fig:symmetric_hmac}. The
message receiver verifies that the received HMAC value is the same as the
output of running the HMAC algorithm on the message. HMAC algorithms inherit
the pre-image resistance property from their underlying secure hash functions,
and have the additional property that an adversary cannot produce the correct
HMAC for a message without the secret key.

\begin{figure}[hbt]
  \centering
  \includegraphics[width=87mm]{figures/symmetric_hmac.pdf}
  \caption{
    In the symmetric key setting, integrity is assured by computing a
    Hash-bassed Message Authentication Code (HMAC) and transmitting it over the
    network along the message. The receiver re-computes the HMAC and compares
    it against the version received from the network.
  }
  \label{fig:symmetric_hmac}
\end{figure}

Asymmetric key primitives that provide integrity guarantees are known as
\textit{signatures}. The message sender provides her private key to a
\textit{signing} algorithm, and transmits the output signature along with the
message, as shown in Figure~\ref{fig:asymmetric_signing}. The message receiver
feeds the sender's public key and the signature to a \textit{signature
verification} algorithm, which returns \textsc{true} if the message matches the
signature, and \textsc{false} if the message has been tampered with.

\begin{figure}[hbt]
  \centering
  \includegraphics[width=87mm]{figures/asymmetric_signing.pdf}
  \caption{
    Signature schemes guarantee integrity in the asymmetric key setting.
    Signatures are created using the sender's private key, and are verified
    using the corresponding public key. A cryptographically secure hash
    function is usually employed to reduce large messages to small hashes,
    which are then signed.
  }
  \label{fig:asymmetric_signing}
\end{figure}

Signing algorithms can only operate on small messages and are computationally
expensive. Therefore, in practice, the message to be transmitted is first ran
through a cryptographically strong hash function, and the hash is provided as
the input to the signing algorithm.

At the time of this writing, the most popular choice for guaranteeing integrity
in shared secret settings is HMAC-SHA, an HMAC function that uses SHA for
hashing.

\textit{Authenticated encryption}, which combines a block cipher with an
operating mode that offers both privacy and integrity guarantees, is often an
attractive alternative to HMAC. The most popular authenticated encryption
operating mode is \textit{Galois/Counter operation
mode}~(GCM)~\cite{mcgrew2004gcm}, which has earned NIST's
recommendation~\cite{fips2017gcm} when combined with AES to form AES-GCM.

The most popular signature algorithm is based on the RSA algorithm. Recently,
elliptic curve cryptography (ECC)~\cite{koblitz1987ecc} has gained a surge in
popularity, thanks to its smaller key sizes. For example, a 384-bit ECC key is
considered to have the same security as a 3072-bit RSA
key~\cite{fips2012keysize, nsa2015suiteb}. The NSA requires the Digital
Signature Standard~(DSS)\cite{fips2013dss}, which specifies schemes based on
RSA and on ECC.


\subsubsection{Freshness}
\label{sec:freshness_crypto}

Freshness guarantees are typically built on top of a system that already offers
integrity guarantees, by adding a unique piece of information to each message.
The main challenge in freshness schemes comes down to economically maintaining
the state needed to generate the unique pieces of information on the sender
side, and verify their uniqueness on the receiver side.

A popular solution for gaining freshness guarantees relies on \textit{nonces},
single-use random numbers. The attractiveness of nonces is that the sender does
not need to maintain any state, but the downside is that the receiver must
store the nonces of all received messages.

Nonces are often combined with a message timestamping and expiration scheme, as
shown in Figure~\ref{fig:timestamped_nonces}. An expiration can greatly reduce
the receiver's storage requirement, as the nonces for expired messages can be
safely discarded. However, the scheme depends on the sender and receiver having
synchronized clocks. The message expiration time is a compromise between the
desire to reduce storage costs and the need to tolerate clock skew and
message transmission and processing delays.

\begin{figure}[hbt]
  \centering
  \includegraphics[width=87mm]{figures/timestamped_nonces.pdf}
  \caption{
    Freshness guarantees can be obtained by adding timestamped nonces on top
    of a system that already offers integrity guarantees. The sender and the
    receiver use synchronized clocks to timestamp each message and discard
    unreasonably old messages. The receiver must check the nonce in each new
    message against a database of the nonces in all the unexpired messages that
    it has seen.
  }
  \label{fig:timestamped_nonces}
\end{figure}

Alternatively, nonces can be used in challenge-response protocols, in a manner
that removes the storage overhead concerns. The challenger generates a nonce
and embeds it in the challenge message. The response to the challenge includes
an acknowledgement of the embedded nonce, so the challenger can distinguish
between a fresh response and a replay attack. The nonce is only stored by the
challenger, and is small in comparison to the rest of the state needed to
validate the response.

\subsection{Cryptographic Constructs}
\label{sec:crypto_constructs}

This section summarizes three constructs that are built on the cryptographic
primitives described in \S~\ref{sec:crypto_primitives}, and are used in the
rest of this work.


\subsubsection{Certificate Authorities}

Asymmetric key cryptographic primitives assume that each party has the correct
public keys for the other parties. This assumption is critical, as the entire
security argument of an asymmteric key system rests on the fact that certain
operations can only be performed by the owners of the private keys
corresponding to the public keys. More concretely, if Eve can convince Bob
that her own public key belongs to Alice, Eve can produce message signatures
that seem to come from Alice.

The introductive material in \S~\ref{sec:crypto_primitives} assumed that each
party transmits its public key over a channel with integrity guarantees. In
practice, this is not a reasonable assumption, and the secure distribution of
public keys is still an open research problem.

The most widespread solution to the public key distribution problem is the
Certificate Authority (CA) system, which assumes the existence of a trusted
authority whose public key is securely transmitted to all the other parties in
the system.

The CA is responsible for securely obtaining the public key of each party, and
for issuing a \textit{certificate} that binds a party's identity (e.g.,
``Alice'') to its public key, as shown in Figure~\ref{fig:certificate}.

\begin{figure}[hbt]
  \centering
  \includegraphics[width=85mm]{figures/certificate.pdf}
  \caption{
    A certificate is a statement signed by a certificate authority (issuer)
    binding the identity of a subject to a public key.
  }
  \label{fig:certificate}
\end{figure}

A certificate is essentially a cryptographic signature produced by the private
key of the certificate's \textit{issuer}, who is generally a CA. The message
signed by the issuer states that a public key belongs to a \textit{subject}.
The certificate message generally contains identifiers that state the intended
use of the certificate, such as ``the key in this certificate can only be used
to sign e-mail messages''. The certificate message usually also includes an
identifier for the issuer's \textit{certification policy}, which summarizes the
means taken by the issuer to ensure the authenticity of the subject's public
key.

A major issue in a CA system is that there is no obvious way to revoke a
certificate. A revocation mechanism is desirable to handle situations where a
party's private key is accidentally exposed, to avoid having an attacker use
the certificate to impersonate the compromised party. While advanced systems
for certificate revocation have been developed, the first line of defense
against key compromise is adding expiration dates to certificates.

In a CA system, each party presents its certificate along with its public key.
Any party that trusts the CA and has obtained the CA's public key securely can
verify any certificate using the process illustrated in
Figure~\ref{fig:certificate_validation}.

\begin{figure}[hbt]
  \centering
  \includegraphics[width=80mm]{figures/certificate_validation.pdf}
  \caption{
    A certificate issued by a CA can be validated by any party that has
    securely obtained the CA's public key. If the certificate is valid, the
    subject public key contained within can be trusted to belong to the subject
    identified by the certificate.
  }
  \label{fig:certificate_validation}
\end{figure}

One of the main drawbacks of the CA system is that the CA's private key becomes
a very attractive attack target. This issue is somewhat mitigated by minimizing
the use of the CA's private key, which reduces the opportunities for its
compromise. The authority described above becomes the \textit{root CA}, and its
private key is only used to produce certificates for the
\textit{intermediate CAs} which, in turn, are responsible for generating
certificates for the other parties in the system, as shown in
Figure~\ref{fig:intermediate_cas}.

\begin{figure}[hbtp]
  \centering
  \includegraphics[width=75mm]{figures/intermediate_cas.pdf}
  \caption{
    A hierarchical CA structure minimizes the usage of the root CA's private
    key, reducing the opportunities for it to get compromised. The root CA only
    signs the certificates of intermediate CAs, which sign the end users'
    certificates.
  }
  \label{fig:intermediate_cas}
\end{figure}

In hierarchical CA systems, the only public key that gets distributed securely
to all the parties is the root CA's public key. Therefore, when two parties
wish to interact, each party must present its own certificate, as well as the
certificate of the issuing CA. For example, given the hierarchy in
Figure~\ref{fig:intermediate_cas}, Alice would prove the authenticity of her
public key to Bob by presenting her certificate, as well as the certificate of
Intermediate CA 1. Bob would first use the steps in
Figure~\ref{fig:certificate_validation} to validate Intermediate CA 1's
certificate against the root CA's public key, which would assure him of the
authenticity of Intermediate CA 1's public key. Bob would then validate Alice's
certificate using Intermediate CA 1's public key, which he now trusts.

In most countries, the government issues ID cards for its citizens, and
therefore acts as as a certificate authority. An ID card, shown in
Figure~\ref{fig:id_card_as_certificate}, is a certificate that binds a
subject's identity, which is a full legal name, to the subject's physical
appearence, which is used as a public key.

The CA system is very similar to the identity document (also known as ID card)
systems used to establish a person's identity, and a comparison between the two
may help further the reader's understanding of the concepts in the CA system.

\begin{figure}[hbt]
  \centering
  \includegraphics[width=85mm]{figures/id_card_as_certificate.pdf}
  \caption{
    An ID card is a certificate that binds a subject's full legal name
    (identity) to the subject's physical appeareance, which acts as a public
    key.
  }
  \label{fig:id_card_as_certificate}
\end{figure}

Each government's ID card issuing operations are regulared by laws, so an ID
card's issue date can be used to track down the laws that make up its
certification policy. Last, the security of ID cards does not (yet) rely on
cryptographic primitives. Instead, ID cards include physical security measures
designed to deter tampering and prevent counterfeiting .


\subsubsection{Key Agreement Protocols}

\subsubsection{Cryptography for Storage}

\subsection{SGX Software Attestation}
\label{sec:sgx_attestation}

The software attestation scheme implemented by SGX follows the principles
outlined in \S~\ref{sec:generic_software_attestation}. An SGX-enabled processor
computes a measurement of the code and data that is loaded in each enclave,
which is similar to the measurement computed by the TPM~(\S~\ref{sec:tpm}). The
software inside an enclave can start a process that results in an SGX
attestation signature, which includes the enclave's measurement and an enclave
message.

\begin{figure}[hbt]
  \centering
  \includegraphics[width=85mm]{figures/sgx_attestation_overview.pdf}
  \caption{
    Setting up an SGX enclave and undergoing the software attestation process
    involves the SGX instructions \texttt{EINIT} and \texttt{EREPORT}, and two
    special enclaves authored by Intel, the SGX Launch Enclave and the SGX
    Quoting Enclave.
  }
  \label{fig:sgx_attestation_overview}
\end{figure}

% EPID signing takes too long for microcode.
%   US 8,972,746 B2 - 33:16-29, 33:58

The cryptographic primitive used in SGX's attestation signature is too complex
to be implemented in hardware, so signing process is performed by a privileged
\textit{Quoting Enclave}, which is issued by Intel, and can access the SGX
attestation key. This enclave is discused in \S~\ref{sec:sgx_quoting_enclave}.

Pushing the signing functionality into the Quoting Enclave creates the need for
a secure communication path between an enclave that desires to undergo
software attestation and the Quoting Enclave. The SGX design solves this
problem with a local attestation mechanism that can be used by an enclave to
prove its identity to any other enclave hosted by the same SGX-enabled CPU.
This scheme, described in \S~\ref{sec:sgx_ereport}, is implemented by the
\texttt{EREPORT} instruction.

% Keys: SDM S 39.4.3
% ISCA SGX Slides 104, 105, 106

The SGX attestation key used by the Quoting Enclave does not exist at the time
SGX-enabled processors leave the factory. The attestation key is provisioned
later, using a largely undocumented process that is known to involve at least
one other enclave issued by Intel, and at least two special \texttt{EGETKEY}
key types. The publicly available details of this process are summarized in
\S~\ref{sec:sgx_quoting_enclave}.

The SGX implementation contained in a CPU's hardware does not directly enforce
the enclave attribute checks that decide which enclaves can access the CPU
secrets used for software attestation. The potentially complex restrictions on
enclave attributes are instead enforced by the \textit{Launch Enclave}, which
is an enclave issued by Intel that gets to approve every other enclave before
it is initialized by \textit{EINIT}~(\S~\ref{sec:sgx_einit_overview}). The
officially documented information about the approval process is discussed in
\S~\ref{sec:sgx_launch_enclave}.

% Enclave License
%   US 8,972,746 B2 - 34:6-23
% Licenses are evaluated into Permits (which became EINITTTOKEN)
%   US 8,972,746 B2 - 34:24-29, 35:27-59
% EINIT requires a Permit to launch a production enclave
%   US 8,972,746 B2 - 35:60-67, 36:1-3, 36:47-52, 38:4-65
% License Enclave creates Permit
%   US 8,972,746 B2 - 36:40-46, 37:50-67, 38:1-3
% EMKPERMIT seems to have gotten merged into EINIT
%   US 8,972,746 B2 - 36:53-67, 37:1-7, 37:24-49
% License became SIGSTRUCT
%   US 8,972,746 B2 - 37:8-23

One of the SGX patents~\cite{intel2013patent1} discloses in no uncertain terms
that the Launch Enclave was introduced to ensure that each enclave's author has
a business relationship with Intel, and implements a software licensing system.
\S~\ref{sec:sgx_licensing} briefly discusses the implications, should this turn
out to be true.


\subsubsection{Local Attestation}
\label{sec:sgx_ereport}

An enclave proves its identity to another \textit{target enclave} via the
\texttt{EREPORT} instruction shown in Figure~\ref{fig:sgx_ereport}. The SGX
instruction produces an attestation \textit{Report} (REPORT) that
cryptographically binds a message supplied by the enclave with the enclave's
measurement-based~(\S~\ref{sec:sgx_measurement}) and
certificate-based~(\S~\ref{sec:sgx_certificate_identity}) identities. The
cryptographic binding is accomplished by a MAC tag computed using a symmetric
key that is only shared between the target enclave and the SGX implementation.

\begin{figure}[hbt]
  \centering
  \includegraphics[width=85mm]{figures/sgx_ereport.pdf}
  \caption{
    \texttt{EREPORT} data flow
  }
  \label{fig:sgx_ereport}
\end{figure}

The \texttt{EREPORT} instruction reads the current enclave's identity
information from the enclave's SECS~(\S~\ref{sec:sgx_secs}), and uses it to
populate the REPORT structure. Specifically, \texttt{EREPORT} copies the
SECS fields indicating the enclave's measurement (MRENCLAVE), certificate-based
identity (MRSIGNER, ISVPROD, ISVSVN), and attributes (ATTRIBUTES). The
attestation report also includes the SVN of the SGX implementation (CPUSVN)
and a 64-byte (512-bit) message supplied by the enclave.

The target enclave that receives the attestation report can convince itself of
the report's authenticy by verifying its MAC tag, using a key obtained by
invoking \texttt{EGETKEY}~(\S~\ref{sec:sgx_egetkey}) and asking for a report
key. \texttt{EREPORT} uses the same key derivation process as \texttt{EGETKEY}
does when invoked with KEYNAME set to the value that represents a Report Key.

The report key returned by \texttt{EGETKEY} is derived from a secret embedded
in the processor~(\S~\ref{sec:sgx_egetkey}), and the key material includes the
target enclave's measurement. The target enclave can be assured that the MAC
tag in the report was produced by the SGX implementation, for the following
reasons. The cryptographic properties of the underlying key derivation and
MAC algorithms ensure that only the SGX implementation can produce the MAC tag,
as it is the only entity who can access the processor's secret, and it would be
impossible for an attacker to derive the report key without knowing the
processor's secret. The SGX design guarantees that key produced by
\texttt{EGETKEY} depends on the calling enclave's measurement, so only the
target enclave can obtain the key used to produce the MAC tag in the report.

\texttt{EREPORT} requires the virtual address of a
\textit{Report Target Info}~(TARGETINFO) structure that contains the
measurement-based identity and attributes of the target enclave. These values
are used to derive the same key that \texttt{EGETKEY} would.

When deriving a report key, \texttt{EGETKEY} does not use the fields
corresponding to the enclave's certificate-based identity (MRSIGNER, ISVPRODID,
ISVSVN) in the key generation material. However, it does use the SVN of the
SGX implementation, which is also written to the REPORT structure.

Last, \texttt{EREPORT} sets the KEYID field in the key generation material to
the contents of an SGX configuration register (CR\_REPORT\_KEYID) that is
initialized with a random value when SGX is initialized. The KEYID value is
also saved in the attestation report, but it is not covered by the MAC tag.


\subsubsection{Enclave Approval}
\label{sec:sgx_launch_enclave}

% ATTRIBUTES: SDM S 38.7.1
% EINIT: SDM S 41.3
% EINITTOKENKEY is bit 5, INTEL_ONLY_MASK is 0x20

Instead of implementing potentially complex access checks

% EINIT Token Structure (EINITTOKEN): SDM S 38.14

\begin{figure}[hbt]
  \centering
  \includegraphics[width=87mm]{figures/sgx_einittoken.pdf}
  \caption{
    The SGX Launch Enclave computes the EINITTOKEN.
  }
  \label{fig:sgx_einittoken}
\end{figure}

The SDM states that the MAC that protects EINITTOKEN's authenticity is computed
using a block cipher-based MAC~(CMAC,~\cite{fips2005cmac}), but stops short of
specifying the underlying cipher. One of the SGX papers~\cite{anati2013sgx}
states that SGX implementation uses a CMAC based on 128-bit AES.



% MRSIGNER: SDM S 39.4.1.2

After an enclave gets cleared by the SGX Launch Enclave and is initialized via
\texttt{EINIT}, it can authenticate itself to a remote party by participating
in a software attestation process.


\subsubsection{The Quoting Enclave}
\label{sec:sgx_quoting_enclave}

While the SDM paints a complete picture of the local attestation mechanism, it
is a lot more secretive about the Quoting Enclave and the underlying keys.
Fortunately, the SGX patents~\cite{intel2013patent1, intel2013patent2} shed
some light on the topic.



\subsubsection{Licensing}
\label{sec:sgx_licensing}

However, the software attestation scheme in SGX's design, summarized in
Figure~\ref{fig:sgx_attestation_overview} is unnecessarily complicated by the
decision to deeply couple software attestation with
\textbf{an enclave licensing mechanism that allows Intel to force itself as an
intermediary in the distribution of all enclave software}.

\subsection{Physical Attacks}
\label{sec:physical_attacks}

Physical attacks are generally classified according to their cost, which
factors in the equipment needed to carry out the attack and the attack's
complexity. Joe Grand's DefCon presentation~\cite{grand2004physicalattacks}
provides a good overview with a large number of intuition-building pictures.

The simplest type of physical attack is a denial of service attack performed by
disconnecting the victim computer's power supply or network cable. The threat
models of most secure architectures ignore this attack, because denial of
service can also be achieved by software attacks that compromise the computer's
system software, such as the hypervisor.


\subsubsection{Port Attacks}

Slightly more involved attacks rely on connecting a device to an existing port
on the victim computer's case or motherboard~(\S~\ref{sec:motherboard}). A
simple example is a \textit{cold boot attack}, where the attacker plugs in a
USB flash drive into the victim's case and causes the computer to boot from
the flash drive, whose malicious system software receives unrestricted access
to the computer's peripherals.

More expensive physical attacks that still require relatively little effort
target the debug ports of various peripherals. The cost of these attacks is
generally dominated by the expense of acquiring the development kits needed to
connect to the debug ports. For example, recent Intel processors include the
Generic Debug eXternal Connection~(GDXC)~\cite{yuffe2011sandybridge,
intel2011gdxc}, which collects and filters the data transfered by the uncore's
ring bus (\S~\ref{sec:cache_coherence}), and reports it to an external
debugger.

The threat models of secure architectures generally ignore debug port attacks,
under the assumption that devices sold for general consumption have their debug
ports irreversibly disabled. In practice, manufacturers have strong incentives
to preserve debugging ports in production hardware, as this facilitates the
diagnosis and repair of defective units. Due to insufficient documentation on
this topic, we ignore the possibility of GDXC-based attacks. Fortunately, a
recent Intel patent~\cite{shanbhogue2015gdxcsgx} indicates that Intel engineers
are tackling at least some classes of attacks targeting debugging ports.


\subsubsection{Bus Tapping Attacks}

More complex physical attacks consist of installing a device that taps a bus on
the computer's motherboard (\S~\ref{sec:motherboard}). \textit{Passive attacks}
are limited to monitoring the bus traffic, whereas \textit{active attacks} can
modify the traffic, or even place new commands on the bus. \textit{Replay
attacks} are a notoriously difficult to defeat class of active attacks, where
the attacker first records the bus traffic, and then selectively replays a
subset of the traffic. Replay attacks bypass systems that rely on static
signatures or HMACs, and generally aim to double-spend a limited resource.

The cost of bus tapping attacks is generally dominated by the cost of the
equipment used to tap the bus, which increases with bus speed and complexity.
For example, the flash chip that stores the computer's firmware is connected to
the PCH via an SPI bus (\S~\ref{sec:motherboard}), which is simpler and much
slower than the DDR bus connecting DRAM to the CPU. Consequently, tapping the
SPI bus is much cheaper than tapping the DDR bus. For this reason, systems
whose security relies on a cryptographiic hash of the firmware will first copy
the firmware into DRAM, hash the DRAM copy of the firmware, and then execute
the firmware from DRAM.

Although the DDR bus's speed makes tapping it very difficult, there are
well-publicized records of successful attempts. The original Xbox console's
booting process was reverse-engineered thanks to a passive tap on the DRAM
bus~\cite{huang2003xbox} that showed that the firmware used to boot the
console was partially stored in its southbridge. The protection mechanisms of
the PlayStation 3 hypervisor were subverted by an active tap on its memory
bus~\cite{hotz2010ps3} that targeted the hypervisor's page tables.


\subsubsection{Chip Tampering Attacks}


The CPU's chip
package is generally considered to be a minimal trust boundary, because
physical attacks that can breach the CPU modify
The most
rigurous threat models only assume that


\subsubsection{Power Analysis Attacks}

\HeadingLevelB{Privileged Software Attacks}
\label{sec:system_software_attacks}

The rest of this section points to successful exploits that execute at each of
the privilege levels described in \S~\ref{sec:rings}, motivating the SGX design
decision to assume that all the privileged software on the computer is
malicious. \cite{rutkowska2015intelsux} describes all the programmable hardware
inside Intel computers, and outlines the security implications of compromising
the software running it.

SMM, the most privileged execution level, is only used to handle a specific
kind of interrupts (\S~\ref{sec:interrupts}), namely
\textit{System Management Interrupts} (SMI). SMIs were initially designed
exclusively for hardware use, and were only triggered by asserting a dedicated
pin (SMI\#) in the CPU's chip package. However, in modern systems, system
software can generate an SMI by using the LAPIC's IPI mechanism. This opens up
the avenue for SMM-based software exploits.

% System Management Mode: SDM S 34
% SMRAM: SDM S 34.4
% SMRAM Caching: SDM S 34.4.2

The SMM handler is stored in  \textit{System Management RAM} (SMRAM) which, in
theory, is not accessible when the processor isn't running in SMM. However, its
protection mechanisms were bypassed multiple times~\cite{duflot2006smm,
rutkowska2008remap, wojtczuk2009smm, kallenberg2014smm}, and SMM-based
rootkits~\cite{wecherowski2009smm, embleton2010smm} have been demonstrated.
Compromising the SMM grants an attacker access to all the software on the
computer, as SMM is the most privileged execution mode.

Xen \cite{zhang2008xen} is a very popular representative of the family of
hypervisors that run in VMX root mode and use hardware virtualization. At
150,000 lines of code~\cite{xen2015loc}, Xen's codebase is relatively small,
especially when compared to a kernel. However, Xen still has had over 40
security vulnerabilities patched in \textbf{each} of the last three years
(2012-2014) \cite{cvedetails2014xen}.

\cite{mccune2010trustvisor} proposes using a very small hypervisor together
with Intel TXT's dynamic root of trust for measurement (DRTM) to implement
trusted execution. \cite{vasudevan2010requirements} argues that a dynamic root
of trust mechanism, like Intel TXT, is necessary to ensure a hypervisor's
integrity.  Unfortunately, the TXT design requires an implementation complex
enough that exploitable security vulnerabilities have creeped in
\cite{wojtczuk2009txt2, wojtczuk2011txt}. Furthermore, any SMM attack can be
used to compromise TXT \cite{wojtczuk2009txt}.

The monolithic kernel design leads to many opportunities for security
vulnerabilities in kernel code. Linux is by far the most popular kernel for
IaaS cloud environments. Linux has \emph{17 million} lines of
code~\cite{anthony2014linuxsize}, and  has had over 100 security
vulnerabilities patched in \textbf{each} of the last three years
(2012-2014)~\cite{cvedetails2014linux, chen2011linux}.

\subsection{Software Attacks on Peripherals}
\label{sec:device_attacks}

The SGX trusted computing base includes the processor package, and excludes the
other hardware in the computer. It follows that SGX must be able to fend off
attacks from rogue devices, such as the PCIe NIC used to compromise Intel TXT
\cite{wojtczuk2011txt}.


the rowhammer DRAM bit-flipping attack
\cite{kim2014rowhammer, google2015rowhammer, gruss2015rowhammer}.


The SGX threat model explicitly considers SMM to be untrusted. However, it does
not account for malicious code running on the Management Engine. Unfortunately,
the ME, PCH and DMI are Intel-proprietary and largely undocumented, so we
cannot assess the impact of an ME attack on software running inside an SGX
enclave.

The flash memory stores the SMM handler and Intel ME firmware, which run with
high privileges. Furthermore, the contents of flash memory persists across
power cycles. This opens up the possibility for an attacker that gains SPI
access to deploy a persistent payload that runs at high privilege. To prevent
against these attacks, most of the firware is signed. For example, the ME
checks that its firmware was signed by a burned-in Intel public key. However,
both the computer firmware checks \cite{wojtczuk2010bios, furtak2014bios} and
the ME firmware checks \cite{tereshkin2009amt} have been subverted in the past.


\subsection{Address Translation Attacks}
\label{sec:paging_attacks}

\S~\ref{sec:system_software_attacks} argues that today's system software is
virtually guaranteed to have security vulnerabilities. This suggests that a
cautious secure architecture shoild avoid having the system software in the
TCB.

However, removing the system software from the TCB requires that the
architecture provides a method for isolating sensitive application code from
the untrusted system software. This is typically accomplished by a designing a
mechanism for loading application code in isolated containers whose contents
can be certified via software
attestation~(\S~\ref{sec:generic_software_attestation}). One of the more
difficult problems faced by these designes is that application software relies
on the memory management services provided by the system software, which is now
untrusted.

Intel's SGX~\cite{mckeen2013sgx, anati2013sgx}, which was inspired by
Bastion~\cite{champagne2010bastion}, leaves the system software in charge of
setting up the page tables (\S~\ref{sec:paging}) used by address translation,
but instates access checks that prevent the system software from directly
accessing the isolated container's memory.

This section discusses some attacks that become relevant when the application
software does not trust the system software which in charge of the page tables.
Understanding these attacks is a prerequisite to reasoning about the security
properties of architectures with this threat model. For example, a large amount
of the mechanisms in SGX are aimed at dealing with a subset of the attacks
described here.


\subsubsection{Passive Attacks}
\label{sec:fault_tracking_attacks}

System software uses the the CPU's address translation feature
(\S~\ref{sec:paging}) to implement page swapping, where infrequently used
memory pages are evicted from DRAM to a slower storage medium. Page swapping
relies the accessed (A) and dirty (D) page table entry attributes
(\S~\ref{sec:page_table_attributes}) to identify the DRAM pages to be evicted,
and on a page fault handler (\S~\ref{sec:faults}) to bring evicted pages back
into DRAM when they are accessed.

Unfortunately, the features that support efficient page swapping turn into a
security liability, when the system software managing the page tables is not
trusted by the application software using the page tables. The system software
can be prevented from reading the application's memory directly by placing the
application in an isolated container. However, potentially malicious system
software can still infer partial information about the application's memory
access patterns, by observing the application's page faults and page table
attributes.

We consider this class of attacks to be passive attacks that exploit the CPU's
address translation feature. It may seem that the page-level memory access
patterns provided by these attacks are not very useful. However,
\cite{xu2015pagefaults} describes how this attack can be carried out against
Intel's SGX, and implements the attack in a few practical settings. In one
scenario, which is particularly concerning for medical image processing,
the outline of a JPEG image is inferred while the image is decompressed inside
a container protected by SGX's isolation guarantees.


\subsubsection{Active Attacks}
\label{sec:memory_mapping_attacks}

We define active address translation attacks to be the class of attacks where
malicious system software modifies the page tables used by an application in
a way that breaks the virtual memory abstraction (\S~\ref{sec:paging}). Memory
mapping attacks do not include scenarios where the system software breaks the
memory abstraction by directly writing to the application's memory pages.

We begin with an example of a straight-forward active attack. In this example,
the application inside a protected container performs a security check to
decide whether to disclose some sensitive information. Depending on the
security check's outcome, the enclave code either calls a \texttt{errorOut}
procedure, or a \texttt{disclose} procedure. The simplest vesion of the attack
assumes that each procedure's code starts at a page boundary, and takes up less
than a page. These assumptions are relaxed in more complex versions of the
attack.

In the most straight-forward setting, the malicious system software directly
modifies the page tables of the application inside the container, as shown in
Figure~\ref{fig:active_mapping_attack}, so the virtual address intended to
store the \texttt{errorOut} procedure is actually mapped to a DRAM page that
contains the \texttt{disclose} procedure. Without any security measures in
place, when the application's code jumps to the virtual address of the
\texttt{errorOut} procedure, the CPU will execute the code of the
\texttt{disclose} procedure instead.

\begin{figure}[hbt]
  \centering
  \includegraphics[width=85mm]{figures/active_mapping_attack.pdf}
  \caption{
    An example of an active memory mapping attack. The application's author
    intends to peform a security check, and only disclose a piece of sensitive
    information if the check passes. Malicious system software maps the virtual
    address of the procedure called when the security check fails to a DRAM
    page that contains the procedure that discloses the sensitive information,
    which is supposed to be called when the security check passes.
  }
  \label{fig:active_mapping_attack}
\end{figure}

The most obvious active attacks on memory mapping can be defeated by tracking
the correct virtual address for each DRAM page that belongs to a protected
container. However, a naive protection measure based on address tracking can be
defeated by a more subtle active attack that relies on the architectural
support for page swapping. Figure~\ref{fig:swap_mapping_attack} illustrates an
attack that does not modify the application's page tables, but produces the
same corrputed CPU view of the application as the straight-forward attack
described above.

\begin{figure}[hbt]
  \centering
  \includegraphics[width=85mm]{figures/swap_mapping_attack.pdf}
  \caption{
    An active memory mapping attack where the system software does not modify
    the page tables. Instead, two pages are evicted from DRAM to a slower
    storage medium. The malicious system software swaps the two pages' contents
    then brings them back into DRAM, building the same incorrect page mapping
    as the direct attack shown in Figure~\ref{fig:active_mapping_attack}. This
    attack defeats protection measures that rely on tracking the virtual and
    disk addresses for DRAM pages.
  }
  \label{fig:swap_mapping_attack}
\end{figure}

In the swapping attack, malicious system software evicts the pages that contain
the \texttt{errorOut} and \texttt{disclose} procedures from DRAM to a slower
medium, such as a hard disk. The system software exchanges the hard disk
bytes storing the two pages, and then brings the two pages back into DRAM.
Remarkably, all the steps taken by this attack are indistinguishable from
legitimate page swapping activity, with the exception of the I/O operations
that exchange the disk bytes that store evicted pages.

The subtle attack described above can be defeated by cryptographically
binding the contents of each page that is evicted from DRAM to the virtual
address that the page should be mapped to. The cryptographic primitive
(\S~\ref{sec:crypto_primitives}) used to perform the binding must obiously
guarantee integrity. Furthermore, it must also guarantee freshness, in order
to foil replay attacks where the system software ``undoes'' an application's
writes by evicting one of its DRAM pages to disk and bringing in an older
version of the same page.

Today's multi-core architectures can be subjected to an even more subtle active
attack, illustrated in Figure~\ref{fig:tlb_mapping_attack}, which can bypass
any protection measures that solely focus on the integrity of the page tables.

\begin{figure}[hbt]
  \centering
  \includegraphics[width=75mm]{figures/tlb_mapping_attack.pdf}
  \caption{
    An active memory mapping attack where the system software does not
    invalidate a core's TLBs when it evicts two pages from DRAM and exchanges
    their locations when reading them back in. The page tables are updated
    correctly, but the core with stale TLB entries has the same incorrect view
    of the protected container's code as in
    Figure~\ref{fig:active_mapping_attack}.
  }
  \label{fig:tlb_mapping_attack}
\end{figure}

For performance reasons, each execution core caches address translation results
in its own translation look-aside buffer~(TLB,~\S~\ref{sec:tlbs}). For
simplicity, the TLBs are not covered by the cache coherence protocol that
synchronizes data caches across cores. Instead, the system software is
responsible for invalidating TLB entries across all the cores when it modifies
the page tables.

Malicious system software can take advantage of the design decisions explained
above by carring out the following attack. While the same software used in the
previous examples is executing on a core, the system software executes on a
different core and evicts the \texttt{errorOut} and \texttt{disclose} pages
from DRAM. Like in the previous attack, the system software loads the
\texttt{disclose} code in the DRAM page that previously held \texttt{errorOut}.
In this attack, however, the system software also updates the page tables.

The core where the system software executed sees the code that the application
developer intended. Therefore, the attack will pass any security checks that
rely cryptographic associations between page contents and page table data, as
long as the checks are performed by the core used to load pages back into DRAM.
However, the core that executes the protected container's code still uses the
old page table data, because the system software did not invalidate its TLB
entries. Assuming the TLBs are not subjected to any additional security checks,
this attack causes the same private information leak as the previous examples.

In order to avoid the last attack described in this section, the trusted
software or hardware that implements protected containers must also ensure that
the system software invalidates the relevant TLB entries on all the cores when
it evicts a page from a protected container to DRAM.

\HeadingLevelB{Cache Timing Attacks}
\label{sec:cache_timing}

Cache timing attacks~\cite{banescu2011cache} are a powerful class of software
attacks that can be mounted entirely by application code running at ring 3
(\S~\ref{sec:rings}). Cache timing attacks do not learn information by reading
the victim's memory, so they bypass the address translation-based isolation
measures (\S~\ref{sec:paging}) implemented in today's kernels and hypervisors.


\HeadingLevelC{Theory}

Cache timing attacks exploit the unfortunate dependency between the location of
a memory access and the time it takes to perform the access. A cache miss
requires at least one memory access to the next level cache, and might require
a second memory access if a write-back occurs. On the Intel architecture, the
latency between a cache hit and a miss can be easily measured by the
\texttt{RDTSC} and \texttt{RDTSCP} instructions (\S~\ref{sec:address_spaces}),
which read a high-resolution time-stamp counter. These instructions have been
designed for benchmarking and optimizing software, so they are available to
ring 3 software.

The fundamental tool of a cache timing attack is an attacker process that
measures the latency of accesses to carefully designated memory locations in
its own address space. The memory locations are chosen so that they map to
the same cache lines as some interesting memory locations in a victim process,
in a cache that is shared between the attacker and the victim. This requires
in-depth knowledge of the shared cache's organization (\S~\ref{sec:cache_org}).

Armed with the knowledge of the cache's organization, the attacker process
sets up the attack by accessing its own memory in such a way that it fills up
all the ways in the cache sets that would hold the victim's interesting memory
locations. After the targeted cache sets are full, the attacker allows the
victim process to execute. When the victim process accesses an interesting
memory locations in its own address space, the shared cache must evict one of
the cache lines holding the attacker's memory locations.

As the victim is executing, the attacker process repeatedly times accesses to
its own memory locations. When the access times indicate that a location was
evicted from the cache, the attacker can conclude that the victim accessed an
interesting memory location in its own cache. Over time, the attacker collects
the results of many measurements and learns a subset of the victim's memory
access pattern. If the victim processes sensitive information using
data-dependent memory fetches, the attacker may be able to deduce the sensitive
information from the learned memory access pattern.


\HeadingLevelC{Practical Considerations}

Cache timing attacks require control over a software process that shares a
cache memory with the victim process. Therefore, a cache timing attack that
aims at the L2 cache would have to rely on the system software to schedule a
software thread on a logical processor in the same core as the target software,
whereas an attack on the L3 cache can be performed using any logical processor
on the same CPU. The latter attacks rely on the fact that the L3 cache is
inclusive, which greatly simplifies the processor's cache coherence
implementation (\S~\ref{sec:cache_coherence}).

The cache sharing requirement implies that L3 cache attacks are feasible in an
IaaS environment, whereas L2 cache attacks become a significant concern when
running sensitive software on a user's desktop.

Out-of-order execution (\S~\ref{sec:out_of_order}) can introduce noise in cache
timing attacks. First, memory accesses may not be performed in program order,
which can impact the lines selected by the cache eviction algorithms. Second,
out-of-order execution may result in cache fills that do not correspond to
executed instructions. For example, a load that follows a faulting instruction
may be scheduled and executed before the fault is detected.

Cache timing attacks must account for speculative execution, as mispredicted
memory accesses can still cause cache fills. Therefore, the attacker may
observe cache fills that don't correspond to instructions that were actually
executed by the victim software. Memory prefetching adds further noise to cache
timing attacks, as the attacker may observe cache fills that don't correspond
to instructions in the victim code, even when accounting for speculative
execution.


\HeadingLevelC{Known Cache Timing Attacks}

Despite these difficulties, cache timing attacks are known to retrieve
cryptographic keys used by AES~\cite{osvik2006aes, bonneau2006aes},
RSA~\cite{brumley2005rsa}, Diffie-Hellman~\cite{kocher1996timing}, and
elliptic-curve cryptography~\cite{brumley2011ecc}.

Early attacks required access to the victim's CPU core, but more sophisticated
recent attacks~\cite{yarom2013llctiming, liu2015llctiming} are able to use the
L3 cache, which is shared by all the cores on a CPU die. L3-based attacks can
be particularly devastating in cloud computing scenarios, where running
software on the same computer as a victim application only requires modest
statistical analysis skills and a small amount of
money~\cite{ristenpart2009colocation}. Furthermore, cache timing attacks were
recently demonstrated using JavaScript code in a page visited by a Web
browser~\cite{oren2015jstiming}.

Given this pattern of vulnerabilities, ignoring cache timing attacks is
dangerously similar to ignoring the string of demonstrated attacks which led to
the deprecation of SHA-1~\cite{nist2014sha1policy, google2014sha1deprecation,
microsoft2014sha1deprecation}.


\HeadingLevelC{Defending against Cache Timing Attacks}
\label{sec:cache_timing_workarounds}

Fortunately, invalidating any of the preconditions for cache timing attacks is
sufficient for defending against them. The easiest precondition to focus on is
that the attacker must have access to memory locations that map to the same
sets in a cache as the victim's memory. This assumption can be invalidated by
the judicious use of a cache partitioning scheme.

Performance concerns aside, the main difficulty associated with cache
partitioning schemes is that they must be implemented by a trusted party. When
the system software is trusted, it can (for example) use the principles behind
page coloring~\cite{taylor1990coloring, kessler1992coloring} to partition the
caches~\cite{lin2008coloring} between mutually distrusting parties. This comes
down setting up the page tables in such a way that no two mutually distrusting
software module are stored in physical pages that map to the same sets in any
cache memory.  However, if the system software is not trusted, the cache
partitioning scheme must be implemented in hardware.

The other interesting precondition is that the victim must access its memory in
a data-dependent fashion that allows the attacker to infer private information
from the observed memory access pattern. It becomes tempting to think that
cache timing attacks can be prevented by eliminating data-dependent memory
accesses from all the code handling sensitive data.

However, removing data-dependent memory accesses is difficult to accomplish in
practice because instruction fetches must also be taken into consideration.
\cite{kasper2009aes} gives an idea of the level of effort required for removing
data-dependent accesses from AES, which is a relatively simple data processing
algorithm. At the time of this writing, we are not aware of any approach that
scales to large pieces of software.

While the focus of this section is cache timing attacks, we would like to point
out that any shared resource can lead to information leakage. A worrying
example is hyper-threading (\S~\ref{sec:cpu_core}), where each CPU core is
represented as two logical processors, and the threads executing on these two
processors share execution units. An attacker that can run a process on a
logical processor sharing a core with a victim process can use
\texttt{RDTSCP}~\cite{petters1999making} to learn which execution units are in
use, and infer what instructions are executed by the victim process.

