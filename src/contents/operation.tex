\section{Enclave Operation}

SGX provides trusted computing by protecting the privacy and integrity of the
computation that is performed inside an enclave, and by providing remote
attestation identifying the software running inside an enclavee. This section
describes the processes used to set up an enclave and perform computation
inside it.

% TODO: find a place for the rest of this section
% Access Control Requirements: SDM S 38.3

Enclave code runs at ring 3 (\S~\ref{sec:rings}), just like application code,
so that it cannot interfere with the system software that manages resources.

% Interactions with VMX: SDM S 42.5, S 42.5.{1,2,3,4,5}

On most computers, the system software that manages SGX resources is an
operating system kernel. When VMX is in use, the hypervisor is responsible for
multiplexing SGX resources across the guest operating systems running in
virtual machines.

\subsection{The Enclave Mode}
\label{sec:enclave_mode}

% Internal CREGs: SDM S 41.1.4
% Access Control Requirements: SDM S 38.3

When executing the code inside an enclave, a logical processor is said to be
\textit{in enclave mode}. When a processor is outside enclave mode, software
cannot access any memory inside the PRM range. In enclave mode, the CPU allows
accesses to EPC pages that belong to the currently executing enclave.

Assuming no interrupts or faults, a logical processor transitions into enclave
mode by executes the EENTER instruction. EENTER can only be executed by
application software (running at ring 3), because SGX was designed to be used
by applications. EENTER can only be used to jump into pre-defined locations
inside an enclave, for the same reasons that system calls only jump at specific
locations in an OS kernel. When all the preconditions (described below) are
met, EENTER causes the logical processor to enter enclave mode, and keeps it at
ring 3.

SGX restricts enclaves to ring 3, preserving the role of system software as
the computer's resource manager. This makes it possible for an infrastructure
owner to allow user-supplied software to create and use enclaves, while having
the assurance that the OS kernel and hypervisor can still protect the
infrastructure from buggy or malicious software.


\subsubsection{Interrupt / Fault Handling}
\label{sec:aex}





\subsection{Evicting EPC Pages}
\label{sec:sgx_ewb}

% Eviction of Enclave Pages: SDM S 39.5.3
% Eviction of an SECS Page: SDM S 39.5.5
% Eviction of a Version Array Page: SDM S 39.5.6

% EPA: SDM S 41.3

\begin{figure}[hbt!]
  \centering
  \includegraphics[width=85mm]{figures/sgx_ewb.pdf}
  \caption{
    The data flow of the EWB instruction that evicts an EPC page. The page's
    content is encrypted in a non-EPC RAM page. A nonce is created and saved
    in an empty slot inside a VA page. The page's EPCM metadata and a MAC
    are saved in a separate area in non-EPC memory.
  }
  \label{fig:sgx_ewb}
\end{figure}

% Loading an Enclave Page: SDM S 39.5.4


