\section{Background}
\label{sec:background}

Arguing about the security of an application running on an mainstream computer
using the Intel architecture requires understanding the interactions between
all the parts of an x86 execution environment. This section provides an
overview of the features referenced by the rest of the paper. Unless specified
otherwise, the information in this section can be found in Intel's
\textit{Software Development Manual} \cite{intel2014sdm} (SDM).

Each of the sub-sections below explains how its information is relevant to
to SGX, but does not introduce any SGX concepts. Experienced readers can safely
skip this section and refer back if necessary.


\subsection{Software Privilege Levels}
\label{sec:rings}

% Outcome: enumerate all software actors and their trust relationships

In an Infrastructure-as-a-Service (IaaS) cloud environment, such as Amazon EC2,
commodity CPUs run software at four different privilege levels, shown in
Figure~\ref{fig:cpu_rings}.

\begin{figure}[hbtp]
  \centering
  \includegraphics[width=85mm]{figures/cpu_rings.pdf}
  \caption{
    The privilege levels in the x86 architecture, and the software that
    typically runs at each security level.
  }
  \label{fig:cpu_rings}
\end{figure}

Each privilege level is strictly more powerful than the ones below it, so a
piece of software can freely read and modify the code and data running at less
privileged levels. Therefore, a software module can be compromised by any piece
of software running at a higher privilege level. It follows that a software
module implicitly trusts all the software running at more privileged levels,
and a system's security analysis must take into account the software at all
privilege levels.

% System Management Mode: SDM S 34

\textit{System Management Mode} (SMM) is intended for use by the motherboard
manufacturers to implement features such as fan control and deep sleep, and/or
to emulate missing hardware. Therefore, the bootstrapping software
(\S~\ref{sec:booting}) in the computer's firmware is responsible for setting up
a continuous subset of DRAM as \textit{System Management RAM}~(SMRAM), and for
loading all the code that needs to run in SMM mode into SMRAM. The SMRAM
enjoys special hardware protections that prevent less privileged software from
accessing the SMM code.

IaaS cloud providers allow their customers to run their operating system of
choice in a virtualized environment. Hardware
virtualization~\cite{uhlig2005vmx}, called \textit{Virtual Machine Extensions}
(VMX) by Intel, adds support for a \textit{hypervisor}, also called a
\textit{Virtual Machine Monitor} (VMM) in the Intel documentation. The
hypervisor runs at a higher privilege level (VMX root mode) than the operating
system, and is responsible for allocating hardware resources across multiple
operating systems that share the same physical machine. The hypervisor uses the
CPU's hardware virtualization features to make each operating system believe it
is running in its own computer, called a \textit{virtual machine} (VM).
Hypervisor code generally runs at ring 0 in VMX root mode.

Hypervisors that run in VMX root mode and take advantage of hardware
virtualization generally have better performance and a smaller codebase than
hypervisors based on binary translation \cite{rosenblum2005virtualization}.

The systems research literature recommends breaking up an operating system into
a small \textit{kernel}, which runs at a high privilege level, known as the
\textit{kernel mode} or \textit{supervisor mode} and, in the Intel
architecture, as \textit{ring 0}. The kernel allocates the computer's resources
to the other system components, such as device drivers and services, which run
at lower privilege levels. However, for performance reasons\footnote{Calling a
procedure in a different ring is much slower than calling code at the same
privilege level.}, mainstream operating systems have large amounts of code
running at ring 0. Their \textit{monolithic kernels} include device drivers,
filesystem code, networking stacks, and video rendering functionality.

Application code, such as a Web server or a game client, runs at the lowest
privilege level, referred to as \textit{user mode} (\textit{ring 3} in the
Intel architecture). In IaaS cloud environments, the virtual machine images
provided by customers run in VMX non-root mode, so the kernel runs in VMX
non-root ring 0, and the application code runs in VMX non-root ring 3.

\subsection{Address Translation}
\label{sec:paging}

Modern OS kernels rely on address translation to manage the access of
applications to the memory address space (\S~\ref{sec:address_spaces}). This
mechanism lets the kernel multiplex DRAM among multiple application processes,
isolate the processes from each other, and restrict applications from accessing
memory-mapped devices directly. The latter two are necessary to prevent an
application's bugs from impacting other applications, or the OS kernel itself.
Hypervisors also use address translation to divide the DRAM among operating
systems that run concurrently, and to virtualize memory-mapped devices.

The software running inside a SGX enclave is subject to address
translation, and the implementation of SGX relies heavily on the translation
implementation in Intel processors. This section summarizes the Intel-specific
software-visible details needed to understand SGX, while \S~\ref{sec:tlbs}
covers some architectural specifics. \cite{jacob1998virtual} describes the
generic concepts and applications of address translation.

The Intel architecture specifies many address translation modes, used to
execute legacy software dating back to 1990 directly on the CPU, without the
overhead of software interpretation. Most modes can be ignored while analyzing
SGX, abd we only cover the translation modes used in modern 64-bit operating
systems and 64-bit cloud environments.

% Canonical Addressing: SDM vol1 S 3.3.7.1
% IA-32e Paging: SDM S 4.5

64-bit desktop operating systems use the addressing mode called IA-32e by
Intel's documentation, which translates 48-bit \textit{virtual addresses} into
\textit{physical addresses} of at most 52 bits\footnote{The size of a physical
  address is CPU-dependent, and is 40 bits for recent desktop CPUs and 44 bits
for recent high-end server CPUs.}.  Figure~\ref{fig:os_paging} illustrates the
address translation process. The bottom 12 bits of a virtual address are not
changed by the translation. The top 36 bits are grouped into four 9-bit indexes
used to navigate a data structure called \textit{the page tables}. Despite its
name, the data structure closely resembles a perfectly balanced 512-ary search
tree where nodes have fixed keys.  Each node is an array of 512 8-byte entries
that contain the physical addresses of the next-level children as well as some
flags. The address of the root node is stored in the CR3 register. The arrays
in the last-level nodes contain the physical addresses that are the result of
the address translation.

\begin{figure}[hbtp]
  \center{\includegraphics[width=85mm]{figures/os_paging.pdf}}
  \caption{
    IA-32e address translation takes in a 48-bit virtual address and outputs
    a 52-bit physical address.
  }
  \label{fig:os_paging}
\end{figure}

% VMX Support for Address Translation: SDM S 4.11

Hypervisors have access to another layer of address translation, named
\textit{extended page tables} (EPT), to multiplex the physical memory across
operating systems. When EPT are enabled, the process above is used to translate
from a virtual address into a \textit{guest-physical address}, effectively
giving each OS kernel the illusion that it controls the entire machine's RAM.
The translation from guest-physical addresses to actual physical addresses uses
the same process as above, except the physical address of the root node is
stored in the extended page table pointer (EPTP) field in the VM's control
structure (VMCS). Figure~\ref{fig:vmx_paging} illustrates the address
translation process in the presence of hardware virtualization.

\begin{figure}[hbtp]
  \center{\includegraphics[width=85mm]{figures/vmx_paging.pdf}}
  \caption{
    Address translation when hardware virtualization is enabled. The
    kernel-managed page tables contain guest-physical addresses, so each level
    in the kernel's page table requires a full walk of the hypervisor's
    extended page table (EPT).  A translation requires up to 20 memory accesses
    (the bold boxes), assuming the physical address of the kernel's PML4 is
    cached.
  }
  \label{fig:vmx_paging}
\end{figure}

Each entry in the page tables has some boolean flags, in addition to the
pointer to the next level. The following flags are particularly interesting for
our goals. The \textit{present} (P) flag is set to 0 to indicate unused parts
of the address space, which do not have physical memory associated with them,
or pages that have been evicted from RAM to a cheaper and slower storage
medium.  The \textit{accessed} (A) flag is set to 1 by the CPU whenever the
address translation machinery reads a page table entry, and the \textit{dirty}
(D) flag is set to 1 by the CPU when an entry is accessed by a memory write
operation. The A and D flags give the hypervisor and kernel insight into
application memory access patterns, providing the input for the algorithms that
select which pages get to be evicted from RAM.

% Page-Level Protection: SDM S 5.11, S 5.11.{1,2,3,4}

Page table entries have flags that provide access control, in addition to the
flags supporting page swapping. The interesting flags are the \textit{writable}
(W) flag, which can be set to 0 to prohibit\footnote{Writes to non-writable
pages result in \#GP exceptions (\S~\ref{sec:faults}).} memory writes to a
page, the \textit{disable execution} (XD) flag, which can be set to 1 to
prevent instruction fetches from a page, and the \textit{supervisor} (S) flag,
which can be set to 1 to prohibit accesses from application software running at
ring 3.

\subsection{Context Switching}
\label{sec:registers}

Application software targeting the 64-bit Intel architecture uses a variety of
CPU registers to interact with the processor's features. The values in these
registers make up an application's state, or context. Kernels multiplex a
CPU\footnote{Actually, each hardware thread is multiplexed among software
threads. See \S~\ref{sec:cpu_core}.}
among multiple software threads by \textit{context switching}, namely saving a
thread's context, and replacing it with another thread's previously saved
context. This section covers the context switching features used by SGX when a
logical processor starts or stops executing code that belongs to an enclave.

\begin{figure}[hbt]
  \center{\includegraphics[width=85mm]{figures/cpu_registers.pdf}}
  \caption{
    CPU registers in the 64-bit Intel architecture. RSP can be used as a
    general-purpose register (GPR), e.g., in pointer arithmetic, but it always
    points to the top of the program's stack. Segment registers are covered in
    \S~\ref{sec:segments}.
  }
  \label{fig:cpu_registers}
\end{figure}

Integers and memory addresses are stored in 16 \textit{general-purpose
registers} (GPRs). RAX, RBX, RCX, RDX, RSI, RDI, RSP, and RBP are extended
versions of the GPRs available to 32-bit programs, and R9-R16 are new
registers. RSP is reserved for pointing to the top of the \textit{stack}, which
is automatically read and modified by the CPU instructions for procedure calls,
(e.g., CALL and RET), and by explicit stack handling instructions such as PUSH
and POP.

All applications also use the RIP register, which contains the address of the
currently executing instruction, and the RFLAGS register, whose bits (e.g.,
the carry flag - CF) are individually used to store comparison results and
control various instructions.

Software might use other registers to interact with specific features, some of
which are shown in Table~\ref{fig:xsave_state}.

\begin{table}[hbt]
  \center{\begin{tabularx}{\columnwidth}{| l | X | l |}
  \hline
  \textbf{Feature} & \textbf{Registers} & \textbf{XCR0 bit}\\
  \hline
  FPU & FP0 - FP7, FSW, FTW & 0 \\
  \hline
  SSE & MM0 - MM7, XMM0 - XMM15, XMCSR & 1 \\
  \hline
  AVX & YMM0 - YMM15 & 2 \\
  \hline
  \end{tabularx}}
  \caption{Sample feature-specific Intel architecture registers.}
  \label{fig:xsave_state}
\end{table}

The Intel architecture provides a future-proof method for an OS kernel to save
the values of feature-specific registers used by an application. The XSAVE
instruction takes in a bitmap of features, and writes the registers used by
the features whose bits are set to 1 in a memory area that can be used by the
XRESTORE instruction to load the saved values back into feature-specific
registers.

Application software declares the features that it plans to use to the kernel,
so the kernel knows what XSAVE bitmap to use when context-switching. When
receiving the system call, the kernel sets the XCR0 register to the feature
bitmap declared by the application. The CPU generates a fault if application
software attempts to use features that are not enabled by XCR0, so applications
cannot modify feature-specific registers that the kernel wouldn't take into
account when context-switching. The kernel can use the CPUID instruction to
learn the size of the XSAVE memory area for a given feature bitmap, and compute
how much memory it needs to allocate for the context of each of the
application's threads.

\subsection{A Computer Map}

This section maps out a computer using the Intel architecture at three zoom
levels: the motherboard, the CPU, and the execution core, focusing on the
concepts needed to understand SGX and analyze its security properties. Most
details in here are documented in Intel's
\textit{Optimization Reference Manual} \cite{intel2014optimization}.


\subsubsection{The Motherboard}
\label{sec:motherboard}

A computer's components are connected by a printed circuit board called a
\textit{motherboard}, which consists of \textit{sockets} connected by
\textit{buses}. Sockets connect chip-carrying \textit{packages} to the board.
The Intel documentation uses the term ``package'' to specifically refer to a
CPU.

Figure~\ref{fig:motherboard} shows the most relevant chips, from an SGX
perspective. The CPU's package (described in \S~\ref{sec:cpu_die}) is the only
piece of trusted hardware in the SGX model. The \textit{Platform Controller
Hub} (PCH) houses (relatively) low-speed I/O controllers driving the slower
buses in the system, like SATA, used by storage devices, and USB, used by
input peripherals.  Motherboards also have a flash memory chip that hosts
firmware which implements the \textit{Unified Extensible Firmware Interface}
(UEFI). The firmware contains the boot code and the SMM handler.

\begin{figure}[hbt]
  \center{\includegraphics[width=85mm]{figures/motherboard.pdf}}
  \caption{
    The motherboard structures that are most relevant to SGX.
  }
  \label{fig:motherboard}
\end{figure}

The relevant buses are the \textit{Quick-Path Interconnect} (QPI)
\cite{intel2009qpi}, a network of point-to-point links that connect processors,
the \textit{double data rate} (DDR) bus that connects a CPU to DRAM, the
\textit {Direct Media Interface} (DMI) bus that connects a CPU to the PCH,
the \textit{Peripheral Component Interconnect Express} (PCIe) bus that connects
a CPU to peripherals such as a \textit{Network Interface Card} (NIC), and the
\textit {Serial Programming Interface} (SPI) used by the PCH to communicate
with the flash memory.

In high-end systems, the PCH also contains a service processor, called the
Intel \textit{Management Engine} (ME) \cite{ruan2014intelme}.  The ME runs
firmware stored in the same flash memory chip as the UEFI firmware. The
Management Engine is intended for remote system management and troubleshooting,
and has tremendous privileges. The ME is running even when the system is in
\textit{Soft Off} mode (ACPI G2/S5), when the CPU and DRAM are unpowered. Also,
the ME can access the network via a NIC without CPU support, can read and
modify DRAM via DMA transfers, and can override the CPU boot vector.

The PCIe bus is an extended, point-to-point version of the PCI standard, which
provides a method for any peripheral connected to the bus to perform
\textit{Direct Memory Access} (DMA), transferring data to and from DRAM without
involving an execution core and spending CPU cycles. The PCI standard includes
a configuration mechanism that assigns a range of DRAM to each peripheral, but
makes no provisions for preventing a rogue peripheral from accessing DRAM
outside the range that has been assigned to it.

The SGX trusted computing base includes the processor package, and excludes the
other hardware in the computer. It follows that SGX must be able to fend off
attacks from rogue devices, such as the PCIe NIC used to compromise Intel TXT
\cite{wojtczuk2011txt}, as well as passive or active bus-tapping attacks, such
as the memory bus tap used to hack the Xbox \cite{huang2003xbox} and the
memory glitching attack that subverted the PlayStation 3 hypervisor
\cite{hotz2010ps3}.

The flash memory stores the SMM handler and Intel ME firmware, which run with
high privileges. Furthermore, the contents of flash memory persists across
power cycles. This opens up the possibility for an attacker that gains SPI
access to deploy a persistent payload that runs at high privilege. To prevent
against these attacks, most of the firware is signed. For example, the ME
checks that its firmware was signed by a burned-in Intel public key. However,
both the computer firmware checks \cite{wojtczuk2010bios} and the ME firmware
checks \cite{tereshkin2009amt} have been subverted in the past.

The SGX threat model explicitly considers SMM to be untrusted. However, it does
not account for malicious code running on the Management Engine. Unfortunately,
the ME, PCH and DMI are Intel-proprietary and largely undocumented, so we
cannot assess the impact of an ME attack on software running inside an SGX
enclave.


\subsubsection{The Processor}
\label{sec:cpu_die}

An Intel processor's die, illustrated in Figure~\ref{fig:cpu_die}, is divided
into two broad areas: the \textit{core area} implements the instruction
execution pipeline typically associated with CPUs, while the \textit{uncore}
provides functions that were traditionally hosted on separate chips, but are
currently integrated on the CPU die to save power and improve latency.

\begin{figure}[hbt]
  \center{\includegraphics[width=85mm]{figures/cpu_die.pdf}}
  \caption{
    The major components in a modern CPU package. \S~\ref{sec:cpu_die} gives
    an uncore overview. \S~\ref{sec:cpu_core} describes execution cores.
    \S~\ref{sec:cache_coherence} takes a deeper look at the uncore.
  }
  \label{fig:cpu_die}
\end{figure}

% Ring Interconnect and Last Level Cache: Optimization S 2.2.5.3
% System Agent: Optimization S 2.2.6

At a conceptual level, the uncore of modern processors includes an
\textit{integrated memory controller} (iMC) that interfaces with the DDR bus,
an \textit{integrated I/O controller} (IIO) that implements PCIe bus lanes and
interacts with the DMI bus, and a growing number of integrated peripherals,
such as a \textit{Graphics Processing Unit} (GPU). The uncore structure is
described in some processor family datasheets \cite{intel2014datasheet,
intel2010datasheet}, and in the overview sections in Intel's uncore performance
monitoring documentation \cite{intel2014uncore, intel2012uncore,
intel2010uncore}.

The SGX design relies on the fact that the processor die includes the memory
and I/O controller, and thus can prevent any device from accessing protected
memory areas via \textit{Direct Memory Access} (DMA) transfers.
\S~\ref{sec:cache_coherence} takes a deeper look at the uncore organization and
at the mechanism used by the SGX implementation to protect sensitive memory.


\subsubsection{The Core}
\label{sec:cpu_core}

Virtually all modern Intel processors have core areas consisting of multiple
copies of the execution core circuitry, each of which is called a
\textit{core}.  At the time of this writing, desktop-class Intel CPUs have 4
cores, and server-class CPUs have as many as 18 cores.

Most Intel CPUs feature \textit{hyper-threading}, which means that a core
(shown in Figure~\ref{fig:cpu_core}) has two copies of the register files
backing the execution context described in \S~\ref{sec:registers}, and can
execute two separate streams of instructions simultaneously. Hyper-threading
increases the utilization of the shared fetch, decode and execution units, in
the presence of memory stalls.

\begin{figure}[hbt]
  \center{\includegraphics[width=85mm]{figures/cpu_core.pdf}}
  \caption{
    CPU core with two logical processors. Each logical processor has its own
    execution context and LAPIC (\S~\ref{sec:interrupts}). All the other core
    resources are shared.
  }
  \label{fig:cpu_core}
\end{figure}

A hyper-threaded core is exposed to system software as two \textit{logical
processors} (LPs), also named \textit{hardware threads} in the Intel
documentation.  The logical processor abstraction allows the code used to
distribute work across processors in a multi-processor system to function
without any change on multi-core hyper-threaded processors.

The high level of resource sharing introduced by hyper-threading introduces a
security vulnerability. Software running on one logical processor can use the
high-performance counter (\texttt{RDTSCP}, \S~\ref{sec:address_spaces})
\cite{petters1999making} to get information about the instructions and memory
access patterns of another piece of software that is executed on the other
logical processor in the same core.

\subsection{CPU Microcode}
\label{sec:microcode}

Intel's SGX patents disclose that all the SGX features, except for DRAM
encryption, were implemented as microcode extensions. The limitations of
microcode can explain seemingly arbitrary decisions in the SGX design, and a
thorough understanding is crucial to evaluating the feasibility of SGX
modification proposals. The first sub-section below presents the relevant facts
pertaining to microcode in Intel's optimization reference
\cite{intel2014optimization} and SDM. The following subsections summarize
information gleamed from Intel's patents and other researchers' findings.


\subsubsection{Official Information}
\label{sec:microcode_official}

% Intel® Microarchitecture Code Name Sandy Bridge Pipeline Overview:
%     Optimization S 2.2.1
% The Front End: Optimization S 2.2.2

The x86 architecture defines a \textit{complex instruction set} (CISC).
However, virtually all modern CPUs are architected following \textit{reduced
instruction set} (RISC) principles. This is accomplished by having the
instruction decode stage (see Figure~\ref{fig:cpu_core}) break down each x86
instruction into \textit{micro-ops} for every instruction. The other CPU stages
work exclusively with micro-ops.

% Legacy Decode Pipeline (Instruction Decode): Optimization S 2.2.2.1
% Instruction Decode: Optimization S 2.3.2.4
% Front End Overview: Optimization S 2.4.2

The majority of x86 instructions are handled by the hardware decoding path,
which can emit at most 4 micro-ops per instruction. Complex instructions use a
slower decoding path that reads micro-ops from a \textit{microcode store ROM}
(MSROM).

% Microcode Update Facilities: SDM S 9.11
% Responsibilities of the BIOS: SDM 9.11.8.1

Modern Intel processors implement a microcode update facility. The SDM
describes microcode updates from the perspective of an OS kernel and
hypervisor. Each core can be updated independently, and the updates must be
re-applied on each boot cycle. A core can be updated multiple times, but each
update must have a bigger version than the core's current version. The current
SDM version at the time of this writing indicates that a microcode update is
up to 16 kB in size.

The update facility increases the attractiveness of developing architectural
features as microcode extensions \cite{intel2008genetic, intel2012clusters}.
The SGX enclave measurements produced by the processor include the microcode
version, hinting that the SGX designers anticipated the need to use microcode
updates.

\subsubsection{Organization}


% Microcode handles exceptions:
%   US 5,987,600 - 2:39-57, 4:13-27, 4:39-53, 4:65-5:6, 8:42-58, 10:54-60,
%                  11:18-42, 12:11-17, 12:54-58, 15:46-48, 15:59-62
%   US 5,889,982 - 11:40-42, 11:44-46,

% Microcode handles memory exceptions (#PF):
%   US 5,987,600 - 14:26-49, 14:55-61, 14:66-15:3
%   US 5,680,565 - 11:29-37,
%   US 5,889,982 - 14:41-43, 15:47-51,

% Microcode handles DTLB and PMH exceptions:
%   US 5,564,111 - Abstract 15-21, 1:46-59, 3:25-45, 7:47-53, 9:33-51,
%                  10:45-54, 10:57-63

% Microcode performs assisted PMH walk
%   US 5,680,565 - Abstract 1-2 and last 3 lines, 4:9-19, 4:22-28, 12:24-25,
%                  13:42-44, 13:48-54, 13:59-64, 14:12-21, 14:23-29, 14:61-66,
%                  15:1-12, 15:16-39

% Microcode handles events (exceptions and assists):
%   US 5.889,982 - 9:23-25, 9:34-42, 15:7-11, 15:27-55, 16:34-38, 16:57-17:3

% Microcode handles traps:
%   US 5,987,600 - 15:16-18, 15:36-40

% Microcode handles interrupts:
%   US 5,987,600 - 16:2-5, 16:18-21

% Microcode implementation details:
%   US 5,987,600 - 5:39-49, 5:53-6:32, 5:35-39, 5:42-53, 11:53-60, 11:64-67,
%                  12:6-10, 12:41-45, 14:15-19
%   US 5,680,565 - 2:53-56
%   US 5,889,982 - 6:49-65, 7:8-12, 10:11-14, 13:16-20,

% PMH implementation (stuffed loads)
%   US 5,680,565 - 2:60-3:3, 3:25-28, 3:33-52, 3:56, 3:58-4:4, 11:17-21,
%                  11:45-48, 11:50-52, 12:30-34, 12:20-22, 12:40-43, 13:20-22,
%                  14:42-58, 15:54-57

% DTLB implementation
%   US 5,564,111 - 1:26-29, 1:36-38, 3:7-21, 3:58-60, 5:36-41, 5:48-57,
%                  6:51-52, 6:55-7:7, 7:16-18, 7:23-24, 8:3-8, 8:39-40,
%                  9:66-10:4, 10:16-23

% Micro-ops table
%   US 7,451,121 - 1:23-25, 1:34-35, 2:64-65
%   US 8,099,587 - 3:1

% Event ROM
%   US 5,889,982 - 16:57-63, 16:66-17:3

% Microcode compression
%   US 7,451,121 - Abstract 1 and 10
%   US 8,099,587 - Abstract 1-3 and 7-10, 8:36-49, 11:10-17

\cite{intel1999exceptions} discloses that when an event (hardware exception or
interrupt) occurs, an event code is dispatched to the MSROM, which provides the
micro-ops that handle the event.


\cite{intel1997pmh} describes the operation of the PMH (\S~\ref{sec:tlbs}), and
discloses that the PMH uses a microcode assist when it needs to set the dirty
or accessed bits in a page table (\S~\ref{sec:paging}).

\cite{intel1999events} discloses that the microcode has an event ROM which
contains pointers to event handlers in the micro-ops table. The event ROM is
indexed by 6-bit event codes. The first 16 events are hardware exceptions, and
the others are microcode assists. \cite{intel1999events} explicitly mentions
page faults (\#PF) as an example of an exception, and the PMH-issued microcode
assist used to set accessed and dirty bits in the page tables.

\cite{intel1996dtlb} confirms that microcode is used to handle faults and
assists generated by the TLB (\S~\ref{sec:tlbs}) and PMH.

\cite{intel2008genetic} and \cite{intel2012clusters} disclose that the MSROM
contains a micro-ops table. \cite{intel2012clusters} states that tables have on
the order of 20,000 entries, and a micro-op has about 70 bits. Microcode may be
partially compressed.



\cite{hawkes2012microcode} used fault injection and timing analysis to conclude
that each recent Intel microcode update is signed with a 2048-bit RSA key and
a (possibly non-standard) 256-bit hash algorithm. This implies that Intel
already has a microcode implementation of RSA-2048 signature checking, which
may explain why SGX uses RSA signatures in its enclave structures.

\cite{chen2014microcode} sets out to analyze the structure of microcode used in
all x86 processors, but is unable to obtain any details about Intel's
microcode. Fortunately, even though the microcode structure is undocumented,
the 4 micro-ops limitation can be used to guess intelligently whether an
architectural feature is implemented in microcode. For example, it is safe to
assume that \texttt{XSAVE} (\S~\ref{sec:registers}), which was takes over 200
micro-ops on recent CPUs \cite{fog2014microops}, is most likely performed in
microcode, whereas simple arithmetic and memory access is handled directly by
hardware.

While Intel publishes the latest microcode versions for its CPUs, the release
notes associated with the updates are not publicly available. This is
unfortunate, as the release notes could be used to confirm guesses that certain
features are implemented in microcode. However, some information can be
inferred by reading through the Errata section in Intel's Specification Updates
\cite{intel2010errata, intel2015errata, intel2015errata2}. The phrase ``it is
possible for BIOS\footnote{Basic Input/Output System (BIOS)
is the predecessor of UEFI-based firmware. Most Intel documentation, including
the SDM, still uses the term BIOS to refer to firmware.} to contain a
workaround for this erratum'' generally means that a microcode update was
issued. For example, Errata AH in \cite{intel2010errata} implies that string
instructions (\texttt{REP MOV}) are implemented in microcode, which was
confirmed by Intel \cite{abraham2006repmov}.

Errata AH43 and AH91 in \cite{intel2010errata}, and AAK73 in
\cite{intel2015errata} imply that address translation (\S~\ref{sec:paging}) is
at least partially implemented in microcode. Errata AAK53, AAK63, and AAK70,
AAK178 in \cite{intel2015errata}, and BT138, BT210,  in \cite{intel2015errata2}
imply that VM entries and exits (\S~\ref{sec:faults}) are implemented in
microcode.

\subsection{Cache Memories}
\label{sec:caching}

At the time of this writing, CPU cores can process data $\approx 200\times$
faster than DRAM can supply it. This gap is bridged by an hierarchy of cache
memories, which are orders of magnitude smaller and an order of magnitude
faster than DRAM. This section reviews the key concepts needed to understand
cache timing attacks (\S~\ref{sec:cache_timing}), which can be used to
learn about an application's memory access patterns. \cite{smith1982cache},
\cite{patterson2013architecture} and \cite{hennessy2012architecture} all
provide good backgrounds on low-level cache implementation concepts.

At a high level, caches exploit the high locality in the memory access patterns
of most applications to hide the main memory's (relatively) high latency. By
\textit{caching} (storing a copy of) the most recently accessed code and data,
these relatively small memories can be used to satisfy 90\%-99\% of an
application's memory accesses.

In an Intel processor, the \textit{first-level} (L1) cache consists of a
separate data cache (D-cache) and an instruction cache (I-cache). The
instruction fetch and decode stage is directly connected to the L1 I-cache, and
uses it to read the streams of instructions for the core's logical processors.
Micro-ops that read from or write to memory are executed by the memory unit
(MEM in Figure~\ref{fig:cpu_core}), which is connected to the L1 D-cache and
forwards memory accesses to it.

Figure \ref{fig:cache_lookup} illustrates the steps taken by a cache when it
receives a memory access. First, a \textit{cache lookup} uses the memory
address to determine if the corresponding data exists in the cache. A
\textit{cache hit} occurs when the address is found, and the cache can resolve
the memory access quickly. Conversely, if the address is not found, a
\textit{cache miss} occurs, and a \textit{cache fill} is required to resolve
the memory access. When doing a fill, the cache forwards the memory access to
the next level of the memory hierarchy and caches the response. Under most
circumstances, a cache fill also triggers a \textit{cache eviction}, in which
some data is removed from the cache to make room for the data coming from the
fill. If the data that is evicted has been modified since it was loaded in the
cache, it must be \textit{written back} to the next level of the memory
hierarchy.

\begin{figure}[hbt]
  \centering
  \includegraphics[width=80mm]{figures/cache_lookup.pdf}
  \caption{
    The steps taken by a cache memory to resolve an access to a memory address
    A. A normal memory access (to cacheable DRAM) always triggers a cache
    lookup. If the access misses the cache, a fill is required, and a
    write-back might be required.
  }
  \label{fig:cache_lookup}
\end{figure}

Table~\ref{fig:cache_timings} shows the key characteristics of the memory
hierarchy implemented by modern Intel CPUs. Each core has its own L1 and L2
cache (see Figure~\ref{fig:cpu_core}), while the L3 cache is in the CPU's
uncore (see Figure~\ref{fig:cpu_die}), and is shared by all the cores in the
package.

% Cache and Memory Subsystem: Optimization S 2.1.4
% Cache Hierarchy: Optimization S 2.2.5

\begin{table}[hbt]
  \centering
  \begin{tabular}{| l | r | r |}
  \hline
  \textbf{Memory} & \textbf{Size} & \textbf{Access Time}\\
  \hline
  Core Registers & 1~KB & no latency \\
  \hline
  L1 D-Cache & 32~KB & 4 cycles \\
  \hline
  L2 Cache & 256~KB & 10 cycles \\
  \hline
  L3 Cache & 8~MB & 40-75 cycles \\
  \hline
  DRAM & 16~GB & 60 ns \\
  \hline
  \end{tabular}
  \caption{
    Approximate sizes and access times for each level in the memory
    hierarchy of an Intel processor, from \cite{intel2010perfanalysis}. Memory
    sizes and access times differ by orders of magnitude across the different
    levels of the hierarchy. This table does not cover multi-processor systems.
  }
  \label{fig:cache_timings}
\end{table}

A cache timing attack that aims at the L2 cache would have to rely on the
system software to schedule a software thread on a logical processor in the
same core as the target software, whereas an attack on the L3 cache can be
performed using any logical processor on the same CPU. This implies that L3
cache attacks are feasible in an IaaS environment, whereas L2 cache attacks
become a possibility when running sensitive software on a user's desktop.

\subsection{Cache Organization}
\label{sec:cache_org}

In the Intel architecture, caches are completely implemented in hardware,
meaning that the software stack has no control over the eviction process.
However, software can gain some control over which data gets evicted by
understanding how the caches are organized, and by cleverly placing its data in
memory. This knowledge can be used to mount a cache timing attack, and it can
also be used by system software to protect the software it manages from cache
timing attacks.

The \textit{cache line} is the atomic unit of cache organization. A cache line
has \textit{data}, a copy of a continuous range of DRAM, and a \textit{tag},
identifying the memory address that the data comes from. Fills and evictions
are performed on entire lines.

The cache line size is the size of the data, and is always a power of two.
Assuming $n$-bit memory addresses and a cache line size of $2^{l}$ bytes, the
lowest $l$ bits of a memory address are an offset into a cache line, and the
highest $n - l$ bits determine the cache line that is used to store the data at
the memory location. All recent processors have 64-byte cache lines.

The L1 and L2 caches in recent processors are multi-way set-associative with
direct set indexing, as shown in Figure~\ref{fig:cpu_cache}. A $W$-way
set-associative cache has its memory divided into \textit{sets}, where each set
has $W$ lines. A memory location can be cached in any of the $w$ lines in a
specific set that is determined by the highest $n - l$ bits of the location's
memory address. Direct set indexing means that the $S$ sets in a cache are
numbered from $0$ to $S - 1$, and the memory location at address $A$ is cached
in the set numbered $A_{n - 1 \ldots n - l} \bmod S$.

In the common case where the number of sets in a cache is a power of two, so $S
= 2^{s}$, the lowest $l$ bits in an address make up the cache line offset, the
next $s$ bits are the set index. The highest $n - s - l$ bits in an address are
not used when selecting where a memory location will be cached.
Figure~\ref{fig:cpu_cache} shows the cache structure and lookup process.

\begin{figure}[hbt]
  \center{\includegraphics[width=85mm]{figures/cpu_cache.pdf}}
  \caption{
    Cache organization and lookup, for a $W$-way set-associative cache with
    $2^{l}$-byte lines and $S = 2^{s}$ sets. The cache works with $n$-bit
    memory addresses. The lowest $l$ address bits point to a specific byte in a
    cache line, the next $s$ bytes index the set, and the highest $n - s - l$
    bits are used to decide if the desired address is in one of the $W$ lines
    in the indexed set.
  }
  \label{fig:cpu_cache}
\end{figure}




\subsection{Cache Coherence}
\label{sec:cache_coherence}

The Intel architecture was designed to support application software that was
not written with caches in mind. One aspect of this is the Total Store Order
(TSO) \cite{owens2009tso} memory model, the guarantee that concurrently running
hardware threads see the same order of DRAM writes. Given that a memory
location might be copied in multiple caches on different cores, or even
different processors, providing the TSO guarantees requires a \textit{cache
coherence protocol} that keeps the cache line copies in sync. This section
covers some cache coherence implementation details that are necessary for
understanding SGX. \cite{hennessy2012architecture} provides a good introduction
to cache coherence principles.

The cache coherence mechanism is not visible to software, so it only briefly
mentioned in the SDM. Fortunately, Intel's optimization reference
\cite{intel2014optimization} and the datasheets referenced in
\S~\ref{sec:cpu_die} provide more information. Intel processors use variations
of the MESIF \cite{goodman2009mesif} protocol, which is implemented in the CPU
and in the protocol layer of the QPI bus.

The SDM and the CPUID instructions indicate that the L3 cache, also known as
the \textit{last-level cache} (LLC) is \textit{inclusive}, meaning that any
location cached by an L1 or L2 cache must also be cached in the LLC. This
design decision reduces complexity in many implementation aspects. We estimate
that the bulk of the cache coherence implementation is in the CPU's uncore,
because cache synchronization can be achieved without having to communicate to
the lower cache levels, which are inside execution cores.

Unfortunately, a cache timing attack can take advantage of the fact that the
LLC is inclusive and shared among CPU cores. This allows an attacker thread
to monitor a target thread that runs on a core in the same CPU die. The
attacker can evict lines in the target core's cache by filling up the L3 cache,
and then probe the L3 cache to find out when the target causes cache evictions.
The evicted lines disclose some bits in the target thread's memory accesses.

The QPI protocol defines \textit{cache agents}, which are connected to the
last-level cache in a processor, and \textit{home agents}, which are connected
to memory controllers. Cache agents make requests to home agents for cache line
data on cache misses, while home agents keep track of cache line ownership, and
obtain the cache line data from other cache line agents, or from the memory
controller. The QPI routing layer supports multiple agents per socket, and each
processor has its own caching agents, and at least one home agent.

The CPU uncore (see Figure~\ref{fig:cpu_die}) has a bidirectional ring
interconnect used for communication between execution cores and the other
uncore components. The ring has one CBox connected to each core. Each CBox also
connects an LLC slice to the ring, and serves as the QPI cache agent for that
L3 cache slice.


\cite{intel2014datasheet} states that the LLC coherence engine is implemented
in the

According to Intel's patents, the SGX memory protection relies on special
entries in the \textit{Source Address Decoder} (SAD) and \textit{Target Address
Decoder} (TAD).  A thorough understanding of the memory hierarchy is required
in order to understand this aspect of the SGX implementation.

UBox - uncore configuration controller; master for reading and writing
physically distributed registers across the uncore using the message
channel; receives interrupts from system and dispatches them to the
appropriate core; system lock master (e.g. QPI bus lock)

CBox - last-level cache (LLC) coherence engine; the interface between a core
and a slice of the LLC; acts as the QPI cache agent for that slice of LLC;
CBoxes are co-located with cores and connected in a ring interconnect

The physical memory space is split up between CBoxes. The ``complex'' caching
algorithm mentioned in the Intel SDM includes a ``hashing'' step that maps a
physical address to a CBox, and thus a slice of the LLC.

Each CBox contains a Source Address Decoder (SAD), and the configurations of
all SADs in a core are identical, replicated by the UBox. The SAD takes in a
memory address and access type, and outputs a transaction type (coherent,
non-coherent, IO) and a node ID.

Home agent - contains the Target Address Decoder (TAD) and interfaces with a
memory controller; a CPU may contain multiple home agents, if it has multiple
memory contollers; the TAD maps a memory address to a specific DRAM channel,
implements logical channel address mapping and interleaving;

\subsection{Caches and Address Translation}
\label{sec:tlbs}

Modern system software relies on address translation (\S~\ref{sec:paging}).
This means that all the memory accesses issued by a CPU core use virtual
addresses, which must undergo translation. Caches must know the physical
address for a memory access, to handle aliasing (multiple virtual addresses
pointing to the same physical address) correctly. However, address translation
requires up to 20 memory accesses (see Figure~\ref{fig:vmx_paging}), so it is
impractical to perform a full address translation for every cache access.
Instead, address translation results are cached in the \textit{translation
look-aside buffer} (TLB).

Table~\ref{fig:tlb_timings} shows the levels of the TLB hierarchy. Recent
processors have separate L1 TLBs for instructions and data, and a shared L2
TLB. Each core has its own TLBs (see Figure~\ref{fig:cpu_core}). When a virtual
address is not contained in a core's TLB, the \textit{Page Miss Handler} (PMH)
performs a \textit{page walk} (page table / EPT traversal) to translate the
virtual address, and the result is stored in the TLB.

\begin{table}[hbt]
  \centering
  \begin{tabular}{| l | r | r |}
  \hline
  \textbf{Memory} & \textbf{Entries} & \textbf{Access Time}\\
  \hline
  L1 I-TLB & 128 + 8 = 136 & 1 cycle \\
  \hline
  L1 D-TLB & 64 + 32 + 4 = 100 & 1 cycle \\
  \hline
  L2 TLB & 1536 + 8 = 1544 & 7 cycles \\
  \hline
  Page Tables & $2^{36} \approx 6 \cdot 10^{10} $ & 18 cycles - 200ms \\
  \hline
  \end{tabular}
  \caption{
    Approximate sizes and access times for each level in the TLB hierarchy,
    from \cite{7zip2014haswell}.
  }
  \label{fig:tlb_timings}
\end{table}

% Caching Translations in TLBs: SDM S 4.10.2.2
% Caches for Paging Structures: SDM S 4.10.3.1

In the Intel architecture, the PMH is implemented in hardware, so the TLB is
never directly exposed to software and its implementation details are not
documented.  The SDM does state that each TLB entry contains the physical
address associated with a virtual address, and the metadata needed to resolve a
memory access. For example, the processor needs to check the writable (W) flag
on every write, and issue a General Protection fault (\#GP) if the write
targets a read-only page.  Therefore, the TLB entry for each virtual address
caches the logical-and of all the relevant W flags in the page table structures
leading up to the page.

The TLB is transparent to application software. However, kernels and
hypervisors must make sure that the TLBs do not get out of sync with the page
tables and EPTs. When changing a page table or EPT, the system software must
use the INVLPG instruction to invalidate any TLB entries for the virtual
address whose translation changed. Some instructions \textit{flush the TLBs},
meaning that they invalidate all the TLB entries, as a side-effect.

% Large Page Size Considerations: SDM S 11.11.9

TLB entries also cache the desired caching behavior (\S~\ref{sec:memory_io})
for their pages. This requires system software to flush the corresponding TLB
entries when changing MTRRs or page table entries. In return, the processor
only needs to compute the desired caching behavior during a TLB miss, as
opposed to computing the caching behavior on every memory access.

% Propagation of Paging-Structure Changes to Multiple Processors: SDM S 4.10.5

The TLB is not covered by the cache coherence mechanism described in
\S~\ref{sec:cache_coherence}. Therefore, when modifying a page table or EPT on
a multi-core / multi-processor system, the system software is responsible for
performing a \textit{TLB shootdown}, which consists of stopping all the logical
processors that use the page table / EPT about to be changed, performing the
changes, executing TLB-invalidating instructions on the stopped logical
processors, and then resuming execution on the stopped logical processors.

Address translation constrains the L1 cache design. On Intel processors, the
set index in an L1 cache only uses the address bits that are not impacted by
address translation, so that the L1 set lookup can be done in parallel with the
TLB lookup. This is critical for achieving a low latency when both the L1 TLB
and the L1 cache are hit.

Given a page size $P = 2^{p}$ bytes, the requirement
above translates to $l + s \le p$. In the Intel architecture, $p = 12$, and all
recent processors have 64-byte cache lines ($l = 6$) and 64 sets ($s = 6$) in
the L1 caches, as shown in Figure~\ref{fig:caching_and_paging}.
The L2 and L3 caches are only accessed if the L1 misses, so the physical
address for the memory access is known at that time, and can be used for
indexing.

\begin{figure}[hbt]
  \centering
  \includegraphics[width=87mm]{figures/caching_and_paging.pdf}
  \caption{
    Virtual addresses from the perspective of cache lookup and address
    translation. The bits used for the L1 set index and line offset are not
    changed by address translation, so the page tables do not impact L1 cache
    placement. Page tables do impact L2 and L3 cache placement. Using large
    pages (2~MB or 1~GB) makes cache placement independent of page tables.
  }
  \label{fig:caching_and_paging}
\end{figure}


