\section{Background}
\label{sec:background}

This section attempts to present the architectural context around Intel's SGX,
because analyzing the security of SGX requires understanding the interactions
between all the parts of an enclave's execution environment. Unless specified
otherwise, the information here is summarized from Intel's
\textit{Software Development Manual} (SDM) \cite{intel2015sdm}.

Each of the sub-sections below explains how its information is relevant to
to SGX, but does not introduce any SGX concepts. Readers familiar with x86's
intricacies can safely skip this section and refer back when necessary.

We use the term \textit{Intel processor} and \textit{Intel CPU} to refer to the
server and desktop versions of Intel's Core line-up, as we focus on the SGX
design's most likely targets. This writing is not interested in Intel's other
processors, such as the embedded line of Atom CPUs, or the failed Itanium line.
Consequently, the terms \textit{Intel computers} and \textit{Intel systems}
refers to computer systems built around Intel's Core processors.

In this paper, the term \textit{Intel architecture} refers to the x86
architecture described in Intel's SDM. The x86 architecture is overly complex,
mostly due to the need to support execute legacy software dating back to 1990
directly on the CPU, without the overhead of software interpretation. In the
interest of space and mental sanity, we only cover the parts of the
architecture visible to modern 64-bit software.

The 64-bit version of the x86 archicture, covered in this section, was actually
invented by Advanced Micro Devices (AMD), and is also known as AMD64,
\texttt{x86\_64}, and x64. The term ``Intel architecture'' highlights our
interest in the architecture's implementation in Intel's chips, and our desire
to understand the mindsets of Intel SGX's designers.


\subsection{Overview}
\label{sec:background_overview}

A computer's main resources are processing power and storage, which is also
known as \textit{memory}. On Intel computers, processing power is offered by
logical processors (\S~\ref{sec:cpu_core}) inside CPUs, and memory is
implemented by DRAM chips (\S~\ref{sec:motherboard}). The software that manages
these resources is called \textit{system software}. An Intel computer typically
runs two kinds of system software, namely operating systems and hypervisors.

A typical Intel computer runs multiple application software instances, called
\textit{processes}. An \textit{operating system} (\S~\ref{sec:rings}), assigns
the computer's resources to the running processes. Server computers, especially
in cloud environments, may run multiple operating system instances at the same
time. This is accomplished by having a \textit{hypervisor}~(\S~\ref{sec:rings})
partition the computer's resources between the operating system instances
running on the computer.

System software uses virtualization techniques to isolate each piece of
software that it manages (process or operating system) from the rest of the
software running on the computer. This isolation is a key tool for keeping
software complexity at manageable levels, as it allows application and OS
developers to focus on their software, and ignore the interactions with other
software that may run on the computer.

A key component of virtualization is address translation (\S~\ref{sec:paging}),
which gives software the impression that it owns all the memory on the
computer. The other key component is the software privilege levels
(\S~\ref{sec:rings}) enforced by the CPU. Hardware privilege separation ensures
that buggy or malicious software cannot damage other software directly or
indirectly, by interfering with the system software managing it.

Processes express their computing power requirements by creating execution
\textit{threads}, which are assigned by the operating system to the computer's
logical processors. A thread contains an execution context
(\S~\ref{sec:registers}), which is the information necessary to perform a
computation, such as the address of the next instruction to be executed.

Operating systems give each process the illusion that it has an infinite amount
of logical processors at its disposal, and multiplex the available logical
processors between the threads created by each process.  Modern operating
systems implement \textit{preemptive multithreading}, where the logical
processors are rotated between all the threads on a system every few
milliseconds. Changing the thread assigned to a logical processor is
accomplished by an execution context switch (\S~\ref{sec:registers}).

Hypervisors expose a fixed number of virtual processors (vCPUs) to each
operating system, and also use context switching to multiplex the logical CPUs
on a computer between the vCPUs presented to the guest operating systems.

The execution core in a logical processor can execute instructions and consume
data at a much faster rate than DRAM can supply them. Many of the complexities
in modern computer architectures stem from the need to cover this speed gap.
Recent Intel CPUs rely on out-of-order execution (\S~\ref{sec:out_of_order})
and caching (\S~\ref{sec:caching}), both of which have security implications.

An Intel processor contains many levels of intermediate memories that are much
faster than DRAM, but also orders of magnitude smaller.  The fastest
intermediate memory is the logical processor's register file
(\S~\ref{sec:address_spaces}, \S~\ref{sec:registers}). The other intermediate
memories are called caches (\S~\ref{sec:caching}). The Intel architecture
requires application software to explicitly manage the register file, which
serves as a high-speed scratch space. At the same time, caches transparently
accelerate DRAM requests, and are mostly invisible to software.

Intel computers have multiple logical processors. As a consequence, they also
have multiple caches distributed across the CPU chip. On multi-socket systems,
the caches are distributed across multiple CPU chips. Therefore, Intel systems
use a cache coherence mechanism (\S~\ref{sec:cache_coherence}), ensuring that
all the caches have the same view of DRAM. Thanks to cache coherence,
programmers can build software that is unaware of caching, and still runs
correctly in the presence of distributed caches. However, cache coherence does
not cover the dedicated caches used by address translation (\S~\ref{sec:tlbs}),
and system software must take special measures to keep these caches consistent.

CPUs communicate with the outside world via I/O devices (also known as
peripherals), such as network interface cards and display adapters
(\S~\ref{sec:computer_map}). Software written for the Intel architecture
communicates with I/O devices via the I/O address space
(\S~\ref{sec:address_spaces}) and via the memory address space, which is also
used to access DRAM. System software must configure the CPU's caches
(\S~\ref{sec:memory_io}) to recognize the memory address ranges used by I/O
devices. Devices can notify the CPU of the occurrence of events by dispatching
interrupts (\S~\ref{sec:interrupts}), which cause a logical processor to stop
executing its current thread, and invoke a special handler in the system
software (\S~\ref{sec:faults}).

Intel systems have a highly complex computer initialization sequence
(\S~\ref{sec:booting}), due to the need to support a large variety of
peripherals, as well as a multitude of operating systems targeting different
versions of the architecture. The initialization sequence is a challenge to any
attempt to secure an Intel computer, and has facilitated many security
compromises (\S~\ref{sec:rings}).

Intel's engineers use the processor's microcode facility
(\S~\ref{sec:microcode}) to implement the more complicated aspects of the Intel
architecture, which greatly helps manage the hardware's complexity. The
microcode is completely invisible to software developers, and its design is
mostly undocumented. However, in order to evaluate the feasibility of any SGX
change proposals, one must be able to distinguish changes that can be achieved
with microcode from changes that require hardware changes.

\subsection{Software Privilege Levels}
\label{sec:rings}

% Outcome: enumerate all software actors and their trust relationships

In an Infrastructure-as-a-Service (IaaS) cloud environment, such as Amazon EC2,
commodity CPUs run software at four different privilege levels, shown in
Figure~\ref{fig:cpu_rings}.

\begin{figure}[hbtp]
  \centering
  \includegraphics[width=85mm]{figures/cpu_rings.pdf}
  \caption{
    The privilege levels in the x86 architecture, and the software that
    typically runs at each security level.
  }
  \label{fig:cpu_rings}
\end{figure}

Each privilege level is strictly more powerful than the ones below it, so a
piece of software can freely read and modify the code and data running at less
privileged levels. Therefore, a software module can be compromised by any piece
of software running at a higher privilege level. It follows that a software
module implicitly trusts all the software running at more privileged levels,
and a system's security analysis must take into account the software at all
privilege levels.

% System Management Mode: SDM S 34

\textit{System Management Mode} (SMM) is intended for use by the motherboard
manufacturers to implement features such as fan control and deep sleep, and/or
to emulate missing hardware. Therefore, the bootstrapping software
(\S~\ref{sec:booting}) in the computer's firmware is responsible for setting up
a continuous subset of DRAM as \textit{System Management RAM}~(SMRAM), and for
loading all the code that needs to run in SMM mode into SMRAM. The SMRAM
enjoys special hardware protections that prevent less privileged software from
accessing the SMM code.

IaaS cloud providers allow their customers to run their operating system of
choice in a virtualized environment. Hardware
virtualization~\cite{uhlig2005vmx}, called \textit{Virtual Machine Extensions}
(VMX) by Intel, adds support for a \textit{hypervisor}, also called a
\textit{Virtual Machine Monitor} (VMM) in the Intel documentation. The
hypervisor runs at a higher privilege level (VMX root mode) than the operating
system, and is responsible for allocating hardware resources across multiple
operating systems that share the same physical machine. The hypervisor uses the
CPU's hardware virtualization features to make each operating system believe it
is running in its own computer, called a \textit{virtual machine} (VM).
Hypervisor code generally runs at ring 0 in VMX root mode.

Hypervisors that run in VMX root mode and take advantage of hardware
virtualization generally have better performance and a smaller codebase than
hypervisors based on binary translation \cite{rosenblum2005virtualization}.

The systems research literature recommends breaking up an operating system into
a small \textit{kernel}, which runs at a high privilege level, known as the
\textit{kernel mode} or \textit{supervisor mode} and, in the Intel
architecture, as \textit{ring 0}. The kernel allocates the computer's resources
to the other system components, such as device drivers and services, which run
at lower privilege levels. However, for performance reasons\footnote{Calling a
procedure in a different ring is much slower than calling code at the same
privilege level.}, mainstream operating systems have large amounts of code
running at ring 0. Their \textit{monolithic kernels} include device drivers,
filesystem code, networking stacks, and video rendering functionality.

Application code, such as a Web server or a game client, runs at the lowest
privilege level, referred to as \textit{user mode} (\textit{ring 3} in the
Intel architecture). In IaaS cloud environments, the virtual machine images
provided by customers run in VMX non-root mode, so the kernel runs in VMX
non-root ring 0, and the application code runs in VMX non-root ring 3.

\subsection{Address Spaces}
\label{sec:address_spaces}

While performing computation, a commodity Intel CPU moves data between four
distinct physical address spaces, shown in Figure~\ref{fig:address_spaces}. The
address spaces overlap partially, in both purpose and contents, which can lead
to confusion. This section gives a high-level overview of the physical address
spaces defined by the Intel architecture, with an emphasis on their purpose and
the methods used to manage them.

\begin{figure}[hbtp]
  \center{\includegraphics[width=55mm]{figures/address_spaces.pdf}}
  \caption{
    The four physical address spaces used by an Intel CPU. The registers and
    MSRs are internal to the CPU, while the memory and I/O address spaces are
    used to communicate with DRAM and other devices via system buses.
  }
  \label{fig:address_spaces}
\end{figure}

The \textit{register} space consists of names that are used to access the CPU's
register file, which is the only memory that operates at the CPU's clock
frequency and can be used without any latency penalty. The register space is
defined by the CPU's architecture, and documented in the SDM.

Some registers, such as the \textit{Control Registers} (CRs) play specific
roles in configuring the CPU's operation. For example, CR3 plays a central role
in address translation (\S~\ref{sec:paging}). These registers can only be
accessed by system software. The rest of the registers make up an application's
\textit{execution context} (\S~\ref{sec:registers}), which is essentially a
high-speed scratch space. These registers can by accessed at all privilege
levels, and their allocation is managed by the software's compiler. Many CPU
instructions only operate on data in registers, and only place their results in
registers.

The \textit{memory} space, generally referred to as \textit{the address space}
\textit{the physical address space}, consists of $2^{36}$ (64 GB) - $2^{40}$
(1 TB) addresses. The memory space is primarily used to access
\textit{Dynamic Random-Access Memory} (DRAM), the computer's main memory, but
it is also used to communicate with \textit{memory-mapped devices} that read
memory requests off a system bus and write replies for the CPU. Some CPU
instructions can read their inputs from the memory space, or store the results
using the memory space.

A better-known example of memory mapping is that at computer startup, memory
addresses 0xFFFFF000 - 0xFFFFFFFF (the 64 KB of memory right below the 4 GB
mark) are mapped to a flash memory device that holds the code for booting the
computer.

This memory space is partitioned between devices and DRAM by the computer's
firmware during the boot stage. Sometimes, system software includes
motherboard-specific code that modifies the memory space partitioning. The OS
kernel relies on address translation, described in \S~\ref{sec:paging}, to
control the applications' access to the memory space. The hypervisor relies on
the same mechanism to control the guest OSes.

The \textit{input/output} (I/O) space consists of $2^{16}$ I/O addresses,
usually called \textit{ports}. The I/O ports are used exclusively to
communicate with devices. The CPU provides specific instructions for reading
from and writing to the I/O space. I/O ports are allocated to devices by formal
or de-facto standards. For example, ports 0xCF8 and 0xCFC are always used to
access the PCI express (\S~\ref{sec:motherboard}) configuration space.

The CPU implements a mechanism for system software to provide fine-grained I/O
access to applications. However, all modern kernels restrict application
software from accessing the I/O space directly, in order to limit the damage
potential of application bugs.

% Architectural MSRs: SDM S 35.1
% Time-Stamp Counter: SDM S 17.13

The \textit{Model-Specific Register} (MSR) space consists of $2^{32}$ MSRs,
which are used to configure the CPU's operation. The MSR space was initially
intended for the use of CPU model-specific firmware, but some MSRs have been
promoted to \textit{architectural MSR} status, making their semantics a part of
the Intel architecture. For example, architectural MSR 0x10 holds a
high-resolution monotonically increasing time-stamp counter.

The CPU provides instructions for reading from and writing to the MSR space.
The instructions can only be used by system software. Some MSRs are also
exposed by instructions accessible to applications. For example, applications
can read the time-stamp counter with the \texttt{RDTSC} and \texttt{RDTSCP},
which are very useful for benchmarking and optimizing software, but also for
mounting timing attacks.

\subsection{Address Translation}
\label{sec:paging}

Modern OS kernels rely on address translation to manage the access of
applications to the memory address space (\S~\ref{sec:address_spaces}). This
mechanism lets the kernel multiplex DRAM among multiple application processes,
isolate the processes from each other, and restrict applications from accessing
memory-mapped devices directly. The latter two are necessary to prevent an
application's bugs from impacting other applications, or the OS kernel itself.
Hypervisors also use address translation to divide the DRAM among operating
systems that run concurrently, and to virtualize memory-mapped devices.

The software running inside a SGX enclave is subject to address
translation, and the implementation of SGX relies heavily on the translation
implementation in Intel processors. This section summarizes the Intel-specific
software-visible details needed to understand SGX, while \S~\ref{sec:tlbs}
covers some architectural specifics. \cite{jacob1998virtual} describes the
generic concepts and applications of address translation.

The Intel architecture specifies many address translation modes, used to
execute legacy software dating back to 1990 directly on the CPU, without the
overhead of software interpretation. Most modes can be ignored while analyzing
SGX, abd we only cover the translation modes used in modern 64-bit operating
systems and 64-bit cloud environments.

% Canonical Addressing: SDM vol1 S 3.3.7.1
% IA-32e Paging: SDM S 4.5

64-bit desktop operating systems use the addressing mode called IA-32e by
Intel's documentation, which translates 48-bit \textit{virtual addresses} into
\textit{physical addresses} of at most 52 bits\footnote{The size of a physical
  address is CPU-dependent, and is 40 bits for recent desktop CPUs and 44 bits
for recent high-end server CPUs.}.  Figure~\ref{fig:os_paging} illustrates the
address translation process. The bottom 12 bits of a virtual address are not
changed by the translation. The top 36 bits are grouped into four 9-bit indexes
used to navigate a data structure called \textit{the page tables}. Despite its
name, the data structure closely resembles a perfectly balanced 512-ary search
tree where nodes have fixed keys.  Each node is an array of 512 8-byte entries
that contain the physical addresses of the next-level children as well as some
flags. The address of the root node is stored in the CR3 register. The arrays
in the last-level nodes contain the physical addresses that are the result of
the address translation.

\begin{figure}[hbtp]
  \center{\includegraphics[width=85mm]{figures/os_paging.pdf}}
  \caption{
    IA-32e address translation takes in a 48-bit virtual address and outputs
    a 52-bit physical address.
  }
  \label{fig:os_paging}
\end{figure}

% VMX Support for Address Translation: SDM S 4.11

Hypervisors have access to another layer of address translation, named
\textit{extended page tables} (EPT), to multiplex the physical memory across
operating systems. When EPT are enabled, the process above is used to translate
from a virtual address into a \textit{guest-physical address}, effectively
giving each OS kernel the illusion that it controls the entire machine's RAM.
The translation from guest-physical addresses to actual physical addresses uses
the same process as above, except the physical address of the root node is
stored in the extended page table pointer (EPTP) field in the VM's control
structure (VMCS). Figure~\ref{fig:vmx_paging} illustrates the address
translation process in the presence of hardware virtualization.

\begin{figure}[hbtp]
  \center{\includegraphics[width=85mm]{figures/vmx_paging.pdf}}
  \caption{
    Address translation when hardware virtualization is enabled. The
    kernel-managed page tables contain guest-physical addresses, so each level
    in the kernel's page table requires a full walk of the hypervisor's
    extended page table (EPT).  A translation requires up to 20 memory accesses
    (the bold boxes), assuming the physical address of the kernel's PML4 is
    cached.
  }
  \label{fig:vmx_paging}
\end{figure}

Each entry in the page tables has some boolean flags, in addition to the
pointer to the next level. The following flags are particularly interesting for
our goals. The \textit{present} (P) flag is set to 0 to indicate unused parts
of the address space, which do not have physical memory associated with them,
or pages that have been evicted from RAM to a cheaper and slower storage
medium.  The \textit{accessed} (A) flag is set to 1 by the CPU whenever the
address translation machinery reads a page table entry, and the \textit{dirty}
(D) flag is set to 1 by the CPU when an entry is accessed by a memory write
operation. The A and D flags give the hypervisor and kernel insight into
application memory access patterns, providing the input for the algorithms that
select which pages get to be evicted from RAM.

% Page-Level Protection: SDM S 5.11, S 5.11.{1,2,3,4}

Page table entries have flags that provide access control, in addition to the
flags supporting page swapping. The interesting flags are the \textit{writable}
(W) flag, which can be set to 0 to prohibit\footnote{Writes to non-writable
pages result in \#GP exceptions (\S~\ref{sec:faults}).} memory writes to a
page, the \textit{disable execution} (XD) flag, which can be set to 1 to
prevent instruction fetches from a page, and the \textit{supervisor} (S) flag,
which can be set to 1 to prohibit accesses from application software running at
ring 3.

\subsection{Context Switching}
\label{sec:registers}

Application software targeting the 64-bit Intel architecture uses a variety of
CPU registers to interact with the processor's features. The values in these
registers make up an application's state, or context. Kernels multiplex a
CPU\footnote{Actually, each hardware thread is multiplexed among software
threads. See \S~\ref{sec:cpu_core}.}
among multiple software threads by \textit{context switching}, namely saving a
thread's context, and replacing it with another thread's previously saved
context. This section covers the context switching features used by SGX when a
logical processor starts or stops executing code that belongs to an enclave.

\begin{figure}[hbt]
  \center{\includegraphics[width=85mm]{figures/cpu_registers.pdf}}
  \caption{
    CPU registers in the 64-bit Intel architecture. RSP can be used as a
    general-purpose register (GPR), e.g., in pointer arithmetic, but it always
    points to the top of the program's stack. Segment registers are covered in
    \S~\ref{sec:segments}.
  }
  \label{fig:cpu_registers}
\end{figure}

Integers and memory addresses are stored in 16 \textit{general-purpose
registers} (GPRs). RAX, RBX, RCX, RDX, RSI, RDI, RSP, and RBP are extended
versions of the GPRs available to 32-bit programs, and R9-R16 are new
registers. RSP is reserved for pointing to the top of the \textit{stack}, which
is automatically read and modified by the CPU instructions for procedure calls,
(e.g., CALL and RET), and by explicit stack handling instructions such as PUSH
and POP.

All applications also use the RIP register, which contains the address of the
currently executing instruction, and the RFLAGS register, whose bits (e.g.,
the carry flag - CF) are individually used to store comparison results and
control various instructions.

Software might use other registers to interact with specific features, some of
which are shown in Table~\ref{fig:xsave_state}.

\begin{table}[hbt]
  \center{\begin{tabularx}{\columnwidth}{| l | X | l |}
  \hline
  \textbf{Feature} & \textbf{Registers} & \textbf{XCR0 bit}\\
  \hline
  FPU & FP0 - FP7, FSW, FTW & 0 \\
  \hline
  SSE & MM0 - MM7, XMM0 - XMM15, XMCSR & 1 \\
  \hline
  AVX & YMM0 - YMM15 & 2 \\
  \hline
  \end{tabularx}}
  \caption{Sample feature-specific Intel architecture registers.}
  \label{fig:xsave_state}
\end{table}

The Intel architecture provides a future-proof method for an OS kernel to save
the values of feature-specific registers used by an application. The XSAVE
instruction takes in a bitmap of features, and writes the registers used by
the features whose bits are set to 1 in a memory area that can be used by the
XRESTORE instruction to load the saved values back into feature-specific
registers.

Application software declares the features that it plans to use to the kernel,
so the kernel knows what XSAVE bitmap to use when context-switching. When
receiving the system call, the kernel sets the XCR0 register to the feature
bitmap declared by the application. The CPU generates a fault if application
software attempts to use features that are not enabled by XCR0, so applications
cannot modify feature-specific registers that the kernel wouldn't take into
account when context-switching. The kernel can use the CPUID instruction to
learn the size of the XSAVE memory area for a given feature bitmap, and compute
how much memory it needs to allocate for the context of each of the
application's threads.

\subsection{Segment Registers}
\label{sec:segments}

The Intel 64-bit architecture gained widespread adoption thanks to its ability
to run software targeting the older 32-bit architecture side-by-side
with 64-bit software~\cite{cnet2005itanium}. This ability comes at the cost of
some warts. While most of these warts can be ignored while reasoning about the
security of 64-bit software, the segment registers and vestigial
segmentation model must be understood.

The semantics of the Intel architecture's instructions include the implicit use
of a few segments which are loaded into the processor's \textit{segment
registers} shown in Figure~\ref{fig:cpu_registers}. Code fetches use the
\textit{code segment}~(CS). Instructions that reference the stack implicitly
use the \textit{stack segment}~(SS). Memory references implicitly use the
\textit{data segment}~(DS) or the \textit{destination segment}~(ES). Via
segment override prefixes, instructions can be modified to use the unnamed
segments FS and GS for memory references.

Modern operating systems effectively disable segmentation by covering the
entire addressable space with one segment, which is loaded in CS, and one data
segment, which is loaded in SS, DS and ES. The FS and GS registers store
segments covering \textit{thread-local storage}~(TLS).

% Segment Selectors: SDM S 3.4.2
% Segment Registers: SDM S 3.4.3

Due to the Intel architecture's 16-bit origins, segment registers are exposed
as 16-bit values, called \textit{segment selectors}. The top 13 bits in a
selector are an index in a \textit{descriptor table}, and the bottom 2 bits are
the selector's ring number, which is also called requested privilege level
(RPL) in the Intel documentation. Also, modern system software only uses rings
0 and 3 (see \S~\ref{sec:rings}).

% Segment Loading Instructions in IA-32e Mode: SDM S 3.4.4
% Limit Checking in 64-bit Mode: SDM S 5.3.1
% Privilege Levels: SDM S 5.5

Each segment register has a hidden \textit{segment descriptor}, which consists
of a \textit{base address}, \textit{limit}, and type information, such as
whether the descriptor should be used for executable code or data.
Figure~\ref{fig:cpu_segment} shows the effect of loading a 16-bit selector into
a segment register. The selector's index is used to read a descriptor from the
descriptor table and copy it into the segment register's hidden descriptor.

\begin{figure}[hbt]
  \centering
  \includegraphics[width=85mm]{figures/cpu_segment.pdf}
  \caption{
    Loading a segment register. The 16-bit value loaded by software is a
    selector consisting of an index and a ring number. The index selects a GDT
    entry, which is loaded into the descriptor part of the segment register.
  }
  \label{fig:cpu_segment}
\end{figure}

In 64-bit mode, all segment limits are ignored. The base addresses in most
segment registers (CS, DS, ES, SS) are ignored. The base addresses in FS and GS
are used, in order to support thread-local storage.
Figure~\ref{fig:cpu_segmentation} outlines the address computation in this
case. The instruction's address, named \textit{logical address} in the Intel
documentation, is added to the base address in the segment register's
descriptor, yielding the virtual address, also named \textit{linear address}.
The virtual address is then translated (\S~\ref{sec:paging}) to a physical
address.

\begin{figure}[hbt]
  \centering
  \includegraphics[width=80mm]{figures/cpu_segmentation.pdf}
  \caption{
    Example address computation process for \texttt{MOV FS:[RDX], 0}.  The
    segment's base address is added to the address in RDX before address
    translation (\S~\ref{sec:paging}) takes place.
  }
  \label{fig:cpu_segmentation}
\end{figure}

Outside the special case of using FS or GS to reference thread-local storage,
the logical and virtual (linear) addresses match. Therefore, most of the time,
we can get away with completely ignoring segmentation. In these cases, we use
the term ``virtual address'' to refer to both the virtual and hte linear
address.

Even though CS is not used for segmentation, 64-bit system software needs to
load a valid selector into it. The CPU uses the ring number in the CS selector
to track the current privilege level, and uses one of the type bits to know
whether it's running 64-bit code, or 32-bit code in compatibility mode.

% Null Segment Selector Checking: SDM S 5.4.1, S 5.4.1.1

The DS and ES segment registers are completely ignored, and can have null
selectors loaded in them. The CPU loads a null selector in SS when switching
privilege levels, discussed in \S~\ref{sec:faults}.

% Segment Loading Instructions in IA-32e Mode: SDM S 3.4.4
% Segmentation in IA-32e Mode: SDM S 3.2.4

Modern kernels only use one descriptor table, the \textit{Global Descriptor
Table} (GDT), whose virtual address is stored in the GDTR register. Table~
\ref{fig:gdt_layout} shows a typical GDT layout that can be used by 64-bit
kernels to run both 32-bit and 64-bit applications.

\begin{table}[hbt]
  \centering
  \begin{tabular}{| l | l |}
  \hline
  \textbf{Descriptor} & \textbf{Selector}\\
  \hline
  Null (must be unused) & 0 \\
  \hline
  Kernel code & 0x08 (index 1, ring 0) \\
  \hline
  Kernel data & 0x10 (index 2, ring 0) \\
  \hline
  User code & 0x1B (index 3, ring 3) \\
  \hline
  User data & 0x1F (index 4, ring 3) \\
  \hline
  TSS & 0x20 (index 5, ring 0) \\
  \hline
  \end{tabular}
  \caption{
    A typical GDT layout in the 64-bit Intel Architecture.
  }
  \label{fig:gdt_layout}
\end{table}

% TSS Descriptor: SDM S 7.2.2
% TSS Descriptor in 64-bit mode: SDM S 7.2.3
% Task Register: SDM S 7.2.4
% Task Management in 64-bit Mode: SDM S 7.7

The last entry in Table~\ref{fig:gdt_layout} is a descriptor for the
\textit{Task State Segment} (TSS), which was designed to implement hardware
context switching, named \textit{task switching} in the Intel documentation.
The descriptor is stored in the \textit{Task Register} (TR), which behaves like
the other segment registers described above.

Task switching was removed from the 64-bit architecture, but the TR segment
register was preserved, and it points to a repurposed TSS data structure. The
64-bit TSS contains an \textit{I/O map}, which indicates what parts of the I/O
address space can be accessed directly from ring 3, and the
\textit{Interrupt Stack Table} (IST), which is used for privilege level
switching (\S~\ref{sec:faults}).

Modern operating systems do not allow application software any direct access to
the I/O address space, so the kernel sets up a single TSS that is loaded into
TR during early initialization, and used to represent all applications running
under the OS.

\subsection{Privilege Level Switching}
\label{sec:privilege_switches}

Applications software needs a method to invoke the kernel, because it cannot
directly perform privileged operations, such as network or disk I/O. At the
same time, ring 3 software cannot be offered the ability to jump arbitrarily
into kernel code, as that would compromise the kernel's ability to isolate
applications and enforce security invariants.\footnote{For example, when an
application wishes to write a file to the disk, the kernel must check if the
application's user has access to that file. If the ring 3 code could perform
an arbitrary jump in kernel space, it would be able to skip the access check.}
Therefore, the processor has designated methods for switching privilege levels,
which protect the integrity of the more privileged software.

This section describes the privilege switching mechanisms that impact the SGX
design, summarized in Figure~\ref{fig:cpu_ring_switch}. Also, understanding the
considerations behind privilege switching is useful when analyzing SGX, because
the process of calling code inside an enclave is similar to switching privilege
levels, as an enclave's code must be able to enforce its own security
invariants, just like an OS kernel.


\subsubsection{System Calls}
\label{sec:syscalls}

% Fast System Calls in 64-Bit Mode: SDM S 5.8.8

\begin{figure}[hbt]
  \center
  \includegraphics[width=85mm]{figures/cpu_ring_switch.pdf}
  \caption{
    Modern privilege switching methods in the 64-bit Intel architecture.
  }
  \label{fig:cpu_ring_switch}
\end{figure}

On modern processors, application software uses the \texttt{SYSCALL}
instruction to invoke ring 0 code, and the kernel uses \texttt{SYSRET} to
switch the privilege level back to ring 3. \texttt{SYSCALL} jumps into a
predefined kernel location, which is specified by writing to a pair of
architectural MSRs (\S~\ref{sec:address_spaces}).  MSRs can only be read or
written by ring 0 code, so application software cannot modify
\texttt{SYSCALL}'s MSRs and abuse the \texttt{SYSCALL} instruction to execute
arbitrary kernel code. The \texttt{SYSRET} instruction switches back to ring 3
and jumps to the address in RCX, which is set by the \texttt{SYSCALL}
instruction. The \texttt{SYSCALL} / \texttt{SYSRET} pair is optimized for speed
by avoiding memory accesses. The design can get away without referencing a
stack because kernel calls are not recursive.


\subsubsection{Faults}
\label{sec:faults}

% Interrupt and Exception Handling: SDM S 6.1, S 6.2
% Access Rights: SDM S 4.6
% Page-Fault Exceptions: SDM S 4.7

The processor also performs a switch from ring 3 to ring 0 when a \textit{
hardware exception} occurs while executing application code. Some exceptions
indicate bugs in the application, whereas other exceptions require kernel
action. A \textit{general protection fault} (\#GP) occurs when software
attempts to perform a disallowed action, such as setting the CR3 register from
ring 3. A \textit{page fault} (\#PF) occurs when address translation encounters
a page table entry whose P flag is 0, or attempting to use a page in a way
inconsistent with the access bits in its page table entry, for example
accessing a page whose S bit is set from ring 3.

% Interrupt Descriptor Table (IDT): SDM S 6.10

When a hardware exception occurs in application code, the CPU performs a ring
switch, and calls the corresponding \textit{exception handler}. For example,
the \#GP handler typically terminates the application's process, while the \#PF
handler reads the swapped out page back into RAM and resumes the application's
execution.

The exception handlers are a part of the OS kernel, and their locations are
specified in the first 32 entries of the Interrupt Descriptor Table (IDT),
whose structure is shown in Table~\ref{fig:idt_entry}. The IDT's physical
address is stored in the IDTR register, which can only be accessed by ring 0
code. Kernels protect the IDT memory using page tables, so that ring 3 software
cannot access it.

\begin{table}[hbt]
  \centering
  \begin{tabular}{| l | r |}
  \hline
  \textbf{Field} & \textbf{Bits} \\
  \hline
  Handler RIP & 64 \\
  \hline
  Handler CS & 16 \\
  \hline
  Interrupt Stack Table (IST) index & 3 \\
  \hline
  \end{tabular}
  \caption{
    The essential fields of an IDT entry in 64-bit mode. Each entry points to a
    hardware exception or interrupt handler.
  }
  \label{fig:idt_entry}
\end{table}

Each IDT entry has a 3-bit index pointing into the Interrupt Stack Table (IST),
which is an array of 8 stack pointers stored in the TSS described in
\S~\ref{sec:segments}.

% 64-Bit Mode Stack Frame: SDM S 6.14.2
% IRET in IA-32e Mode: SDM S 6.14.3
% Stack Switching in IA-32e Mode: SDM S 6.14.4
% Interrupt Stack Table: SDM S 6.14.5

When a hardware exception occurs, the execution state may be corrupted, and the
current stack cannot be relied on. Therefore, the CPU first uses the handler's
IDT entry to set up a known good stack. SS is loaded with a null descriptor,
and RSP is set to the IST value pointed by the IDT entry. After switching to a
reliable stack, the CPU pushes the snapshot in Table~\ref{fig:fault_stack} on
the stack, then loads the IDT entry's values into the CS and RIP registers,
which trigger the execution of the exception handler.

\begin{table}[hbt]
  \centering
  \begin{tabular}{| l | r |}
  \hline
  \textbf{Field} & \textbf{Bits} \\
  \hline
  Exception SS & 64 \\
  \hline
  Exception RSP & 64 \\
  \hline
  RFLAGS & 64 \\
  \hline
  Exception CS & 64 \\
  \hline
  Exception RIP & 64 \\
  \hline
  Exception code & 64 \\
  \hline
  \end{tabular}
  \caption{
    The snapshot pushed on the handler's stack when a hardware exception
    occurs. IRET restores registers from this snapshot.
  }
  \label{fig:fault_stack}
\end{table}

After the exception handler completes, it uses the \texttt{IRET} (interrupt
return) instruction to load the registers from the on-stack snapshot and switch
back to ring 3.

The Intel architecture gives the fault handler complete control over the parts
of the execution context not listed in Table~\ref{fig:fault_stack}. This
privilege is used by some handlers (e.g., \#GP) to perform context switches
(\S~\ref{sec:registers}) after a process is terminated due to a bug. However,
in the SGX threat model, system software is not trusted, and giving it access
to an enclave's execution context would expose potentially sensitive
information, and present an opportunity to compromise the enclave's integrity.
Therefore, SGX cannot use the current fault handling process, and must modify
it.


\subsubsection{VMX Privilege Level Switching}
\label{sec:vmx}

% Life Cycle of VMM Software: SDM S 23.4
% Virtual-Machine Control Structure: SDM S 23.5

Intel systems that take advantage of the hardware virtualization support to run
multiple operating systems at the same time use a hypervisor that manages the
VMs. The hypervisor creates a \textit{Virtual Machine Control Structure} (VMCS)
for each operating system instance that it wishes to run, and assigns a VM to a
logical processor by executing the \texttt{VMENTER} instruction.

% VM Exists: SDM S 27
% EP{T_Induced VM Exits: SDM S 28.2.3

When a logical processor encounters a fault that must be handled by the
hypervisor, the logical processor performs a VM exit. For example, if the
address translation process encounters an EPT entry with the P flag set to 0,
the CPU performs a VM exit, and the hypervisor has an opportunity to bring the
page into RAM.

The VMCS shows a great application of the encapsulation principle
\cite{liskov1974adt}, which is generally used in high-level software, to
computer architecture. The Intel architecture specifies that each VMCS resides
in DRAM and is 4~KB in size. However, the architecture does not specify the
VMCS format, and instead requires the hypervisor to interact with the VMCS via
CPU instructions such as \texttt{VMREAD} and \texttt{VMWRITE}.

This approach allows Intel to add VMX features that require VMCS format
changes, without the burden of having to maintain backwards compatibility.
This is no small feat, given that huge amounts of complexity in the Intel
architecture were introduced due to compatibility requirements.

\subsection{A Computer Map}

This section maps out a computer using the Intel architecture at three zoom
levels: the motherboard, the CPU, and the execution core, focusing on the
concepts needed to understand SGX and analyze its security properties. Most
details in here are documented in Intel's
\textit{Optimization Reference Manual} \cite{intel2014optimization}.


\subsubsection{The Motherboard}
\label{sec:motherboard}

A computer's components are connected by a printed circuit board called a
\textit{motherboard}, which consists of \textit{sockets} connected by
\textit{buses}. Sockets connect chip-carrying \textit{packages} to the board.
The Intel documentation uses the term ``package'' to specifically refer to a
CPU.

Figure~\ref{fig:motherboard} shows the most relevant chips, from an SGX
perspective. The CPU's package (described in \S~\ref{sec:cpu_die}) is the only
piece of trusted hardware in the SGX model. The \textit{Platform Controller
Hub} (PCH) houses (relatively) low-speed I/O controllers driving the slower
buses in the system, like SATA, used by storage devices, and USB, used by
input peripherals.  Motherboards also have a flash memory chip that hosts
firmware which implements the \textit{Unified Extensible Firmware Interface}
(UEFI). The firmware contains the boot code and the SMM handler.

\begin{figure}[hbt]
  \center{\includegraphics[width=85mm]{figures/motherboard.pdf}}
  \caption{
    The motherboard structures that are most relevant to SGX.
  }
  \label{fig:motherboard}
\end{figure}

The relevant buses are the \textit{Quick-Path Interconnect} (QPI)
\cite{intel2009qpi}, a network of point-to-point links that connect processors,
the \textit{double data rate} (DDR) bus that connects a CPU to DRAM, the
\textit {Direct Media Interface} (DMI) bus that connects a CPU to the PCH,
the \textit{Peripheral Component Interconnect Express} (PCIe) bus that connects
a CPU to peripherals such as a \textit{Network Interface Card} (NIC), and the
\textit {Serial Programming Interface} (SPI) used by the PCH to communicate
with the flash memory.

In high-end systems, the PCH also contains a service processor, called the
Intel \textit{Management Engine} (ME) \cite{ruan2014intelme}.  The ME runs
firmware stored in the same flash memory chip as the UEFI firmware. The
Management Engine is intended for remote system management and troubleshooting,
and has tremendous privileges. The ME is running even when the system is in
\textit{Soft Off} mode (ACPI G2/S5), when the CPU and DRAM are unpowered. Also,
the ME can access the network via a NIC without CPU support, can read and
modify DRAM via DMA transfers, and can override the CPU boot vector.

The PCIe bus is an extended, point-to-point version of the PCI standard, which
provides a method for any peripheral connected to the bus to perform
\textit{Direct Memory Access} (DMA), transferring data to and from DRAM without
involving an execution core and spending CPU cycles. The PCI standard includes
a configuration mechanism that assigns a range of DRAM to each peripheral, but
makes no provisions for preventing a rogue peripheral from accessing DRAM
outside the range that has been assigned to it.

The SGX trusted computing base includes the processor package, and excludes the
other hardware in the computer. It follows that SGX must be able to fend off
attacks from rogue devices, such as the PCIe NIC used to compromise Intel TXT
\cite{wojtczuk2011txt}, as well as passive or active bus-tapping attacks, such
as the memory bus tap used to hack the Xbox \cite{huang2003xbox} and the
memory glitching attack that subverted the PlayStation 3 hypervisor
\cite{hotz2010ps3}.

The flash memory stores the SMM handler and Intel ME firmware, which run with
high privileges. Furthermore, the contents of flash memory persists across
power cycles. This opens up the possibility for an attacker that gains SPI
access to deploy a persistent payload that runs at high privilege. To prevent
against these attacks, most of the firware is signed. For example, the ME
checks that its firmware was signed by a burned-in Intel public key. However,
both the computer firmware checks \cite{wojtczuk2010bios} and the ME firmware
checks \cite{tereshkin2009amt} have been subverted in the past.

The SGX threat model explicitly considers SMM to be untrusted. However, it does
not account for malicious code running on the Management Engine. Unfortunately,
the ME, PCH and DMI are Intel-proprietary and largely undocumented, so we
cannot assess the impact of an ME attack on software running inside an SGX
enclave.


\subsubsection{The Processor}
\label{sec:cpu_die}

An Intel processor's die, illustrated in Figure~\ref{fig:cpu_die}, is divided
into two broad areas: the \textit{core area} implements the instruction
execution pipeline typically associated with CPUs, while the \textit{uncore}
provides functions that were traditionally hosted on separate chips, but are
currently integrated on the CPU die to save power and improve latency.

\begin{figure}[hbt]
  \center{\includegraphics[width=85mm]{figures/cpu_die.pdf}}
  \caption{
    The major components in a modern CPU package. \S~\ref{sec:cpu_die} gives
    an uncore overview. \S~\ref{sec:cpu_core} describes execution cores.
    \S~\ref{sec:cache_coherence} takes a deeper look at the uncore.
  }
  \label{fig:cpu_die}
\end{figure}

% Ring Interconnect and Last Level Cache: Optimization S 2.2.5.3
% System Agent: Optimization S 2.2.6

At a conceptual level, the uncore of modern processors includes an
\textit{integrated memory controller} (iMC) that interfaces with the DDR bus,
an \textit{integrated I/O controller} (IIO) that implements PCIe bus lanes and
interacts with the DMI bus, and a growing number of integrated peripherals,
such as a \textit{Graphics Processing Unit} (GPU). The uncore structure is
described in some processor family datasheets \cite{intel2014datasheet,
intel2010datasheet}, and in the overview sections in Intel's uncore performance
monitoring documentation \cite{intel2014uncore, intel2012uncore,
intel2010uncore}.

The SGX design relies on the fact that the processor die includes the memory
and I/O controller, and thus can prevent any device from accessing protected
memory areas via \textit{Direct Memory Access} (DMA) transfers.
\S~\ref{sec:cache_coherence} takes a deeper look at the uncore organization and
at the mechanism used by the SGX implementation to protect sensitive memory.


\subsubsection{The Core}
\label{sec:cpu_core}

Virtually all modern Intel processors have core areas consisting of multiple
copies of the execution core circuitry, each of which is called a
\textit{core}.  At the time of this writing, desktop-class Intel CPUs have 4
cores, and server-class CPUs have as many as 18 cores.

Most Intel CPUs feature \textit{hyper-threading}, which means that a core
(shown in Figure~\ref{fig:cpu_core}) has two copies of the register files
backing the execution context described in \S~\ref{sec:registers}, and can
execute two separate streams of instructions simultaneously. Hyper-threading
increases the utilization of the shared fetch, decode and execution units, in
the presence of memory stalls.

\begin{figure}[hbt]
  \center{\includegraphics[width=85mm]{figures/cpu_core.pdf}}
  \caption{
    CPU core with two logical processors. Each logical processor has its own
    execution context and LAPIC (\S~\ref{sec:interrupts}). All the other core
    resources are shared.
  }
  \label{fig:cpu_core}
\end{figure}

A hyper-threaded core is exposed to system software as two \textit{logical
processors} (LPs), also named \textit{hardware threads} in the Intel
documentation.  The logical processor abstraction allows the code used to
distribute work across processors in a multi-processor system to function
without any change on multi-core hyper-threaded processors.

The high level of resource sharing introduced by hyper-threading introduces a
security vulnerability. Software running on one logical processor can use the
high-performance counter (\texttt{RDTSCP}, \S~\ref{sec:address_spaces})
\cite{petters1999making} to get information about the instructions and memory
access patterns of another piece of software that is executed on the other
logical processor in the same core.

\subsection{Cache Memories}
\label{sec:caching}

At the time of this writing, CPU cores can process data $\approx 200\times$
faster than DRAM can supply it. This gap is bridged by an hierarchy of cache
memories, which are orders of magnitude smaller and an order of magnitude
faster than DRAM. This section reviews the key concepts needed to understand
cache timing attacks (\S~\ref{sec:cache_timing}), which can be used to
learn about an application's memory access patterns. \cite{smith1982cache},
\cite{patterson2013architecture} and \cite{hennessy2012architecture} all
provide good backgrounds on low-level cache implementation concepts.

At a high level, caches exploit the high locality in the memory access patterns
of most applications to hide the main memory's (relatively) high latency. By
\textit{caching} (storing a copy of) the most recently accessed code and data,
these relatively small memories can be used to satisfy 90\%-99\% of an
application's memory accesses.

In an Intel processor, the \textit{first-level} (L1) cache consists of a
separate data cache (D-cache) and an instruction cache (I-cache). The
instruction fetch and decode stage is directly connected to the L1 I-cache, and
uses it to read the streams of instructions for the core's logical processors.
Micro-ops that read from or write to memory are executed by the memory unit
(MEM in Figure~\ref{fig:cpu_core}), which is connected to the L1 D-cache and
forwards memory accesses to it.

Figure \ref{fig:cache_lookup} illustrates the steps taken by a cache when it
receives a memory access. First, a \textit{cache lookup} uses the memory
address to determine if the corresponding data exists in the cache. A
\textit{cache hit} occurs when the address is found, and the cache can resolve
the memory access quickly. Conversely, if the address is not found, a
\textit{cache miss} occurs, and a \textit{cache fill} is required to resolve
the memory access. When doing a fill, the cache forwards the memory access to
the next level of the memory hierarchy and caches the response. Under most
circumstances, a cache fill also triggers a \textit{cache eviction}, in which
some data is removed from the cache to make room for the data coming from the
fill. If the data that is evicted has been modified since it was loaded in the
cache, it must be \textit{written back} to the next level of the memory
hierarchy.

\begin{figure}[hbt]
  \centering
  \includegraphics[width=80mm]{figures/cache_lookup.pdf}
  \caption{
    The steps taken by a cache memory to resolve an access to a memory address
    A. A normal memory access (to cacheable DRAM) always triggers a cache
    lookup. If the access misses the cache, a fill is required, and a
    write-back might be required.
  }
  \label{fig:cache_lookup}
\end{figure}

Table~\ref{fig:cache_timings} shows the key characteristics of the memory
hierarchy implemented by modern Intel CPUs. Each core has its own L1 and L2
cache (see Figure~\ref{fig:cpu_core}), while the L3 cache is in the CPU's
uncore (see Figure~\ref{fig:cpu_die}), and is shared by all the cores in the
package.

% Cache and Memory Subsystem: Optimization S 2.1.4
% Cache Hierarchy: Optimization S 2.2.5

\begin{table}[hbt]
  \centering
  \begin{tabular}{| l | r | r |}
  \hline
  \textbf{Memory} & \textbf{Size} & \textbf{Access Time}\\
  \hline
  Core Registers & 1~KB & no latency \\
  \hline
  L1 D-Cache & 32~KB & 4 cycles \\
  \hline
  L2 Cache & 256~KB & 10 cycles \\
  \hline
  L3 Cache & 8~MB & 40-75 cycles \\
  \hline
  DRAM & 16~GB & 60 ns \\
  \hline
  \end{tabular}
  \caption{
    Approximate sizes and access times for each level in the memory
    hierarchy of an Intel processor, from \cite{intel2010perfanalysis}. Memory
    sizes and access times differ by orders of magnitude across the different
    levels of the hierarchy. This table does not cover multi-processor systems.
  }
  \label{fig:cache_timings}
\end{table}

A cache timing attack that aims at the L2 cache would have to rely on the
system software to schedule a software thread on a logical processor in the
same core as the target software, whereas an attack on the L3 cache can be
performed using any logical processor on the same CPU. This implies that L3
cache attacks are feasible in an IaaS environment, whereas L2 cache attacks
become a possibility when running sensitive software on a user's desktop.

\subsection{Cache Organization}
\label{sec:cache_org}

In the Intel architecture, caches are completely implemented in hardware,
meaning that the software stack has no control over the eviction process.
However, software can gain some control over which data gets evicted by
understanding how the caches are organized, and by cleverly placing its data in
memory. This knowledge can be used to mount a cache timing attack, and it can
also be used by system software to protect the software it manages from cache
timing attacks.

The \textit{cache line} is the atomic unit of cache organization. A cache line
has \textit{data}, a copy of a continuous range of DRAM, and a \textit{tag},
identifying the memory address that the data comes from. Fills and evictions
are performed on entire lines.

The cache line size is the size of the data, and is always a power of two.
Assuming $n$-bit memory addresses and a cache line size of $2^{l}$ bytes, the
lowest $l$ bits of a memory address are an offset into a cache line, and the
highest $n - l$ bits determine the cache line that is used to store the data at
the memory location. All recent processors have 64-byte cache lines.

The L1 and L2 caches in recent processors are multi-way set-associative with
direct set indexing, as shown in Figure~\ref{fig:cpu_cache}. A $W$-way
set-associative cache has its memory divided into \textit{sets}, where each set
has $W$ lines. A memory location can be cached in any of the $w$ lines in a
specific set that is determined by the highest $n - l$ bits of the location's
memory address. Direct set indexing means that the $S$ sets in a cache are
numbered from $0$ to $S - 1$, and the memory location at address $A$ is cached
in the set numbered $A_{n - 1 \ldots n - l} \bmod S$.

In the common case where the number of sets in a cache is a power of two, so $S
= 2^{s}$, the lowest $l$ bits in an address make up the cache line offset, the
next $s$ bits are the set index. The highest $n - s - l$ bits in an address are
not used when selecting where a memory location will be cached.
Figure~\ref{fig:cpu_cache} shows the cache structure and lookup process.

\begin{figure}[hbt]
  \center{\includegraphics[width=85mm]{figures/cpu_cache.pdf}}
  \caption{
    Cache organization and lookup, for a $W$-way set-associative cache with
    $2^{l}$-byte lines and $S = 2^{s}$ sets. The cache works with $n$-bit
    memory addresses. The lowest $l$ address bits point to a specific byte in a
    cache line, the next $s$ bytes index the set, and the highest $n - s - l$
    bits are used to decide if the desired address is in one of the $W$ lines
    in the indexed set.
  }
  \label{fig:cpu_cache}
\end{figure}

\subsection{Cache Coherence}
\label{sec:cache_coherence}

The Intel architecture was designed to support application software that was
not written with caches in mind. One aspect of this support is the
\textit{Total Store Order} (TSO) \cite{owens2009tso} memory model, which
promises that all the hardware threads in a computer see the same order of DRAM
writes. The same memory location might be simultaneously cached by different
cores' caches, or even by caches on separate chips, so providing the TSO
guarantees requires a \textit{cache coherence protocol} that keeps the cached
copies in sync. This section covers some cache coherence implementation details
that are necessary for understanding SGX. \cite{hennessy2012architecture}
provides a good introduction to cache coherence principles.

The cache coherence mechanism is not visible to software, so it is only briefly
mentioned in the SDM. Fortunately, Intel's optimization reference
\cite{intel2014optimization} and the datasheets referenced in
\S~\ref{sec:cpu_die} provide more information. Intel processors use variations
of the MESIF \cite{goodman2009mesif} protocol, which is implemented in the CPU
and in the protocol layer of the QPI bus.

The SDM and the \texttt{CPUID} instruction output indicate that the L3 cache,
also known as the \textit{last-level cache} (LLC) is \textit{inclusive},
meaning that any location cached by an L1 or L2 cache must also be cached in
the LLC. This design decision reduces complexity in many implementation
aspects. We estimate that the bulk of the cache coherence implementation is in
the CPU's uncore, because cache synchronization can be achieved without having
to communicate to the lower cache levels, which are inside execution cores.

Unfortunately, a cache timing attack can take advantage of the fact that the
LLC is inclusive and shared among CPU cores. This allows an attacker thread
to monitor a victim thread that runs on a core in the same CPU die. The
attacker can evict lines in the target core's cache by filling up the L3 cache,
and then probe the L3 cache to find out when the target causes cache evictions.
The evicted lines disclose some of the bits in the memory addresses accessed by
the victim.

The QPI protocol defines \textit{cache agents}, which are connected to the
last-level cache in a processor, and \textit{home agents}, which are connected
to memory controllers. Cache agents make requests to home agents for cache line
data on cache misses, while home agents keep track of cache line ownership, and
obtain the cache line data from other cache line agents, or from the memory
controller. The QPI routing layer supports multiple agents per socket, and each
processor has its own caching agents, and at least one home agent.


% Ring Interconnect and Last Level Cache: Optimization S 2.5.5.3

Figure \ref{fig:cpu_uncore} shows that the CPU uncore has a bidirectional ring
interconnect, which is used for communication between execution cores and the
other uncore components. The execution cores are connected to the ring by
\textit{CBoxes}, which route their LLC accesses. The routing is static, as the
LLC is divided into same-size slices (common slice sizes are 1.5Mb and 2.5Mb),
and an undocumented hashing scheme maps each possible physical address to
exactly one LLC slice. Intel's documentation states that the hashing scheme was
designed to avoid having a slice become a hotspot. The hashing scheme is the
reason why the L3 cache is documented as having a ``complex'' indexing scheme,
as opposed to the direct indexing used in the L1 and L2 caches.

\begin{figure}[hbt]
  \center{\includegraphics[width=90mm]{figures/cpu_uncore.pdf}}
  \caption{
    The stops on the ring interconnect used for inter-core and core-uncore
    communication.
  }
  \label{fig:cpu_uncore}
\end{figure}


The number of LLC slices matches the number of cores in the CPU, and each LLC
slice shares a CBox with a core. The CBoxes implement the cache coherence
engine, so each CBox acts as the QPI cache agent for its LLC slice. CBoxes
use a \textit{Source Address Decoder} (SAD) to route DRAM requests to the
appropriate home agents. Conceptually, the SAD takes in a memory address and
access type, and outputs a transaction type (coherent, non-coherent, IO) and a
node ID. Each CBox contains a SAD replica, and the configurations of all SADs
in a package are identical.

The SAD configurations are kept in sync by the \textit{UBox}, which is the
uncore configuration controller, and connects the \textit{System agent} to the
ring. The UBox is responsible for reading and writing physically distributed
registers across the uncore. The UBox also receives interrupts from system and
dispatches them to the appropriate core.

On recent Intel processors, the uncore also contains at least one memory
controller. Each integrated memory controller (iMC or MBox in Intel's
documentation) is connected to the ring by a \textit{home agent} (HA or
\textit{BBox} in Intel's datasheets). Each home agent contains a
\textit{Target Address Decoder} (TAD), which maps each physical DRAM address to
a specific DRAM channel.

The integration of the memory controller on the CPU brings the ability to
filter DMA. Accesses from a peripheral connected to the PCIe bus will be
handled by the integrated I/O controller (IIO) and placed on the ring
interconnect via the Ubox, then reach the iMC. Therefore, on modern systems,
DMA goes through both the SAD and TAD. Intel TXT takes advantage of this to set
up a protected DRAM range. For example, \cite{intel2015datasheet} documents an
``Intel TXT DMA Protected Range'' IIO configuration register. According to
Intel's patents, the SGX implementation uses the same mechanism to protect
enclave memory from outside accesses.

Very recent processors also include the Generic Debug eXternal Connection
(GDXC) \cite{yuffe2011sandybridge, intel2011gdxc}, which collects and filters
ring traffic, and reports it to an external debugger. While GDXC is very useful
for debugging the CPU as well as the systems embedding it, it may compromise
the security of SGX enclaves. Due to the insufficient documentation on this
topic, this paper ignores the possibility of GDXC-based attacks.

\subsection{Out-of-Order and Speculative Execution}
\label{sec:out_of_order}

CPU cores can execute instructions orders of magnitude faster than DRAM can
read data. CPU architects attempt to bridge this gap by using hyper-threading
(\S~\ref{sec:cpu_die}), caching (\S~\ref{sec:caching}), and out-of-order and
speculative execution. Out-of-order execution can safely be ignored when
developing a general understanding of SGX, but shows up behind the decision
decisions in SGX's implementation. Out-of-order and speculative execution can
also introduce noise in the data obtained from cache timing attacks. This
section provides an overview of out-of-order and speculative execution that is
sufficient for reasoning about SGX and cache timing attacks.
\cite{patterson2013architecture} and \cite{hennessy2012architecture} cover the
concepts in great depth, while Intel's optimization manual
\cite{intel2014optimization} provides details specific to Intel CPUs.

% The Haswell Microarchitecture: Optimization S 2.1

Figure~\ref{fig:cpu_out_of_order} provides a more detailed view of the CPU core
components involved in out-of-order execution, and omits some less relevant
details from Figure~\ref{fig:cpu_core}.

\begin{figure}[hbt]
  \centering
  \includegraphics[width=87mm]{figures/cpu_out_of_order.pdf}
  \caption{
    The structures in a CPU core that are relevant to out-of-order and
    speculative execution. Instructions are decoded into micro-ops, which are
    scheduled on one of the execution unit's ports. The branch predictor
    enables speculative execution when a branch is encountered.
  }
  \label{fig:cpu_out_of_order}
\end{figure}

% Intel Microarchitecture Code Name Sandy Bridge Pipeline Overview:
%     Optimization S 2.2.1
% The Front End: Optimization S 2.2.2

The Intel architecture defines a \textit{complex instruction set} (CISC).
However, virtually all modern CPUs are architected following \textit{reduced
instruction set} (RISC) principles. This is accomplished by having the
instruction decode stages break down each instruction into \textit{micro-ops},
which resemble RISC instructions. The other stages of the execution pipeline
work exclusively with micro-ops.


\subsubsection{Out-of-Order Execution}

% Out of Order Engine: Optimization S 2.2.23
% The Execution Core: S 2.2.4

Different types of instructions require different logic circuits, called
\textit{functional units}. For example, the arithmetic logic unit (ALU), which
performs arithmetic operations, is completely different from the load store
unit, which peforms memory operations. Different circuits can be used at the
same time, so each CPU core can execute multiple micro-ops in parallel.

The core's out-of-order engine receives decoded micro-ops, identifies the
micro-ops that can execute in parallel, assigns them to functional units, and
combines the outputs of the units so that the results are equivalent to having
the micro-ops executed sequentially in the order in which they come from the
decode stages.

For example, consider the sequence of pseudo micro-ops\footnote{The set of
micro-ops used by Intel CPUs is not publicly documented. The fictional examples
in this section suffice for illustration purposes.} in
Table~\ref{fig:out_of_order_micro_ops} below. The \texttt{OR} uses the result
of the \texttt{LOAD}, but the \texttt{ADD} does not. Therefore, a good
scheduler can have the load store unit execute the \texttt{LOAD} and the ALU
execute the \texttt{ADD}, all in the same clock cycle.

\begin{table}[hbt]
  \centering
  \begin{tabular}{| l | l | l |}
  \hline
  \textbf{\#} & \textbf{Micro-op} & \textbf{Meaning}\\
  \hline
  1 & \texttt{LOAD RAX, RSI} & RAX $\leftarrow$ DRAM[RSI]\\
  \hline
  2 & \texttt{OR RDI, RDI, RAX} & RDI $\leftarrow$ RDI $\lor$ RAX\\
  \hline
  3 & \texttt{ADD RSI, RSI, RCX} & RSI $\leftarrow$ RSI + RCX\\
  \hline
  4 & \texttt{SUB RBX, RSI, RDX} & RBX $\leftarrow$ RSI - RDX\\
  \hline
  \end{tabular}
  \caption{
    Pseudo micro-ops for the out-of-order execution example.
  }
  \label{fig:out_of_order_micro_ops}
\end{table}

% Renamer: Optimization S 2.2.3.1

The out-of-order engine in recent Intel CPUs works roughly as follows.
Micro-ops received from the decode queue are written into a \textit{reorder
buffer} (ROB) while they are \textit{in-flight} in the execution unit. The
\textit{register allocation table} (RAT) matches each register with the last
reorder buffer entry that updates it. The \textit{renamer} uses the RAT to
rewrite the source and destination fields of micro-ops when they are written in
the ROB, as illustrated in Tables \ref{fig:out_of_order_rob} and
\ref{fig:out_of_order_rat}. Note that the ROB representation makes it easy to
determine the dependencies between micro-ops.

\begin{table}[hbt]
  \centering
  \begin{tabular}{| l | l | l | l | l |}
  \hline
  \textbf{\#} & \textbf{Op} & \textbf{Source 1} & \textbf{Source 2} &
  \textbf{Destination}\\
  \hline
  1 & LOAD & RSI & $\emptyset$ & RAX \\
  \hline
  2 & OR & RDI & ROB \#1 & RSI \\
  \hline
  3 & ADD & RSI & RCX & RSI \\
  \hline
  4 & SUB & ROB \# 3 & RDX & RBX \\
  \hline
  \end{tabular}
  \caption{
    Data written by the renamer into the reorder buffer (ROB), for the
    micro-ops in Table~\ref{fig:out_of_order_micro_ops}.
  }
  \label{fig:out_of_order_rob}
\end{table}

\begin{table}[hbt]
  \centering
  \begin{tabular}{| l | r | r | r | r | r | r |}
  \hline
  \textbf{Register} & RAX & RBX & RCX & RDX & RSI & RDI \\
  \hline
  \textbf{ROB \#} & \#1 & \#4 & $\emptyset$ & $\emptyset$ & \#3 & \#2 \\
  \hline
  \end{tabular}
  \caption{
    Relevant entries of the register allocation table after the micro-ops in
    Table~\ref{fig:out_of_order_micro_ops} are inserted into the ROB.
  }
  \label{fig:out_of_order_rat}
\end{table}

% Scheduler: Optimization S 2.2.3.2

The scheduler decides which micro-ops in the ROB get executed, and places them
in the \textit{reservation station}. The reservation station has one port
for each functional unit that can execute micro-ops independently. Each
reservation station port port holds one micro-op from the ROB. The reservation
station port waits until the micro-op's dependencies are satisfied and forwards
the micro-op to the functional unit. When the functional unit completes
executing the micro-op, its result is \textit{written back} to the ROB, and
forwarded to any other reservation station port that depends on it.

The ROB stores the results of completed micro-ops until they are
\textit{retired}, meaning that the results are \textit{committed} to the
register file and the micro-ops are removed from the ROB. Although micro-ops
can be executed out-of-order, they must be retired in program order, in order
to handle exceptions correctly. When a micro-op causes a hardware exception
(\S~\ref{sec:faults}), all the following micro-ops in the ROB are
\textit{squashed}, and their results are discarded.

In the example above, the \texttt{ADD} can complete before the \texttt{LOAD},
because it does not require a memory access. However, the \textit{ADD}'s result
cannot be committed before \texttt{LOAD} completes. Otherwise, if the
\textit{ADD} is committed and the \textit{LOAD} causes a page fault, software
will observe an incorrect value for the  RSI register.

% Load and Store Operation Overview: Optimization S 2.2.5

The ROB is tailored for discovering register dependencies between micro-ops.
However, micro-ops that execute out-of-order can also have memory dependencies.
For this reason, out-of-order engines have a \textit{load buffer} and a
\textit{store buffer} that keep track of in-flight memory operations and are
used to resolve memory dependencies.

Out-of-order execution can introduce noise in cache timing attacks. First,
memory accesses may not be performed in program order, which can impact the
lines selected by the cache eviction algorithms. Second, out-of-order execution
may result in cache fills that do not correspond to executed instructions. For
example, a load that follows a faulting instruction may be scheduled and
executed before the fault is detected.


\subsubsection{Speculative Execution}

% Branch Prediction: Optimization S 2.2.2.3

Branch instructions, also called \textit{branches}, change the instruction
pointer (RIP, \S~\ref{sec:registers}), if a condition is met (\textit{the
branch is taken}). They implement conditional statements (\texttt{if}) and
looping statements, such as \texttt{while} and \texttt{for}. The most
well-known branching instructions in the Intel architecture are in the
\texttt{j\textit{cc}} family, such as \texttt{je} (jump if equal).

Branches pose a challenge to the decode stage, because the instruction that
should be fetched after a branch is not known until the branching condition is
evaluated. In order to avoid stalling the decode stage, modern CPU designs
include \textit{branch predictors} that use historical information to guess
whether a branch will be taken or not.

When the decode stage encounters a branch instruction, it asks the branch
predictor for a guess as to whether the branch will be taken or not. The
decode stage bundles the branch condition and the predictor's guess into a
branch check micro-op, and then continues decoding on the path indicated by the
predictor. The micro-ops following the branch check are marked as
\textit{speculative}.

When the branch check micro-op is executed, the branch unit checks whether the
branch predictor's guess was correct. If that is the case, the branch check is
retired successfully. The scheduler handles \textit{mispredictions} by
squashing all the micro-ops following the branch check, and by signaling the
instruction decoder to flush the micro-op decode queue and start fetching the
instructions that follow the correct branch.

Cache timing attacks must account for speculative execution, as mispredicted
memory accesses can still cause cache fills. Therefore, the attacker may
observe cache fills that don't correspond to instructions that were actually
executed by the victim software.

% Data Prefetching: Optimization S 2.2.5.4

Modern CPUs also attempt to predict memory read patterns, so they can
\textit{prefetch} the memory locations that are about to be read into the
cache. Prefetching minimizes the latency of successfully predicted read
operations, as their data will already be cached. This is accomplished by
exposing circuits called prefetchers to memory accesses and cache misses. Each
prefetcher can recognize a particular access pattern, such as squentially
reading an array's elements. When memory accesses match the pattern that a
prefetcher was built to recognize, the prefetcher loads the cache line
corresponding to the next memory access in its pattern.

Prefetching introduces noise in cache timing attacks, as the attacker may
observe cache fills that don't correspond to instructions in the victim code,
even when accounting for speculative execution.

\subsection{Caching and Memory-Mapped Devices}
\label{sec:memory_io}

The Intel architecture provides a few methods for specifying the cache behavior
for various ranges of the memory address space (\S~\ref{sec:address_spaces}).
This is used for memory mapped to devices, such as a graphics unit's
framebuffer. Caching framebuffer memory is indesirable, because the delay
between the time when a write is issued and the time when the corresponding
cache lines are evicted and written back to memory could lead to visual
artifacts on the user's display. This section covers the cacheability
mechanisms presented in the SDM, because understanding them is important for
analyzing SGX.


\subsubsection{Caching Behaviors}
\label{sec:cacheability_options}

% Memory-Mapped I/O: SDM vol1 S 16.3.1
% Methods of Caching Available: SDM S 11.3
% Ordering I/O: SDM vol1 S 16.6

\textit{Uncacheable} (UC) memory has the same semantics the I/O address space
(\S~\ref{sec:address_spaces}). UC memory is useful when a device's behavior is
dependent on the order of memory reads and writes, such as in the case of
memory-mapped command and data registers for a PCIe NIC
(\S~\ref{sec:motherboard}). The out of order execution engine
(\S~\ref{sec:out_of_order}) does not reorder UC memory accesses, and does not
issue speculative reads to UC memory.

\textit{Write Combining} (WC) memory addresses the specific needs of
framebuffers. WC memory is similar to UC memory, but the out of order engine
may reorder memory accesses, and may perform speculative reads. The processor
stores writes to WC memory in a write combining buffer, and attempts to group
multiple writes into a (more efficient) line write bus transaction.

\textit{Write Through} (WT) memory is cached, but write misses do not cause
cache fills. This is useful for preventing large memory-mapped device memories
that are rarely read, such as framebuffers, from taking up cache memory. WT
memory is covered by the cache coherence engine, may receive speculative reads,
and is subject to operation reordering.

DRAM is represented as \textit{Write Back} (WB) memory, which is optimized
under the assumption that all the devices that need to observe the memory
operations participate in the cache coherence protocol. WB memory is cached as
described in \S~\ref{sec:caching}, receives speculative reads, and operations
targeting it are subject to reordering.

\textit{Write Protected} (WP) memory is similar to WB memory, with the
exception that every write is propagated to the system bus. It is intended for
memory-mapped buffers, where the order of operations does not matter, but the
devices that need to observe the writes cannot participate in the cache
coherence protocol.


\subsubsection{Memory Caching Configuration}
\label{sec:cacheability_config}

% Cache Control Registers and Bits: SDM S 11.5.1
% Memory Type Range Registers (MTRRs): SDM S 11.11
% PCD and PWT Flags: SDM S 20.30.2

On recent Intel processors, the cache's behavior is mainly configured by the
\textit{Memory Type Range Registers} (MTRRs) and by
\textit{Page Attribute Table} (PAT) indices in the page tables
(\S~\ref{sec:paging}). The behavior is also impacted by the Cache Disable (CD)
and Not-Write through (NW) bits in Control Register 0
(CR0, \S~\ref{sec:address_spaces}), as well as by equivalent bits in page table
entries, namely Page-level Cache Disable (PCD) and Page-level Write-Through
(PWT).

The MTRRs were intended to be configurd by the computer's firmware during the
boot sequence. Fixed MTRRs cover pre-determined ranges of memory, such as the
areas that had special meaning in the computers using 16-bit Intel processors.
The ranges covered by \textit{variable MTRRs} can be configured by system
software. We describe the method used to specify these ranges, because the
principles behind it are also used by SGX.

% Variable Range MTRRs: SDM S 11.11.2.3
% Range Size and Alignment Requirement: SDM S 11.11.4

Each MTRR range is specified using a \textit{range base} and a
\textit{range mask}. A memory address belongs to the range if computing a
bitwise AND between the address and the range mask results in the range base.
This verification has a low-cost hardware implementation, shown in
Figure~\ref{fig:mtrr_match}, and can be used to express any range whose size is
a power of two, and whose starting address is a multiple of its size. A range's
starting address is its base, and the range's size is one plus its mask.

\begin{figure}[hbt]
  \center{\includegraphics[width=85mm]{figures/mtrr_match.pdf}}
  \caption{
    The circuit for computing whether a physical address matches an MTRR range.
    Assuming a CPU with 48-bit physical addresses, the circuit uses 36 AND
    gates and a binary tree of 35 XNOR (equality test) gates. The circuit
    outputs 1 if the address belongs to the MTRR range. The bottom 12 address
    bits are ignored, because MTRR ranges must be aligned to 4kb page
    boundaries.
  }
  \label{fig:mtrr_match}
\end{figure}

The PAT is intended to allow the operating system or hypervisor to tweak the
caching behaviors specified in the MTRRs by the computer's firmware. The PAT
has 8 entries that specify caching behaviors, and is stored in its entirety in
a MSR. Each page table entry contains a 3-bit index that points to a PAT entry,
so the system software that controls the page tables can specify caching
behavior at a very fine granularity.

The SGX manual states that the EPC (the memory used to store enclave data) can
only be set up as UC or WB. While no further explanation is provided, we assume
that the UC option was provided in order to attempt to mitigate against some
cache-timing attacks.

\subsection{Caches and Address Translation}
\label{sec:tlbs}

Modern system software relies on address translation (\S~\ref{sec:paging}).
This means that all the memory accesses issued by a CPU core use virtual
addresses, which must undergo translation. Caches must know the physical
address for a memory access, to handle aliasing (multiple virtual addresses
pointing to the same physical address) correctly. However, address translation
requires up to 20 memory accesses (see Figure~\ref{fig:vmx_paging}), so it is
impractical to perform a full address translation for every cache access.
Instead, address translation results are cached in the \textit{translation
look-aside buffer} (TLB).

Table~\ref{fig:tlb_timings} shows the levels of the TLB hierarchy. Recent
processors have separate L1 TLBs for instructions and data, and a shared L2
TLB. Each core has its own TLBs (see Figure~\ref{fig:cpu_core}). When a virtual
address is not contained in a core's TLB, the \textit{Page Miss Handler} (PMH)
performs a \textit{page walk} (page table / EPT traversal) to translate the
virtual address, and the result is stored in the TLB.

\begin{table}[hbt]
  \centering
  \begin{tabular}{| l | r | r |}
  \hline
  \textbf{Memory} & \textbf{Entries} & \textbf{Access Time}\\
  \hline
  L1 I-TLB & 128 + 8 = 136 & 1 cycle \\
  \hline
  L1 D-TLB & 64 + 32 + 4 = 100 & 1 cycle \\
  \hline
  L2 TLB & 1536 + 8 = 1544 & 7 cycles \\
  \hline
  Page Tables & $2^{36} \approx 6 \cdot 10^{10} $ & 18 cycles - 200ms \\
  \hline
  \end{tabular}
  \caption{
    Approximate sizes and access times for each level in the TLB hierarchy,
    from \cite{7zip2014haswell}.
  }
  \label{fig:tlb_timings}
\end{table}

% Caching Translations in TLBs: SDM S 4.10.2.2
% Caches for Paging Structures: SDM S 4.10.3.1

In the Intel architecture, the PMH is implemented in hardware, so the TLB is
never directly exposed to software and its implementation details are not
documented.  The SDM does state that each TLB entry contains the physical
address associated with a virtual address, and the metadata needed to resolve a
memory access. For example, the processor needs to check the writable (W) flag
on every write, and issue a General Protection fault (\#GP) if the write
targets a read-only page.  Therefore, the TLB entry for each virtual address
caches the logical-and of all the relevant W flags in the page table structures
leading up to the page.

The TLB is transparent to application software. However, kernels and
hypervisors must make sure that the TLBs do not get out of sync with the page
tables and EPTs. When changing a page table or EPT, the system software must
use the INVLPG instruction to invalidate any TLB entries for the virtual
address whose translation changed. Some instructions \textit{flush the TLBs},
meaning that they invalidate all the TLB entries, as a side-effect.

% Large Page Size Considerations: SDM S 11.11.9

TLB entries also cache the desired caching behavior (\S~\ref{sec:memory_io})
for their pages. This requires system software to flush the corresponding TLB
entries when changing MTRRs or page table entries. In return, the processor
only needs to compute the desired caching behavior during a TLB miss, as
opposed to computing the caching behavior on every memory access.

% Propagation of Paging-Structure Changes to Multiple Processors: SDM S 4.10.5

The TLB is not covered by the cache coherence mechanism described in
\S~\ref{sec:cache_coherence}. Therefore, when modifying a page table or EPT on
a multi-core / multi-processor system, the system software is responsible for
performing a \textit{TLB shootdown}, which consists of stopping all the logical
processors that use the page table / EPT about to be changed, performing the
changes, executing TLB-invalidating instructions on the stopped logical
processors, and then resuming execution on the stopped logical processors.

Address translation constrains the L1 cache design. On Intel processors, the
set index in an L1 cache only uses the address bits that are not impacted by
address translation, so that the L1 set lookup can be done in parallel with the
TLB lookup. This is critical for achieving a low latency when both the L1 TLB
and the L1 cache are hit.

Given a page size $P = 2^{p}$ bytes, the requirement
above translates to $l + s \le p$. In the Intel architecture, $p = 12$, and all
recent processors have 64-byte cache lines ($l = 6$) and 64 sets ($s = 6$) in
the L1 caches, as shown in Figure~\ref{fig:caching_and_paging}.
The L2 and L3 caches are only accessed if the L1 misses, so the physical
address for the memory access is known at that time, and can be used for
indexing.

\begin{figure}[hbt]
  \centering
  \includegraphics[width=87mm]{figures/caching_and_paging.pdf}
  \caption{
    Virtual addresses from the perspective of cache lookup and address
    translation. The bits used for the L1 set index and line offset are not
    changed by address translation, so the page tables do not impact L1 cache
    placement. Page tables do impact L2 and L3 cache placement. Using large
    pages (2~MB or 1~GB) makes cache placement independent of page tables.
  }
  \label{fig:caching_and_paging}
\end{figure}

\subsection{Hardware Interrupts}
\label{sec:interrupts}

\subsection{The Boot Process}
\label{sec:booting}

When a computer is powered up, it undergoes a \textit{bootstrapping} process,
also called \textit{booting}, for simplicity. Although many steps in the boot
process depend on the motherboard and components in a computer, the process
does follow a high-level structure that is prescribed in the SDM. This section
provides the details needed to analyze SGX's security properties.
\cite{intel2010booting} provides a good reference on the entire booting
process.

\subsubsection{CPU Hardware Initialization}
\label{sec:cpu_init}

% Initialization Overview: SDM S 9.1

Right after a computer is powered up, all the logical processors (LPs) on the
motherboard undergo \textit{hardware initialization}, which invalidates the
caches (\S~\ref{sec:caching}) and TLBs (\S~\ref{sec:tlbs}), performs a
\textit{Built-In Self Test} (BIST), and sets all the registers
(\S~\ref{sec:registers}) to pre-specified values.

% Multiple-Processor Initialization: SDM S 8.4
% BSP and AP Processors: SDM S 8.4.1
% MP Initialization Protocol Algorithms for MP Systems: SDM S 8.4.3
% An ivy bridge CPUID: family 06h, extended model 3, model 58, stepping 9

After hardware initialization, the LPs perform the Multi-Processor (MP)
initialization algorithm, which results in one LP being selected as the
\textit{bootstrap processor} (BSP), and all the other LPs being classified as
\textit{application processors} (APs).

According to the SDM, the details of the MP initialization algorithm for recent
CPUs depend on the motherboard and firmware. In principle, after completing
hardware initialization, all LPs attempt to issue a special no-op transaction
on the QPI bus. A single LP will suceed in issuing the no-op, thanks to
the QPI arbitration mechanism, and to the UBox (\S~\ref{sec:cache_coherence})
in each CPU package, which also serves as a ring arbiter. The arbitration
priority of each LP is based on its APIC ID APIC ID (\S~\ref{sec:interrupts}),
which is provided by the motherboard when the system powers up. The LP that
issues the no-op becomes the BSP. Upon failing to issue the no-op, the other
LPs become APs, and enter the \textit{wait-for-SIPI} state.

% Typical BSP Initialization Sequence: SDM S 8.4.4.1

The BSP sets its RIP register to point to the firmware reset code, which must
be present at 0xFFFFFFF0 (16 bytes below the 4 GB mark). This is accomplished
by having the initial SAD (\S~\ref{sec:cache_coherence}) and PCH
(\S~\ref{sec:motherboard}) configurations map the 4 KB below the 4 GB mark of
the memory address space (\S~\ref{sec:address_spaces}) to the SPI flash chip
that stores the motherboard's firmware.

\subsubsection{Boot Firmware}
\label{sec:firmware_boot}

\cite{intel2010booting} and \cite{coreboot2015manual} describe the
initialization steps performed by the firmware, from an implementor's
perspective. A few steps are interesting from the perspective of SGX and
caching attacks.

% Preventing Caching: SDM S 11.5.3

When the BSP starts executing firmware code, DRAM is not available. The
firmware places the BSP in \textit{Cache-as-RAM} (CAR) mode to be able to use a
call stack and other high-level constructs. Ater CAR is enabled, the memory
initialization code, which is typically Intel's \textit{Memory Reference Code}
(MRC), is loaded into the cache. When executed, the memory initialization code
discovers the DRAM chips connected to the motherboard and sets them up, and
enables and configures the memory controllers.

After DRAM becomes available, the boot firmware is copied into DRAM and the BSP
is taken out of CAR mode. The BSP's LAPIC is initialized and used to send a
broadcast \textit{Startup Inter-Processor Interrupt} (SIPI) to wake up the
APs. The interrupt vector in a SIPI indicates the DRAM address of the firmware
code for initializing APs.

% Typical AP Initialization Sequence: SDM S 8.4.4.2



\subsection{CPU Microcode}
\label{sec:microcode}

Intel's SGX patents disclose that all the SGX features, except for DRAM
encryption, were implemented as microcode extensions. The limitations of
microcode can explain seemingly arbitrary decisions in the SGX design, and a
thorough understanding is crucial to evaluating the feasibility of SGX
modification proposals. The first sub-section below presents the relevant facts
pertaining to microcode in Intel's optimization reference
\cite{intel2014optimization} and SDM. The following subsections summarize
information gleamed from Intel's patents and other researchers' findings.


\subsubsection{Official Information}
\label{sec:microcode_official}

% Intel® Microarchitecture Code Name Sandy Bridge Pipeline Overview:
%     Optimization S 2.2.1
% The Front End: Optimization S 2.2.2

The x86 architecture defines a \textit{complex instruction set} (CISC).
However, virtually all modern CPUs are architected following \textit{reduced
instruction set} (RISC) principles. This is accomplished by having the
instruction decode stage (see Figure~\ref{fig:cpu_core}) break down each x86
instruction into \textit{micro-ops} for every instruction. The other CPU stages
work exclusively with micro-ops.

% Legacy Decode Pipeline (Instruction Decode): Optimization S 2.2.2.1
% Instruction Decode: Optimization S 2.3.2.4
% Front End Overview: Optimization S 2.4.2

The majority of x86 instructions are handled by the hardware decoding path,
which can emit at most 4 micro-ops per instruction. Complex instructions use a
slower decoding path that reads micro-ops from a \textit{microcode store ROM}
(MSROM).

% Microcode Update Facilities: SDM S 9.11
% Responsibilities of the BIOS: SDM 9.11.8.1

Modern Intel processors implement a microcode update facility. The SDM
describes microcode updates from the perspective of an OS kernel and
hypervisor. Each core can be updated independently, and the updates must be
re-applied on each boot cycle. A core can be updated multiple times, but each
update must have a bigger version than the core's current version. The current
SDM version at the time of this writing indicates that a microcode update is
up to 16 kB in size.

The update facility increases the attractiveness of developing architectural
features as microcode extensions \cite{intel2008genetic, intel2012clusters}.
The SGX enclave measurements produced by the processor include the microcode
version, hinting that the SGX designers anticipated the need to use microcode
updates.

\subsubsection{Organization}


% Microcode handles exceptions:
%   US 5,987,600 - 2:39-57, 4:13-27, 4:39-53, 4:65-5:6, 8:42-58, 10:54-60,
%                  11:18-42, 12:11-17, 12:54-58, 15:46-48, 15:59-62
%   US 5,889,982 - 11:40-42, 11:44-46,

% Microcode handles memory exceptions (#PF):
%   US 5,987,600 - 14:26-49, 14:55-61, 14:66-15:3
%   US 5,680,565 - 11:29-37,
%   US 5,889,982 - 14:41-43, 15:47-51,

% Microcode handles DTLB and PMH exceptions:
%   US 5,564,111 - Abstract 15-21, 1:46-59, 3:25-45, 7:47-53, 9:33-51,
%                  10:45-54, 10:57-63

% Microcode performs assisted PMH walk
%   US 5,680,565 - Abstract 1-2 and last 3 lines, 4:9-19, 4:22-28, 12:24-25,
%                  13:42-44, 13:48-54, 13:59-64, 14:12-21, 14:23-29, 14:61-66,
%                  15:1-12, 15:16-39

% Microcode handles events (exceptions and assists):
%   US 5.889,982 - 9:23-25, 9:34-42, 15:7-11, 15:27-55, 16:34-38, 16:57-17:3

% Microcode handles traps:
%   US 5,987,600 - 15:16-18, 15:36-40

% Microcode handles interrupts:
%   US 5,987,600 - 16:2-5, 16:18-21

% Microcode implementation details:
%   US 5,987,600 - 5:39-49, 5:53-6:32, 5:35-39, 5:42-53, 11:53-60, 11:64-67,
%                  12:6-10, 12:41-45, 14:15-19
%   US 5,680,565 - 2:53-56
%   US 5,889,982 - 6:49-65, 7:8-12, 10:11-14, 13:16-20,

% PMH implementation (stuffed loads)
%   US 5,680,565 - 2:60-3:3, 3:25-28, 3:33-52, 3:56, 3:58-4:4, 11:17-21,
%                  11:45-48, 11:50-52, 12:30-34, 12:20-22, 12:40-43, 13:20-22,
%                  14:42-58, 15:54-57

% DTLB implementation
%   US 5,564,111 - 1:26-29, 1:36-38, 3:7-21, 3:58-60, 5:36-41, 5:48-57,
%                  6:51-52, 6:55-7:7, 7:16-18, 7:23-24, 8:3-8, 8:39-40,
%                  9:66-10:4, 10:16-23

% Micro-ops table
%   US 7,451,121 - 1:23-25, 1:34-35, 2:64-65
%   US 8,099,587 - 3:1

% Event ROM
%   US 5,889,982 - 16:57-63, 16:66-17:3

% Microcode compression
%   US 7,451,121 - Abstract 1 and 10
%   US 8,099,587 - Abstract 1-3 and 7-10, 8:36-49, 11:10-17

\cite{intel1999exceptions} discloses that when an event (hardware exception or
interrupt) occurs, an event code is dispatched to the MSROM, which provides the
micro-ops that handle the event.


\cite{intel1997pmh} describes the operation of the PMH (\S~\ref{sec:tlbs}), and
discloses that the PMH uses a microcode assist when it needs to set the dirty
or accessed bits in a page table (\S~\ref{sec:paging}).

\cite{intel1999events} discloses that the microcode has an event ROM which
contains pointers to event handlers in the micro-ops table. The event ROM is
indexed by 6-bit event codes. The first 16 events are hardware exceptions, and
the others are microcode assists. \cite{intel1999events} explicitly mentions
page faults (\#PF) as an example of an exception, and the PMH-issued microcode
assist used to set accessed and dirty bits in the page tables.

\cite{intel1996dtlb} confirms that microcode is used to handle faults and
assists generated by the TLB (\S~\ref{sec:tlbs}) and PMH.

\cite{intel2008genetic} and \cite{intel2012clusters} disclose that the MSROM
contains a micro-ops table. \cite{intel2012clusters} states that tables have on
the order of 20,000 entries, and a micro-op has about 70 bits. Microcode may be
partially compressed.



\cite{hawkes2012microcode} used fault injection and timing analysis to conclude
that each recent Intel microcode update is signed with a 2048-bit RSA key and
a (possibly non-standard) 256-bit hash algorithm. This implies that Intel
already has a microcode implementation of RSA-2048 signature checking, which
may explain why SGX uses RSA signatures in its enclave structures.

\cite{chen2014microcode} sets out to analyze the structure of microcode used in
all x86 processors, but is unable to obtain any details about Intel's
microcode. Fortunately, even though the microcode structure is undocumented,
the 4 micro-ops limitation can be used to guess intelligently whether an
architectural feature is implemented in microcode. For example, it is safe to
assume that \texttt{XSAVE} (\S~\ref{sec:registers}), which was takes over 200
micro-ops on recent CPUs \cite{fog2014microops}, is most likely performed in
microcode, whereas simple arithmetic and memory access is handled directly by
hardware.

While Intel publishes the latest microcode versions for its CPUs, the release
notes associated with the updates are not publicly available. This is
unfortunate, as the release notes could be used to confirm guesses that certain
features are implemented in microcode. However, some information can be
inferred by reading through the Errata section in Intel's Specification Updates
\cite{intel2010errata, intel2015errata, intel2015errata2}. The phrase ``it is
possible for BIOS\footnote{Basic Input/Output System (BIOS)
is the predecessor of UEFI-based firmware. Most Intel documentation, including
the SDM, still uses the term BIOS to refer to firmware.} to contain a
workaround for this erratum'' generally means that a microcode update was
issued. For example, Errata AH in \cite{intel2010errata} implies that string
instructions (\texttt{REP MOV}) are implemented in microcode, which was
confirmed by Intel \cite{abraham2006repmov}.

Errata AH43 and AH91 in \cite{intel2010errata}, and AAK73 in
\cite{intel2015errata} imply that address translation (\S~\ref{sec:paging}) is
at least partially implemented in microcode. Errata AAK53, AAK63, and AAK70,
AAK178 in \cite{intel2015errata}, and BT138, BT210,  in \cite{intel2015errata2}
imply that VM entries and exits (\S~\ref{sec:faults}) are implemented in
microcode.

