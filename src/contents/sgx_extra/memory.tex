\subsection{SGX Memory Organization}
\label{sec:memory}

This section provides an overview of the memory model and  data structures used
by an enclave.




\subsubsection{The Enclave Page Cache Map (EPCM)}

% Enclave Page Cache Map (EPCM): SDM S 37.5.1, SDM S 38.19
% SECINFO.FLAGS: SDM S 38.11.1
% PAGE_TYPE Field Definition: SDM S 38.11.2

SGX relies on the system software to set up each enclave's page tables
according to the enclave developer's design, but does not trust the system
software to do so. An SGX-enabled CPU prevents direct active address
translation attacks using the \textit{Enclave Page Cache Map}~(EPCM), which has
an entry containing the security metadata shown in
Table~\ref{fig:sgx_epcm_entry}
for each EPC page.



% Access Control Requirements: SDM S 38.3

SGX's main weapon against memory mapping attacks is the ENCLAVEADDRESS metadata
field, which contains the expected virtual address (\S~\ref{sec:segments}) used
to access the page. The expected virtual address must be specified when a page
is allocated, and cannot be changed until the page is freed.



\subsubsection{The Implementation of EPC Protection}

The memory controller is
integrated on the CPU die (see Figure~\ref{fig:cpu_die}), so it can be trusted
to prevent devices attached to the system bus from performing DMA transfers
to/from the PRM.

System software manages physical memory by directly modifying the contents of
page tables and EPTs (\S~\ref{sec:paging}), and is responsible for performing
TLB shootdowns (\S~\ref{sec:tlbs}) to ensure that the state not covered by
cache coherence \S~\ref{sec:cache_coherence} is synchronized across logical
processors. If the system software does not perform TLB shootdowns correctly,
application software can experience inconsistent views of memory.

In the context of SGX, an incorrect TLB shootdown can can result in having an
EPC page simultaneously accessible by two different enclaves, which would
compromise the SGX security guarantees. Therefore, the SGX instructions used
for EPC management ensure that the system software performs TLB shootdowns for
the entries that represent EPC pages.


% PRMRR documented in HASP papers and both SGX manuals, completely removed from
% SDM. It still exists in Coreboot. Couldn't find other Skywell code.
The SGX manual states that the EPC (the memory used to store enclave data) can
only be set up as UC or WB. While no further explanation is provided, we assume
that the UC option was provided in order to attempt to mitigate against some
cache-timing attacks.


Having ELRANGE follow the memory type range constraints provides a cheap way to
verify if a virtual address belongs to the address, in hardware. This comes in
handy when EENTER has to disable all the hardware breakpoints inside ELRANGE.


% Access Control Requirements: SDM S 38.3

Pages that store key SGX structures cannot be accessed directly, even by the
code executing inside their enclaves. Furthermore, the SGX instructions that
operate on SGX data structures check the EPCM type fields of their inputs
against the expected types. This type system prevents software from
intentionally or accidentally corrupting the key SGX data structures.
